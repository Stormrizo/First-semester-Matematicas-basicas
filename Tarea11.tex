\documentclass[12pt]{article} 
\usepackage[utf8]{inputenc}
\usepackage[spanish]{babel}
\usepackage{amsfonts}
\usepackage{amsmath, amsthm, amssymb}
\usepackage{graphicx}
\usepackage[left=2.54cm,right=2.54cm,top=2.54cm,bottom=2.54cm]{geometry}
\usepackage{pstricks}
\begin{document}

\thispagestyle{empty} 
\begin{center} \LARGE{\bf Benemérita Universidad Autónoma de Puebla} \\[0.5cm]
\begin{figure}[htb] \centering \includegraphics[scale=.2]{LogoBUAPpng.png} \end{figure}
\LARGE{Facultad de Ciencias Físico Matemáticas}\\[0.5cm]
\begin{figure}[htb] \centering \includegraphics[scale=.39]{LogoFCFMBUAP.png} \end{figure} 
\Large{Licenciatura en Física Teórica}\\[0.5cm]
\large{Primer semestre} \end{center}
\begin{center} { \Large \bfseries{Tarea 11: Conjunto Potencia o Partes de A }} \\ \end{center}
\large{\bf Curso: } Matemáticas básicas \textbf{(N.R.C.:25598)}\\
\large{\bf Alumno:} Julio Alfredo Ballinas García $\left(202107583\right)$ \\
\large{\bf Docente:} Dra. María Araceli Juárez Ramírez\\
\large{\bf Grupo:} 102\\ \begin{center} 
\vfill
\textsc{14 de septiembre de 2021} \end{center}  
\newpage
\sffamily
{\LARGE{\sffamily Conjunto Potencia o Partes de A\\}}

\begin{enumerate}
    \item [I.] Dados los conjuntos $A = \{x\in\mathbb{N}\mid x=3k \wedge2<x<20\}$, \\
    $B = \{x\in Alfabeto\mid x $ es vocal $ \}$
\end{enumerate}

\begin{enumerate}
    \item Halla $n(A)$ y $n(B)$.
    \item Hallar el conjunto $P(A)$ y $P(B)$, así como la cardinalidad de cada uno.\\
\end{enumerate}

\begin{enumerate}
    \item [II.] Dado $C = \{x\in B \mid x $ es vocal débil$\}$
\end{enumerate}

\begin{enumerate}
    \item [3.] Hallar $P(C)$ y $P(P(C))$, sin hallarlo concretamente, diga cuantos tendría $P(P(P(C)))$.\\ \\ \\
\end{enumerate}

La solución a estos problemas es la siguiente:
\newpage
\section{\sffamily{Halla $n(A)$ y $n(B)$}}
\begin{figure}[htb] \centering \includegraphics[scale=.21]{1-1.jpg} \end{figure}
\newpage

\section{\sffamily{Hallar el conjunto $P(A)$ y $P(B)$, así como la cardinalidad de cada uno.}}

\begin{figure}[htb] \centering \includegraphics[scale=.21]{2-1.jpg} \end{figure}
\newpage

\begin{figure}[htb] \centering \includegraphics[scale=.21]{3-1.jpg} \end{figure}
\newpage

\section{\sffamily{Hallar $P(C)$ y $P(P(C))$, sin hallarlo concretamente, diga cuantos tendría $P(P(P(C)))$.}}

\begin{figure}[htb] \centering \includegraphics[scale=.21]{4-1.jpg} \end{figure}
\newpage

\end{document}
