\documentclass[12pt]{article} 
\usepackage[left=2.54cm,right=2.54cm,top=2.54cm,bottom=2.54cm]{geometry}
\usepackage[utf8]{inputenc}
\usepackage[spanish]{babel}
\usepackage{pdfpages}
\usepackage{soul}
\usepackage{csquotes}
\usepackage{afterpage}
\usepackage{parskip}
\usepackage{float}
\usepackage{enumitem}
\usepackage{multicol}
\newenvironment{Figura}
  {\par\medskip\noindent\minipage{\linewidth}}
  {\endminipage\par\medskip}
\usepackage{caption}
\usepackage{amsfonts}
\usepackage{amsmath, amsthm, amssymb}
\renewcommand{\qedsymbol}{$\blacksquare$}
\usepackage{graphicx}
\usepackage{pstricks} 

\usepackage{xcolor}
\definecolor{prussianblue}{RGB}{1, 45, 75} 
\definecolor{brightturquoise}{RGB}{1, 196, 254} 
\definecolor{Aguamarina}{rgb}{0.5, 1.0, 0.83}
\definecolor{mandarina_atomica}{rgb}{1.0, 0.6, 0.4}
\definecolor{blizzardblue}{rgb}{0.67, 0.9, 0.93}
\definecolor{Ebony Clay}{RGB}{35, 44, 67}
\definecolor{Tuscany}{RGB}{205, 111, 52}
\definecolor{prussianblue}{RGB}{1, 45, 75} 
\definecolor{brightturquoise}{RGB}{1, 196, 254} 
\definecolor{Apple Green}{RGB}{125, 191, 3}
\definecolor{Aguamarina}{rgb}{0.5, 1.0, 0.83}
\definecolor{mandarina_atomica}{rgb}{1.0, 0.6, 0.4}
\definecolor{blizzardblue}{rgb}{0.67, 0.9, 0.93}
\definecolor{bluegray}{rgb}{0.4, 0.6, 0.8}
\definecolor{coolgrey}{rgb}{0.55, 0.57, 0.67}
\definecolor{tealgreen}{rgb}{0.0, 0.51, 0.5}
\definecolor{ticklemepink}{rgb}{0.99, 0.54, 0.67}
\definecolor{thulianpink}{rgb}{0.87, 0.44, 0.63}
\definecolor{wildwatermelon}{rgb}{0.99, 0.42, 0.52}
\definecolor{wisteria}{rgb}{0.79, 0.63, 0.86}
\definecolor{yellow(munsell)}{rgb}{0.94, 0.8, 0.0}
\definecolor{trueblue}{rgb}{0.0, 0.45, 0.81}	\definecolor{tropicalrainforest}{rgb}{0.0, 0.46, 0.37}
\definecolor{tearose(rose)}{rgb}{0.96, 0.76, 0.76}
\definecolor{antiquefuchsia}{rgb}{0.57, 0.36, 0.51}	\definecolor{bittersweet}{rgb}{1.0, 0.44, 0.37}	\definecolor{carrotorange}{rgb}{0.93, 0.57, 0.13}
\definecolor{cinereous}{rgb}{0.6, 0.51, 0.48}
\definecolor{darkcoral}{rgb}{0.8, 0.36, 0.27}	\definecolor{orange(colorwheel)}{rgb}{1.0, 0.5, 0.0}
\definecolor{palatinateblue}{rgb}{0.15, 0.23, 0.89} \definecolor{pakistangreen}{rgb}{0.0, 0.4, 0.0} 	\definecolor{vividviolet}{rgb}{0.62, 0.0, 1.0} 
\definecolor{tigre}{rgb}{0.88, 0.55, 0.24} 		\definecolor{plum(traditional)}{rgb}{0.56, 0.27, 0.52} 	\definecolor{persianred}{rgb}{0.8, 0.2, 0.2} 	\definecolor{orange(webcolor)}{rgb}{1.0, 0.65, 0.0} 	\definecolor{onyx}{rgb}{0.06, 0.06, 0.06}
\definecolor{blue-violet}{rgb}{0.54, 0.17, 0.89}
\definecolor{byzantine}{rgb}{0.74, 0.2, 0.64}
\definecolor{byzantium}{rgb}{0.44, 0.16, 0.39}
\definecolor{darkmagenta}{rgb}{0.55, 0.0, 0.55} 
\definecolor{Gallery}{RGB}{236, 236, 236} 
\definecolor{darkviolet}{rgb}{0.58, 0.0, 0.83} 	\definecolor{deepmagenta}{rgb}{0.8, 0.0, 0.8}
\definecolor{Mercury}{RGB}{228, 228, 228} 
\definecolor{Alto}{RGB}{220, 220, 220}
\definecolor{Woodsmoke}{RGB}{4, 4, 5} 
\definecolor{Iron}{RGB}{227, 227, 228} 
\definecolor{Bluechill}{RGB}{11, 150, 144}
\definecolor{Deep Sea Green}{RGB}{8, 83, 94}
\definecolor{Sun}{RGB}{251, 175, 17} 
\definecolor{Lochmara}{RGB}{9, 116, 189}  
\definecolor{Green vogue}{RGB}{4, 40, 85}  
\definecolor{Hippie Blue}{RGB}{92, 148, 179}  
\definecolor{Saratoga}{RGB}{85, 100, 19}  
\definecolor{Earls Green}{RGB}{177, 196, 56}  
\definecolor{Cavern Pink}{RGB}{231, 190, 194} 
\definecolor{Tamarillo}{RGB}{155, 23, 33} 
\definecolor{Cinnabar}{RGB}{225, 71, 53} 
\definecolor{Horizon}{RGB}{88, 132, 169} 
\definecolor{Tarawera}{RGB}{6, 48, 70}
\definecolor{Fiery Orange}{RGB}{180, 92, 22}
\definecolor{Lemon Ginger}{RGB}{170, 164, 40}
\definecolor{Burnt Sienna}{RGB}{236, 119, 88}
\definecolor{Milano Red}{RGB}{184, 12, 11}
\definecolor{Lochinvar}{RGB}{36, 142, 137} 
\definecolor{Saffron}{RGB}{242, 190, 48} 
\definecolor{Clairvoyant}{RGB}{48, 4, 60}
\definecolor{Bottle Green}{RGB}{8, 52, 28}
\definecolor{Mahogany}{RGB}{182, 64, 3}
\definecolor{Medium Aquamarine}{RGB}{116, 204, 159}
\newenvironment{MyColorPar}[1]{%
    \leavevmode\color{#1}\ignorespaces%
}{%
}%

\begin{document}

\begingroup
\begin{titlepage}
	\AddToShipoutPicture*{\put(79,350){\includegraphics[scale=.3]{descarga.png}}}
	\noindent
	\vspace{1mm}
\end{titlepage}
\endgroup

\pagestyle{empty} 
\setlength{\parindent}{0pt}
\sffamily

%%%%%%%%%%%%%%%%%%%%%%%%%%%%%%%%%%%%%%%%%%%%%%%%%%%%%%%%%%%%%%%%%%%
%%%%%%%%%%%%%%%%%%%%%%%%%%%%%%%%%%%%%%%%%%%%%%%%%%%%%%%%%%%%%%%%%%%

\begin{center} 

    \LARGE{\bf{\textsf{Benemérita Universidad Autónoma de Puebla}}} \\[0.5cm]
    
\begin{figure}[htb] \centering

    \includegraphics[scale=.25]{LogoBUAPpng.png} 

\end{figure}

%%%%%%%%%%%%%%%%%%%%%%%%%%%%%%%%%%%%%%%%%%%%%%%%%%%%%%%%%%%%%%%%%%%
%%%%%%%%%%%%%%%%%%%%%%%%%%%%%%%%%%%%%%%%%%%%%%%%%%%%%%%%%%%%%%%%%%%

    \LARGE{Facultad de Ciencias Físico Matemáticas}\\[0.5cm]

\begin{figure}[htb] \centering

    \includegraphics[scale=.4]{LogoFCFMBUAP.png} 
    
\end{figure} 

%%%%%%%%%%%%%%%%%%%%%%%%%%%%%%%%%%%%%%%%%%%%%%%%%%%%%%%%%%%%%%%%%%%
%%%%%%%%%%%%%%%%%%%%%%%%%%%%%%%%%%%%%%%%%%%%%%%%%%%%%%%%%%%%%%%%%%%

    \Large{Licenciatura en Física Teórica}\\[0.5cm]
    \Large{Primer semestre} 

\end{center} \vspace{0.3cm}
%%%%%%%%%%%%%%%%%%%%%%%%%%%%%%%%%%%%%%%%%%%%%%%%%%%%%%%%%%%%%%%%%%%
%%%%%%%%%%%%%%%%%%%%%%%%%%%%%%%%%%%%%%%%%%%%%%%%%%%%%%%%%%%%%%%%%%%

\begin{center}

    {\Large{\bfseries{{\textcolor{Mahogany}{Tarea 34}}}}} \\ 
    
\end{center}

    \large{\bf{\textsf{Curso:}}} {\bfseries{{\textcolor{Bottle Green}{Matemáticas básicas \bfseries{(N.R.C.:25598)}}}}} \\
    \large{\bf{\textsf{Alumno:}}} {\bfseries{{\textcolor{prussianblue}{Julio Alfredo Ballinas García {\large{{$\mid$}}} 202107583}}}}  \\
    \large{\bf{\textsf{Docente:}}} {\bfseries{{\textcolor{Clairvoyant}{Dra. María Araceli Juárez Ramírez}}}}\\
    \large{\bf{\textsf{Grupo:}}} {\bfseries{{\textcolor{Apple Green}{102}}}}\\

\vfill
    
\begin{center} 

    {\texttt{\textcolor{Cinnabar}{{\underline{Venció}:}}}} 20 de noviembre $\mid$ {\texttt{\textcolor{Cinnabar}{{\underline{Entregada}:}}}} 25 de noviembre 2021
    
\end{center}

\newpage

\section*{\textsf{Escribir como es la tasa de crecimiento si $f$ es estrictamente decreciente o sólo decreciente}} \vspace{0.5cm}

{\textcolor{Cinnabar}{\bfseries{Solución:}}} \vspace{0.5cm}

Criterio para saber si una función es estrictamente creciente o estrictamente decreciente en $I$ para $x_{1}$ $\neq$ $x_{2}$ \vspace{0.5cm}

{\begin{MyColorPar}{Hippie Blue}
Taza de crecimiento: \vspace{0.5cm}

\hspace{4cm} {\LARGE{$\frac{f(x_{1})-f(x_{2})}{x_{1}-x_{2}}$}} $=$ {\LARGE{$\frac{f(x_{2})-f(x_{1})}{x_{2}-x_{1}}$}} 
\end{MyColorPar}} \vspace{0.5cm} 

{\textcolor{Cinnabar}{$\bullet$}} {\bfseries{Definición:}} Decimos $f:$ $\mathbb{R}$ $\rightarrow$ $\mathbb{R}$ es estrictamente decreciente en $I$, si $\forall x_{1}$,$ x_{2}$ \vspace{0.2cm}

$\in$ $I$ tal que $x_{1}$ $<$ $x_{2}$ $\Longrightarrow$ $f(x_{1})$ $>$ $f(x_{2})$, luego $f(x_{1})$ $-$ $f(x_{2})$ $>$ $0$ o además \vspace{0.2cm}

\mbox{$x_{2}$ $-$ $x_{1}$ $>$ $0$}. \vspace{0.5cm}

Luego la tasa de crecimiento: \vspace{0.5cm}

\hspace{4cm} {\LARGE{$\frac{f(x_{1})-f(x_{2})}{x_{2}-x_{1}}$}} $>$ $0$ \hspace{0.3cm} $\boldsymbol{\longleftarrow}$ {\fbox{Taza de crecimiento de una función}}

\hspace{10cm} {\fbox{estrictamente decreciente}} \vspace{0.5cm}

{\textcolor{Cinnabar}{$\bullet$}} {\bfseries{Definición:}} Decimos $f:$ $\mathbb{R}$ $\rightarrow$ $\mathbb{R}$ es decreciente en $I$, si $\forall x_{1}$,$ x_{2}$ $\in$ $I$ tal que  \vspace{0.2cm}

$x_{1}$ $<$ $x_{2}$ $\Longrightarrow$ $f(x_{1})$ $\geq$ $f(x_{2})$, luego $f(x_{1})$ $-$ $f(x_{2})$ $\geq$ $0$ o además $x_{2}$ $-$ $x_{1}$ $>$ $0$   \vspace{0.5cm}

La tasa de crecimiento es: \vspace{0.5cm}

\hspace{4cm} {\LARGE{$\frac{f(x_{1})-f(x_{2})}{x_{2}-x_{1}}$}} $\geq$ $0$ \hspace{0.3cm} $\boldsymbol{\longleftarrow}$ {\fbox{Taza de crecimiento de una función}}

\hspace{12cm} {\fbox{decreciente}} 

\newpage

\section*{\textsf{Investigar por tasa de crecimiento los intervalos de crecimiento de:}}
 
\hspace{4cm} $h(x)$ $=$ $x^{2}$ $+$ $2x$ $+$ $1$ \vspace{0.5cm}

{\textcolor{Cinnabar}{\bfseries{Solución:}}} \vspace{0.5cm}

\hspace{4cm} $x^{2}$ $+$ $2x$ $+$ $1$ $=$ $0$ \vspace{0.5cm}

Factorizando: \vspace{0.5cm}

\hspace{4cm} $\Longleftrightarrow$ ($x$ $+$ $1$) $\cdot$ ($x$ $+$ $1$) $=$ $0$ \vspace{0.5cm}

Por {\textcolor{carrotorange}{\bfseries{teorema 5 (R$_{5}$)}}}

\hspace{4cm} $\Longleftrightarrow$ $x$ $+$ $1$ $=$ $0$ $\vee$ $x$ $+$ $1$ $=$ $0$ \vspace{0.5cm}

\hspace{4cm} $\Longleftarrow$ $x$ $=$ $1$ \vspace{0.5cm}

Podemos observar que la gráfica conforma una parábola cóncava hacia arriba, con intersección en el eje $x$ cuando $x$ $=$ $-1$ e intersección en el eje $y$ cuando \mbox{$y$ $=$ $1$} 

\begin{figure}[htb] \centering

    \includegraphics[scale=.4]{grafica xd.png} 
    
\end{figure} 

\newpage


Observamos mediante la gráfica de {\textcolor{tropicalrainforest}{\bfseries{Desmos}}} que en el intervalo que comprende el eje $x$ que va de $\big{(}$ $-$ $\infty$, $-$ $1$ $\big{)}$, $h(x)$ es estrictamente decreciente y, por otro parte, en el intervalo que comprende el eje $x$ que va de $\big{[}$ $-$ $1$, $\infty$ $\big{]}$, es estrictamente creciente. \vspace{0.5cm}




