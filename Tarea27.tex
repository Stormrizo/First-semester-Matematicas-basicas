\documentclass[12pt]{article} 
\usepackage[left=2.54cm,right=2.54cm,top=2.54cm,bottom=2.54cm]{geometry}
\usepackage[utf8]{inputenc}
\usepackage[spanish]{babel}
\usepackage{pdfpages}
\usepackage{csquotes}
\usepackage{schemata}
\usepackage{afterpage}
\usepackage{parskip}
\usepackage{float}
\usepackage{enumitem}
\usepackage{multicol}
\newenvironment{Figura}
  {\par\medskip\noindent\minipage{\linewidth}}
  {\endminipage\par\medskip}
\usepackage{caption}
\usepackage{amsfonts}
\usepackage{amsmath, amsthm, amssymb}
\renewcommand{\qedsymbol}{$\blacksquare$}
\usepackage{graphicx}
\usepackage{pstricks} 

\usepackage{xcolor}
\definecolor{prussianblue}{RGB}{1, 45, 75} 
\definecolor{brightturquoise}{RGB}{1, 196, 254} 
\definecolor{verde_manzana}{rgb}{0.55, 0.71, 0.0}
\definecolor{Aguamarina}{rgb}{0.5, 1.0, 0.83}
\definecolor{mandarina_atomica}{rgb}{1.0, 0.6, 0.4}
\definecolor{blizzardblue}{rgb}{0.67, 0.9, 0.93}
\definecolor{bluegray}{rgb}{0.4, 0.6, 0.8}
\definecolor{coolgrey}{rgb}{0.55, 0.57, 0.67}
\definecolor{tealgreen}{rgb}{0.0, 0.51, 0.5}
\definecolor{ticklemepink}{rgb}{0.99, 0.54, 0.67}
\definecolor{thulianpink}{rgb}{0.87, 0.44, 0.63}
\definecolor{wildwatermelon}{rgb}{0.99, 0.42, 0.52}
\definecolor{wisteria}{rgb}{0.79, 0.63, 0.86}
\definecolor{yellow(munsell)}{rgb}{0.94, 0.8, 0.0}
\definecolor{trueblue}{rgb}{0.0, 0.45, 0.81}	\definecolor{tropicalrainforest}{rgb}{0.0, 0.46, 0.37}
\definecolor{tearose(rose)}{rgb}{0.96, 0.76, 0.76}
\definecolor{antiquefuchsia}{rgb}{0.57, 0.36, 0.51}	\definecolor{bittersweet}{rgb}{1.0, 0.44, 0.37}	\definecolor{carrotorange}{rgb}{0.93, 0.57, 0.13}
\definecolor{cinereous}{rgb}{0.6, 0.51, 0.48}
\definecolor{darkcoral}{rgb}{0.8, 0.36, 0.27}	\definecolor{orange(colorwheel)}{rgb}{1.0, 0.5, 0.0}
\definecolor{palatinateblue}{rgb}{0.15, 0.23, 0.89} \definecolor{pakistangreen}{rgb}{0.0, 0.4, 0.0} 	\definecolor{vividviolet}{rgb}{0.62, 0.0, 1.0} 
\definecolor{tigre}{rgb}{0.88, 0.55, 0.24} 		\definecolor{plum(traditional)}{rgb}{0.56, 0.27, 0.52} 	\definecolor{persianred}{rgb}{0.8, 0.2, 0.2} 	\definecolor{orange(webcolor)}{rgb}{1.0, 0.65, 0.0} 	\definecolor{onyx}{rgb}{0.06, 0.06, 0.06}
\definecolor{blue-violet}{rgb}{0.54, 0.17, 0.89}
\definecolor{byzantine}{rgb}{0.74, 0.2, 0.64}
\definecolor{byzantium}{rgb}{0.44, 0.16, 0.39}
\definecolor{darkmagenta}{rgb}{0.55, 0.0, 0.55} 	\definecolor{darkviolet}{rgb}{0.58, 0.0, 0.83} 	\definecolor{deepmagenta}{rgb}{0.8, 0.0, 0.8}
\definecolor{Dark Burgundy}{RGB}{128, 7, 15}
\definecolor{Thunderbird}{RGB}{198, 16, 27}
\definecolor{Terracotta}{RGB}{234, 119, 106}
\definecolor{Totem pole}{RGB}{154, 23, 4}  
\definecolor{Tahiti Gold}{RGB}{226, 138, 6} 
\definecolor{Flame Pea}{RGB}{223, 85, 66}
\definecolor{Boston Blue}{RGB}{62, 145, 163}
\definecolor{Tarawera}{RGB}{6, 48, 70}
\definecolor{Bluechill}{RGB}{11, 150, 144}
\definecolor{Deep Sea Green}{RGB}{8, 83, 94}
\definecolor{Sun}{RGB}{251, 175, 17} 
\definecolor{Lochmara}{RGB}{9, 116, 189}  
\definecolor{Green vogue}{RGB}{4, 40, 85}  
\definecolor{Hippie Blue}{RGB}{92, 148, 179}  
\definecolor{Saratoga}{RGB}{85, 100, 19}  
\definecolor{Earls Green}{RGB}{177, 196, 56}  
\definecolor{Cavern Pink}{RGB}{231, 190, 194} 
\definecolor{Tamarillo}{RGB}{155, 23, 33} 
\definecolor{Cinnabar}{RGB}{225, 71, 53} 
\definecolor{Horizon}{RGB}{88, 132, 169} 
\definecolor{Fiery Orange}{RGB}{180, 92, 22}
\definecolor{Lemon Ginger}{RGB}{170, 164, 40}
\definecolor{Burnt Sienna}{RGB}{236, 119, 88}
\definecolor{Milano Red}{RGB}{184, 12, 11}
\definecolor{Fiord}{RGB}{59, 75, 102}
\newenvironment{MyColorPar}[1]{%
    \leavevmode\color{#1}\ignorespaces%
}{%
}%

\begin{document}

\begingroup
\begin{titlepage}
	\AddToShipoutPicture*{\put(79,350){\includegraphics[scale=.3]{descarga.png}}}
	\noindent
	\vspace{1mm}
\end{titlepage}
\endgroup

\pagestyle{empty} 
\setlength{\parindent}{0pt}
\sffamily

%%%%%%%%%%%%%%%%%%%%%%%%%%%%%%%%%%%%%%%%%%%%%%%%%%%%%%%%%%%%%%%%%%%
%%%%%%%%%%%%%%%%%%%%%%%%%%%%%%%%%%%%%%%%%%%%%%%%%%%%%%%%%%%%%%%%%%%

\begin{center} 

    \LARGE{\bf{\textsf{Benemérita Universidad Autónoma de Puebla}}} \\[0.5cm]
    
\begin{figure}[htb] \centering

    \includegraphics[scale=.25]{LogoBUAPpng.png} 

\end{figure}

%%%%%%%%%%%%%%%%%%%%%%%%%%%%%%%%%%%%%%%%%%%%%%%%%%%%%%%%%%%%%%%%%%%
%%%%%%%%%%%%%%%%%%%%%%%%%%%%%%%%%%%%%%%%%%%%%%%%%%%%%%%%%%%%%%%%%%%

    \LARGE{Facultad de Ciencias Físico Matemáticas}\\[0.5cm]

\begin{figure}[htb] \centering

    \includegraphics[scale=.4]{LogoFCFMBUAP.png} 
    
\end{figure} 

%%%%%%%%%%%%%%%%%%%%%%%%%%%%%%%%%%%%%%%%%%%%%%%%%%%%%%%%%%%%%%%%%%%
%%%%%%%%%%%%%%%%%%%%%%%%%%%%%%%%%%%%%%%%%%%%%%%%%%%%%%%%%%%%%%%%%%%

    \Large{Licenciatura en Física Teórica}\\[0.5cm]
    \Large{Primer semestre} 

\end{center} \vspace{0.3cm}
%%%%%%%%%%%%%%%%%%%%%%%%%%%%%%%%%%%%%%%%%%%%%%%%%%%%%%%%%%%%%%%%%%%
%%%%%%%%%%%%%%%%%%%%%%%%%%%%%%%%%%%%%%%%%%%%%%%%%%%%%%%%%%%%%%%%%%%

\begin{center}

    {\Large{\bfseries{{\textcolor{carrotorange}{Tarea 27 (teorema 39)}}}}} \\ 
    
\end{center}

    \large{\bf{\textsf{Curso:}}} {\bfseries{{\textcolor{brightturquoise}{Matemáticas básicas \bfseries{(N.R.C.:25598)}}}}} \\
    \large{\bf{\textsf{Alumno:}}} {\bfseries{{\textcolor{prussianblue}{Julio Alfredo Ballinas García {\large{{$\mid$}}} 202107583}}}}  \\
    \large{\bf{\textsf{Docente:}}} {\bfseries{{\textcolor{wisteria}{Dra. María Araceli Juárez Ramírez}}}}\\
    \large{\bf{\textsf{Grupo:}}} {\bfseries{{\textcolor{verde_manzana}{102}}}}\\

\vfill
    
\begin{center} 

    {\small{\textsf{\underline{Tarea retrasada:} venció 26 de octubre de 2021 {\red{23:59 PM}}} {\LARGE{ $\mid$ }}\textsf{{\underline{Fecha de hoy:}} 05 de noviembre de 2021}}}

\end{center}

\newpage

\section*{\textsf{Hallar la segunda solución de la ecuación:}}

\hspace{4cm}{\Large{$ax^{2}$ $+$ $bx$ $+$ $c$ $=$ $0$ (Con $a$ $\neq$ $0$)}}

\section*{\textsf{Recordemos que en clase se halló:}}

\hspace{4cm} $x$ $=$ {\LARGE{$-$ $\frac{b}{2a}$ $-$ $\frac{\sqrt{b^{2}-4ac}}{2a}$}} 
\vspace{0.5cm}

\section*{\textsf{Debemos hallar:}}

\hspace{4cm} $x$ $=$ {\LARGE{$-$ $\frac{b}{2a}$ $+$ $\frac{\sqrt{b^{2}-4ac}}{2a}$}} \vspace{0.5cm}

\begin{MyColorPar}{Cinnabar}
\bfseries{\underline{Inicio:}}
\end{MyColorPar} \vspace{0.5cm}

\begin{MyColorPar}{Tarawera}
\bfseries Usamos la ecuación {\black{$x$ $=$ {\LARGE{$-$ $\frac{b}{2a}$ $-$ $\frac{\sqrt{b^{2}-4ac}}{2a}$}}}} como punto de partida \vspace{0.5cm}

para llegar a: {\black{$x$ $=$ {\LARGE{$-$ $\frac{b}{2a}$ $+$ $\frac{\sqrt{b^{2}-4ac}}{2a}$}}}} \vspace{0.5cm}

Sumamos a ambos lados de la ecuación: {\black{$x$ $=$ {\LARGE{$-$ $\frac{b}{2a}$ $-$ $\frac{\sqrt{b^{2}-4ac}}{2a}$}}}} el \vspace{0.5cm} 

conjugado de $\bigg($ {\LARGE{$-$ $\frac{b}{2a}$ $-$ $\frac{\sqrt{b^{2}-4ac}}{2a}$}} $\bigg)$ y este es: {\LARGE{$\frac{b}{2a}$ $+$ $\frac{\sqrt{b^{2}-4ac}}{2a}$}}

Tenemos entonces:
\end{MyColorPar} \vspace{0.5cm}

$x$ $+$ $\bigg($ {\LARGE{$\frac{b}{2a}$ $+$ $\frac{\sqrt{b^{2}-4ac}}{2a}$}} $\bigg)$ $=$ {\LARGE{$-$ $\frac{b}{2a}$ $-$ $\frac{\sqrt{b^{2}-4ac}}{2a}$}} $+$ $\bigg($ {\LARGE{$\frac{b}{2a}$ $+$ $\frac{\sqrt{b^{2}-4ac}}{2a}$}} $\bigg)$ \vspace{0.5cm}

\begin{MyColorPar}{Tarawera}
\bfseries Esto es equivalente a:
\end{MyColorPar} \vspace{0.5cm}

$x$ $+$ $\bigg($ {\LARGE{$\frac{b}{2a}$ $+$ $\frac{\sqrt{b^{2}-4ac}}{2a}$}} $\bigg)$ $=$ {\LARGE{$-$ $\frac{b}{2a}$ $+$ {\LARGE{$\frac{b}{2a}$}} $-$ $\frac{\sqrt{b^{2}-4ac}}{2a}$}} $+$ {\LARGE{ $\frac{\sqrt{b^{2}-4ac}}{2a}$}} 

\newpage

\begin{MyColorPar}{Tarawera}
\bfseries Esto es equivalente a:
\end{MyColorPar} \vspace{0.5cm}

$x$ $+$ $\bigg($ {\LARGE{$\frac{b}{2a}$ $+$ $\frac{\sqrt{b^{2}-4ac}}{2a}$}} $\bigg)$ $=$ $0$ \vspace{0.5cm}

$x$  $=$ {\LARGE{$-$}} $\bigg($ {\LARGE{$\frac{b}{2a}$ $+$ $\frac{\sqrt{b^{2}-4ac}}{2a}$}} $\bigg)$ $+$ $0$ \vspace{0.5cm}

$x$  $=$ {\LARGE{$-$}} $\bigg($ {\LARGE{$\frac{b}{2a}$ $+$ $\frac{\sqrt{b^{2}-4ac}}{2a}$}} $\bigg)$ $+$ $\bigg($ {\LARGE{$\frac{\sqrt{b^{2}-4ac}}{2a}$}} $-$ {\LARGE{$\frac{\sqrt{b^{2}-4ac}}{2a}$}} $\bigg)$ \vspace{0.5cm}

$x$  $=$ {\LARGE{$-$}} {\LARGE{$\frac{b}{2a}$ $-$ $\frac{\sqrt{b^{2}-4ac}}{2a}$}} $+$ $\bigg($ {\LARGE{$\frac{\sqrt{b^{2}-4ac}}{2a}$}} $-$ {\LARGE{$\frac{\sqrt{b^{2}-4ac}}{2a}$}} $\bigg)$ \vspace{0.5cm}

\begin{MyColorPar}{Thunderbird}
\bfseries Solución dos no hallada.
\end{MyColorPar} \vspace{0.2cm}

\begin{MyColorPar}{pakistangreen}
$\bullet$ $\bullet$ $\bullet$ $\bullet$ $\bullet$ $\bullet$ $\bullet$ $\bullet$ $\bullet$ $\bullet$ $\bullet$ $\bullet$ $\bullet$ $\bullet$ $\bullet$ $\bullet$ $\bullet$ $\bullet$ $\bullet$ $\bullet$ $\bullet$ $\bullet$ $\bullet$ $\bullet$ $\bullet$ $\bullet$ $\bullet$ $\bullet$ $\bullet$ $\bullet$ $\bullet$ $\bullet$ $\bullet$ $\bullet$ $\bullet$ $\bullet$ $\bullet$ $\bullet$ $\bullet$ $\bullet$ 
\end{MyColorPar} \vspace{.5cm}

\section*{{\textsf{Caso ii)}}}
\hspace{4cm} {\Large{$b^{2}$ $-$ $4ac$ $=$ $0$}} \vspace{0.5cm}

\hspace{3cm} * $\bigg [$ $x$ $+$ {\LARGE{$\frac{b}{2a}$}} $\pm$ {\LARGE{$\frac{\sqrt{b^{2}-4ac}}{2a}$}} $\bigg ]$ $=$ $0$ \vspace{0.5cm}
 
\hspace{3.5cm} $x$ $+$ {\LARGE{$\frac{b}{2a}$}} $\pm$ {\LARGE{$\frac{\sqrt{0}}{2a}$}} $=$ $0$

\hspace{3.5cm} $x$ $+$ {\LARGE{$\frac{b}{2a}$}} $\pm$ $0$ $=$ $0$

\hspace{3.5cm} $x$ $+$ {\LARGE{$\frac{b}{2a}$}} $=$ $0$

\hspace{3.5cm} $x$ $=$ $-$ {\LARGE{$\frac{b}{2a}$}}

\hspace{3.5cm} $\qedsymbol$

\newpage

\begin{MyColorPar}{pakistangreen}
$\bullet$ $\bullet$ $\bullet$ $\bullet$ $\bullet$ $\bullet$ $\bullet$ $\bullet$ $\bullet$ $\bullet$ $\bullet$ $\bullet$ $\bullet$ $\bullet$ $\bullet$ $\bullet$ $\bullet$ $\bullet$ $\bullet$ $\bullet$ $\bullet$ $\bullet$ $\bullet$ $\bullet$ $\bullet$ $\bullet$ $\bullet$ $\bullet$ $\bullet$ $\bullet$ $\bullet$ $\bullet$ $\bullet$ $\bullet$ $\bullet$ $\bullet$ $\bullet$ $\bullet$ $\bullet$ $\bullet$ 
\end{MyColorPar} 


\vspace{0.5cm}
\section*{{\textsf{Caso iii)}}}
\hspace{4cm} {\Large{$b^{2}$ $-$ $4ac$ $<$ $0$}} \vspace{0.5cm}

\hspace{3cm} * $\bigg [$ $x$ $+$ {\LARGE{$\frac{b}{2a}$}} $\pm$ {\LARGE{$\frac{\sqrt{b^{2}-4ac}}{2a}$}} $\bigg ]$ $=$ $0$ \vspace{0.5cm}

No es posible realizar la operación anterior, dado que no existe manera alguna que demuestre que la operación del radical con respecto a los números negativos nos de como resultado un valor $x$ $\in$ $\mathbb{R}$.

La solución posible es recurrir al uso de los números complejos ($a$ $+$ $bi$), ($x$ $+$ $yi$) o $i$, pero para este ámbito o área no podemos tomarlos en cuenta. 

\hspace{13cm} $\qedsymbol$
\end{document}
