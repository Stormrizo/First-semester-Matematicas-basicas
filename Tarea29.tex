\documentclass[12pt]{article} 
\usepackage[utf8]{inputenc}
\usepackage[spanish]{babel}
\usepackage{pdfpages}
\usepackage{csquotes}
\usepackage{schemata}
\usepackage{afterpage}
\usepackage{parskip}
\usepackage{float}
\usepackage{enumitem}
\usepackage{multicol}
\newenvironment{Figura}
  {\par\medskip\noindent\minipage{\linewidth}}
  {\endminipage\par\medskip}
\usepackage{caption}
\usepackage{amsfonts}
\usepackage{amsmath, amsthm, amssymb}
\renewcommand{\qedsymbol}{$\blacksquare$}
\usepackage{graphicx}
\usepackage[left=2.54cm,right=2.54cm,top=2.54cm,bottom=2.54cm]{geometry}
\usepackage{pstricks} 

\usepackage{xcolor}
\definecolor{prussianblue}{RGB}{1, 45, 75} 
\definecolor{brightturquoise}{RGB}{1, 196, 254} 
\definecolor{verde_manzana}{rgb}{0.55, 0.71, 0.0}
\definecolor{Aguamarina}{rgb}{0.5, 1.0, 0.83}
\definecolor{mandarina_atomica}{rgb}{1.0, 0.6, 0.4}
\definecolor{blizzardblue}{rgb}{0.67, 0.9, 0.93}
\definecolor{bluegray}{rgb}{0.4, 0.6, 0.8}
\definecolor{coolgrey}{rgb}{0.55, 0.57, 0.67}
\definecolor{tealgreen}{rgb}{0.0, 0.51, 0.5}
\definecolor{ticklemepink}{rgb}{0.99, 0.54, 0.67}
\definecolor{thulianpink}{rgb}{0.87, 0.44, 0.63}
\definecolor{wildwatermelon}{rgb}{0.99, 0.42, 0.52}
\definecolor{wisteria}{rgb}{0.79, 0.63, 0.86}
\definecolor{yellow(munsell)}{rgb}{0.94, 0.8, 0.0}
\definecolor{trueblue}{rgb}{0.0, 0.45, 0.81}	\definecolor{tropicalrainforest}{rgb}{0.0, 0.46, 0.37}
\definecolor{tearose(rose)}{rgb}{0.96, 0.76, 0.76}
\definecolor{antiquefuchsia}{rgb}{0.57, 0.36, 0.51}	\definecolor{bittersweet}{rgb}{1.0, 0.44, 0.37}	\definecolor{carrotorange}{rgb}{0.93, 0.57, 0.13}
\definecolor{cinereous}{rgb}{0.6, 0.51, 0.48}
\definecolor{darkcoral}{rgb}{0.8, 0.36, 0.27}	\definecolor{orange(colorwheel)}{rgb}{1.0, 0.5, 0.0}
\definecolor{palatinateblue}{rgb}{0.15, 0.23, 0.89} \definecolor{pakistangreen}{rgb}{0.0, 0.4, 0.0} 	\definecolor{vividviolet}{rgb}{0.62, 0.0, 1.0} 
\definecolor{tigre}{rgb}{0.88, 0.55, 0.24} 		\definecolor{plum(traditional)}{rgb}{0.56, 0.27, 0.52} 	\definecolor{persianred}{rgb}{0.8, 0.2, 0.2} 	\definecolor{orange(webcolor)}{rgb}{1.0, 0.65, 0.0} 	\definecolor{onyx}{rgb}{0.06, 0.06, 0.06}
\definecolor{blue-violet}{rgb}{0.54, 0.17, 0.89}
\definecolor{byzantine}{rgb}{0.74, 0.2, 0.64}
\definecolor{byzantium}{rgb}{0.44, 0.16, 0.39}
\definecolor{darkmagenta}{rgb}{0.55, 0.0, 0.55} 	\definecolor{darkviolet}{rgb}{0.58, 0.0, 0.83} 	\definecolor{deepmagenta}{rgb}{0.8, 0.0, 0.8}
\definecolor{Dark Burgundy}{RGB}{128, 7, 15}
\definecolor{Thunderbird}{RGB}{198, 16, 27}
\definecolor{Terracotta}{RGB}{234, 119, 106}
\definecolor{Totem pole}{RGB}{154, 23, 4}  
\definecolor{Tahiti Gold}{RGB}{226, 138, 6} 
\definecolor{Flame Pea}{RGB}{223, 85, 66}
\definecolor{Boston Blue}{RGB}{62, 145, 163}


\definecolor{Fiord}{RGB}{59, 75, 102}
\newenvironment{MyColorPar}[1]{%
    \leavevmode\color{#1}\ignorespaces%
}{%
}%

\begin{document}

\begingroup
\begin{titlepage}
	\AddToShipoutPicture*{\put(79,350){\includegraphics[scale=.3]{descarga.png}}}
	\noindent
	\vspace{1mm}
\end{titlepage}
\endgroup

\pagestyle{empty} 
\setlength{\parindent}{0pt}
\sffamily

%%%%%%%%%%%%%%%%%%%%%%%%%%%%%%%%%%%%%%%%%%%%%%%%%%%%%%%%%%%%%%%%%%%
%%%%%%%%%%%%%%%%%%%%%%%%%%%%%%%%%%%%%%%%%%%%%%%%%%%%%%%%%%%%%%%%%%%

\begin{center} 

    \LARGE{\bf{\textsf{Benemérita Universidad Autónoma de Puebla}}} \\[0.5cm]
    
\begin{figure}[htb] \centering

    \includegraphics[scale=.25]{LogoBUAPpng.png} 

\end{figure}

%%%%%%%%%%%%%%%%%%%%%%%%%%%%%%%%%%%%%%%%%%%%%%%%%%%%%%%%%%%%%%%%%%%
%%%%%%%%%%%%%%%%%%%%%%%%%%%%%%%%%%%%%%%%%%%%%%%%%%%%%%%%%%%%%%%%%%%

    \LARGE{Facultad de Ciencias Físico Matemáticas}\\[0.5cm]

\begin{figure}[htb] \centering

    \includegraphics[scale=.4]{LogoFCFMBUAP.png} 
    
\end{figure} 

%%%%%%%%%%%%%%%%%%%%%%%%%%%%%%%%%%%%%%%%%%%%%%%%%%%%%%%%%%%%%%%%%%%
%%%%%%%%%%%%%%%%%%%%%%%%%%%%%%%%%%%%%%%%%%%%%%%%%%%%%%%%%%%%%%%%%%%

    \Large{Licenciatura en Física Teórica}\\[0.5cm]
    \Large{Primer semestre} 

\end{center} \vspace{0.3cm}
%%%%%%%%%%%%%%%%%%%%%%%%%%%%%%%%%%%%%%%%%%%%%%%%%%%%%%%%%%%%%%%%%%%
%%%%%%%%%%%%%%%%%%%%%%%%%%%%%%%%%%%%%%%%%%%%%%%%%%%%%%%%%%%%%%%%%%%

\begin{center}

    {\Large{\bfseries{{\textcolor{carrotorange}{Tarea 29 (inducción)}}}}} \\ 
    
\end{center}

    \large{\bf{\textsf{Curso:}}} {\bfseries{{\textcolor{brightturquoise}{Matemáticas básicas \bfseries{(N.R.C.:25598)}}}}} \\
    \large{\bf{\textsf{Alumno:}}} {\bfseries{{\textcolor{prussianblue}{Julio Alfredo Ballinas García {\large{{$\mid$}}} 202107583}}}}  \\
    \large{\bf{\textsf{Docente:}}} {\bfseries{{\textcolor{wisteria}{Dra. María Araceli Juárez Ramírez}}}}\\
    \large{\bf{\textsf{Grupo:}}} {\bfseries{{\textcolor{verde_manzana}{102}}}}\\

\vfill
    
\begin{center} 

    {\small{\textsf{\underline{Tarea retrasada:} venció 01 de noviembre {\red{23:59 PM}}} {\LARGE{ $\mid$ }}\textsf{{\underline{Fecha de hoy:}} 02 de noviembre}}}
    
\end{center}

\newpage

%%%%%%%%%%%%%%%%%%%%%%%%%%%%%%%%%%%%%%%%%%%%%%%%%%%%%%%%%%%%%%%%%%%
%%%%%%%%%%%%%%%%%%%%%%%%%%%%%%%%%%%%%%%%%%%%%%%%%%%%%%%%%%%%%%%%%%%

\section*{Tarea: Mostrar por inducción los siguientes incisos.} \vspace{0.2cm}

\begin{enumerate}[label=(\alph*)]
    \item 1 $+$ 3 $+$ 5 $+$ $...$ $+$ (2n $-$ 1) $=$ n$^{2}$ \hspace{0.3cm} {\textcolor{Tahiti Gold}{Esta suma genera los números cuadrados.}} \vspace{0.2cm}
    
    \item 2 $+$ 4 $+$ $...$ $+$ 2n $=$ n(n $+$ 1) \hspace{0.3cm} {\textcolor{Tahiti Gold}{Esta suma genera los números rectangulares.}} \vspace{0.2cm}
\end{enumerate}



\hspace{0.1cm} \schema[closed]
	{
	\schemabox{(c) 1 $+$ 2 $+$ 2$^{2}$ $+$ $...$ $+$ 2$^{n}$ $=$ 2$^{(n+1)}$ $-$ 1 \\ \\
	(d) 1 $+$ 3 $+$ 3$^{2}$ $+$ $...$ $+$ 3$^{n}$ $=$ {\LARGE{{$\frac{3^{(n+1)} \hspace{0.1cm} - \hspace{0.1cm} 1}{2}$}}} }
	}
	{
	\schemabox{{\textcolor{Tahiti Gold}{1 $=$ 2$^{0}$ \hspace{0.2cm} $\vee$ \hspace{0.2cm}  1 $=$ 3$^{0}$ }}}
	} 
	
\vspace{0.3cm}

\begin{MyColorPar}{Tahiti Gold}

Es decir es posible hacer el primer paso de la inducción para $n$ $=$ $0$. 

Aunque se debe probar para $n$ $=$ $1$ también. 

\end{MyColorPar} \vspace{0.5cm}

{\textcolor{Terracotta}{\underline {Solución:}}}
%%%%%%%%%%%%%%%%%%%%%%%%%%%%%%%%%%%%%%%%%%%%%%%%%%%%%%%%%%%%%%%%%%%%
%%%%%%%%%%%%%%%%%%%%%%%%%%%%%%%%%%%%%%%%%%%%%%%%%%%%%%%%%%%%%%%%%%
%%%%%%%%%%%%%%%%%%%%%%%%%%%%%%%%%%%%%%%%%%%%%%%%%%%%%%%%%%%%%%%%%%%
%%%%%%%%%%%%%%%%%%%%%%%%%%%%%%%%%%%%%%%%%%%%%%%%%%%%%%%%%%%%%%%%%%
\section*{(a) 1 $+$ 3 $+$ 5 $+$ $...$ $+$ (2n $-$ 1) $=$ n$^{2}$ } 

 \begin{MyColorPar}{Boston Blue}
 
 Tenemos que garantizar que se cumplan las siguientes propiedades: 
 
\end{MyColorPar} \vspace{0.2cm}

 \begin{MyColorPar}{verde_manzana}
 $i$) $n$ $=$ $1$
 
 $ii$) $k$ $=$ $\left\{n \in \mathbb{N} \mid 1 + 3 + 5 + ... + (2n - 1) = n^{2} \right\}$ y $n$ $\in$ $k$, es decir que si $n$ satisface 1 $+$ 3 $+$ 5 $+$ $...$ $+$ (2n $-$ 1) $=$ $n$$^{2}$ \hspace{0.2cm} $\Longrightarrow$ \hspace{0.2cm} su consecutivo (n $+$ 1) también lo garantizará.
 \end{MyColorPar} \vspace{0.2cm}
 
{\textcolor{verde_manzana}{$i$)  n $=$ 1}}
 
\hspace{4cm} ( 2(1) $-$ 1 ) $=$ 1$^{2}$ \vspace{0.2cm}

\hspace{5.3cm}  2 $-$ 1 $=$ 1 \vspace{0.2cm}

\hspace{6.3cm} 1 $=$ 1 

\newpage

 {\textcolor{verde_manzana}{$ii$) Hipótesis de inducción (H.I.). }} Suponemos que se cumple para $n$\vspace{0.2cm} 
 
\hspace{4cm} {\fbox{1 $+$ 3 $+$ 5 $+$ $...$ $+$ (2n $-$ 1)}} $=$  {\fbox{n$^{2}$}} \vspace{0.2cm}
 
{\textcolor{verde_manzana}{Tesis inductiva:}}  \vspace{0.2cm}
 
Ahora debemos probar que se cumple para cuando $n$ $=$ $n$ $+$ $1$ \vspace{0.2cm}
 
\hspace{2cm} 1 $+$ 3 $+$ 5 $+$ $...$ $+$ (2n $-$ 1) $+$ (2(n $+$ 1) $-$ 1) $=$ (n $+$ 1)$^{2}$ \vspace{0.2cm} 

Queremos llegar a: (n $+$ 1)$^{2}$ $=$ n$^{2}$ $+$ 2n $+$ 1. Para ello es importante usar la equivalencia de nuestra {\textcolor{verde_manzana}{Hipótesis de inducción (H.I.). }}  \vspace{0.2cm}

Tenemos entonces: \vspace{0.5cm}

\hspace{1cm} {\textcolor{verde_manzana}{\fbox{{\black{1 $+$ 3 $+$ 5 $+$ $...$ $+$ (2n $-$ 1)}}}}}  $+$ (2(n $+$ 1) $-$ 1) $=$ {\textcolor{verde_manzana}{\fbox{{\black{n$^{2}$}}}}} $+$ (2(n $+$ 1) $-$ 1) \vspace{0.2cm}

\hspace{10.98cm} $=$ n$^{2}$ $+$ (2n $+$ 2 $-$ 1) \vspace{0.2cm}

\hspace{10.98cm} $=$ n$^{2}$ $+$ (2n $+$ 1) \vspace{0.2cm}

\hspace{10.98cm} $=$ n$^{2}$ $+$ 2n $+$ 1 \vspace{0.2cm}
 
\hspace{11.5cm} \qedsymbol \vspace{0.5cm}

%%%%%%%%%%%%%%%%%%%%%%%%%%%%%%%%%%%%%%%%%%%%%%%%%%%%%%%%%%%%%%%%%%%%
%%%%%%%%%%%%%%%%%%%%%%%%%%%%%%%%%%%%%%%%%%%%%%%%%%%%%%%%%%%%%%%%%%%
%%%%%%%%%%%%%%%%%%%%%%%%%%%%%%%%%%%%%%%%%%%%%%%%%%%%%%%%%%%%%%%%%%%
%%%%%%%%%%%%%%%%%%%%%%%%%%%%%%%%%%%%%%%%%%%%%%%%%%%%%%%%%%%%%%%%%%%%%
\section*{(b) 2 $+$ 4 $+$ $...$ $+$ 2n $=$ n(n $+$ 1) } 

 \begin{MyColorPar}{Boston Blue}
 
 Tenemos que garantizar que se cumplan las siguientes propiedades: 
 
\end{MyColorPar} \vspace{0.2cm}

 \begin{MyColorPar}{verde_manzana}
 $i$) $n$ $=$ $1$
 
 $ii$) $k$ $=$ $\left\{n \in \mathbb{N} \mid 2 + 4 + ... + 2n = n(n + 1) \right\}$ y $n$ $\in$ $k$, es decir que si $n$ satisface 2 $+$ 4 $+$ $...$ $+$ 2n $=$ n(n $+$ 1) \hspace{0.2cm} $\Longrightarrow$ \hspace{0.2cm} su consecutivo (n $+$ 1) también lo garantizará.
 \end{MyColorPar} \vspace{0.2cm}
 
 \newpage
 
{\textcolor{verde_manzana}{$i$)  n $=$ 1}} \vspace{0.2cm}

\hspace{4cm} 2(1) $=$ 1(1 $+$ 1) \vspace{0.2cm}

\hspace{4.6cm} 2 $=$ 1(2) \vspace{0.2cm}

\hspace{4.6cm} 2 $=$ 2 \vspace{0.2cm}

{\textcolor{verde_manzana}{$ii$) Hipótesis de inducción (H.I.). }} Suponemos que se cumple para $n$\vspace{0.2cm} 
 
\hspace{4cm} {\fbox{2 $+$ 4 $+$ $...$ $+$ 2n}} $=$  {\fbox{n(n $+$ 1)}} \vspace{0.2cm}
 
{\textcolor{verde_manzana}{Tesis inductiva:}}  \vspace{0.2cm}
 
Ahora debemos probar que se cumple para cuando $n$ $=$ $n$ $+$ $1$ \vspace{0.2cm}
 
\hspace{2cm} 2 $+$ 4 $+$ $...$ $+$ 2n $+$ 2(n $+$ 1) $=$ (n $+$ 1)(n $+$ 1 $+$ 1) \vspace{0.2cm} 

Queremos llegar a: (n $+$ 1)(n $+$ 1 $+$ 1) $=$ (n $+$ 1)(n $+$ 2) $=$ n$^{2}$ $+$ 3n + 2. Para ello es importante usar la equivalencia de nuestra  {\textcolor{verde_manzana}{Hipótesis de inducción (H.I.). }}  \vspace{0.2cm}

Tenemos entonces: \vspace{0.5cm}

\hspace{1cm} {\textcolor{verde_manzana}{\fbox{{\black{ 2 $+$ 4 $+$ $...$ $+$ 2n}}}}} $+$ 2(n $+$ 1) $=$ {\textcolor{verde_manzana}{\fbox{{\black{n(n $+$ 1)}}}}} $+$ 2(n $+$ 1) \vspace{0.2cm}

\hspace{7.7cm} $=$ n$^{2}$ $+$ n $+$ 2n $+$ 2 \vspace{0.2cm}

\hspace{7.7cm} $=$ n$^{2}$ $+$ 3n $+$ 2 \vspace{0.2cm}
 
\hspace{8cm} \qedsymbol \vspace{0.5cm}

\newpage

%%%%%%%%%%%%%%%%%%%%%%%%%%%%%%%%%%%%%%%%%%%%%%%%%%%%%%%%%%%%%%%%%%%%%%%%%%%%%%%%%%%%%%%%%%%%%%%%%%%%%%%%%%%%%%%%%%%%%%%%%%%%%%%%%%%%%%%%%%%%%%%%%%%%%%%%%%%%%%%%%%%%%%%%%%%%%%%%%%%%%%%%%%%%%%%%%%%%%%%%%%%%%%%%%%%%%%%%%%%%%%%%%%%%%%%%%%%%%%%%%%%%%%%%%%%%%%%%%%%%%%%%%%%%%%%%%%%

\section*{(c) 1 $+$ 2 $+$ 2$^{2}$ $+$ $...$ $+$ 2$^{n}$ $=$ 2$^{(n+1)}$ $-$ 1} 

 \begin{MyColorPar}{Boston Blue}
 
 Tenemos que garantizar que se cumplan las siguientes propiedades: 
 
\end{MyColorPar} \vspace{0.2cm}

 \begin{MyColorPar}{verde_manzana}
 $i$) $n$ $=$ $1$
 
 $ii$) $k$ $=$ $\left\{n \in \mathbb{N} \mid 1 + 2 + 2^{2} + ... + 2^{n} = 2^{(n+1)} - 1 \right\}$ y $n$ $\in$ $k$, es decir que si $n$ satisface 1 $+$ 2 $+$ 2$^{2}$ $+$ $...$ $+$ 2$^{n}$ $=$ 2$^{(n+1)}$ $-$ 1 \hspace{0.2cm} $\Longrightarrow$ \hspace{0.2cm} su consecutivo (n $+$ 1) también lo garantizará.
 \end{MyColorPar} \vspace{0.2cm}

 
{\textcolor{verde_manzana}{$i$)  n $=$ 1}} \vspace{0.2cm}

\hspace{4cm} 2$^{1}$ $=$ 2$^{(1 + 1)}$ $-$ 1  \vspace{0.2cm}

\hspace{4.2cm} 2 $=$ 2$^{(2)}$ $-$ 1  \vspace{0.2cm}

\hspace{4.2cm} 2 $=$ 4 $-$ 1 \vspace{0.2cm}

\hspace{2cm} Pero \hspace{1cm} 2 $\neq$ 3 \hspace{0.2cm} No se cumple para n $=$ 1 \vspace{0.2cm}

{\textcolor{verde_manzana}{$ii$)  n $=$ 0}} \vspace{0.2cm}

\hspace{4cm} 2$^{0}$ $=$ 2$^{(0 + 1)}$ $-$ 1  \vspace{0.2cm}

\hspace{4.2cm} 1 $=$ 2$^{(1)}$ $-$ 1  \vspace{0.2cm}

\hspace{4.2cm} 1 $=$ 2 $-$ 1 \vspace{0.2cm}

\hspace{4.2cm} 1 $=$ 1 \hspace{0.2cm} Sí se cumple para n $=$ 0 \vspace{0.2cm}

{\textcolor{verde_manzana}{$iii$) Hipótesis de inducción (H.I.). }} Suponemos que se cumple para $n$\vspace{0.2cm} 
 
\hspace{4cm} {\fbox{1 $+$ 2 $+$ 2$^{2}$ $+$ $...$ $+$ 2$^{n}$}} $=$  {\fbox{2$^{(n+1)}$ $-$ 1}} \vspace{0.2cm}
 
{\textcolor{verde_manzana}{Tesis inductiva:}}  \vspace{0.2cm}
 
Ahora debemos probar que se cumple para cuando $n$ $=$ $n$ $+$ $1$ \vspace{0.2cm}
 
\hspace{2cm} 1 $+$ 2 $+$ 2$^{2}$ $+$ $...$ $+$ 2$^{n}$ $+$ 2$^{n+1}$ $=$ 2$^{(n + 1 + 1)}$ $-$ 1 \vspace{0.2cm} 

Queremos llegar a: 2$^{(n + 1 + 1)}$ $-$ 1 $=$ 2$^{(n + 2)}$ $-$ 1. Para ello es importante usar la equivalencia de nuestra  {\textcolor{verde_manzana}{Hipótesis de inducción (H.I.). }}  \vspace{0.2cm}

Tenemos entonces: \vspace{0.5cm}

\hspace{1cm} {\textcolor{verde_manzana}{\fbox{{\black{ 1 $+$ 2 $+$ 2$^{2}$ $+$ $...$ $+$ 2$^{n}$}}}}} $+$ 2$^{n+1}$ $=$ {\textcolor{verde_manzana}{\fbox{{\black{2$^{(n+1)}$ $-$ 1}}}}} $+$ 2$^{n+1}$ \vspace{0.2cm}

\hspace{7.7cm} $=$ 2$^{(n+1)}$ $-$ 1 $+$ 2$^{n+1}$ \vspace{0.2cm}

\hspace{7.7cm} $=$ 2$^{(n+1)}$ $+$ 2$^{n+1}$ $-$ 1  \vspace{0.2cm}

\hspace{7.7cm} $=$ 2 $\cdot$ 2$^{(n+1)}$ $-$ 1  \vspace{0.2cm}

\hspace{7.7cm} $=$ 2$^{(n+1+1)}$ $-$ 1  \vspace{0.2cm}

\hspace{7.7cm} $=$ 2$^{(n+2)}$ $-$ 1  \vspace{0.2cm}
 
\hspace{8cm} \qedsymbol \vspace{0.5cm}

%%%%%%%%%%%%%%%%%%%%%%%%%%%%%%%%%%%%%%%%%%%%%%%%%%%%%%%%%%%%%%%%%%%%%%%%%%%%%%%%%%%%%%%%%%%%%%%%%%%%%%%%%%%%%%%%%%%%%%%%%%%%%%%%%%%%%%%%%%%%%%%%%%%%%%%%%%%%%%%%%%%%%%%%%%%%%%%%%%%%%%%%%%%%%%%%%%%%%%%%%%%%%%%%%%%%%%%%%%%%%%%%%%%%%%%%%%%%%%%%%%%%%%%%%%%%%%%%%%%%%%%%%%%%%%%%%%%

\section*{(d) 1 $+$ 3 $+$ 3$^{2}$ $+$ $...$ $+$ 3$^{n}$ $=$ {\LARGE{{$\frac{3^{(n+1)} \hspace{0.1cm} - \hspace{0.1cm} 1}{2}$}}}} 

 \begin{MyColorPar}{Boston Blue}
 
 Tenemos que garantizar que se cumplan las siguientes propiedades: 
 
\end{MyColorPar} \vspace{0.2cm}

 \begin{MyColorPar}{verde_manzana}
 $i$) $n$ $=$ $1$
 
 $ii$) $k$ $=$ $\left\{n \in \mathbb{N} \mid 1 + 3 + 3^{2} + ... + 3^{n} = {\LARGE{{\frac{3^{(n+1)} \hspace{0.1cm} - \hspace{0.1cm} 1}{2}}}} \right\}$ y $n$ $\in$ $k$, es decir que si $n$ satisface 1 $+$ 3 $+$ 3$^{2}$ $+$ $...$ $+$ 3$^{n}$ $=$ {\LARGE{{$\frac{3^{(n+1)} \hspace{0.1cm} - \hspace{0.1cm} 1}{2}$}}} \hspace{0.2cm} $\Longrightarrow$ \hspace{0.2cm} su consecutivo (n $+$ 1) también lo garantizará.
 \end{MyColorPar} \vspace{0.2cm}

 
{\textcolor{verde_manzana}{$i$)  n $=$ 1}} \vspace{0.2cm}

\hspace{4cm} 3$^{1}$ $=$ {\LARGE{{$\frac{3^{(1+1)} \hspace{0.1cm} - \hspace{0.1cm} 1}{2}$}}} \vspace{0.2cm}

\hspace{4cm} 3 $=$ {\LARGE{{$\frac{3^{(2)} \hspace{0.1cm} - \hspace{0.1cm} 1}{2}$}}} \vspace{0.2cm}

\hspace{4cm} 3 $=$ {\LARGE{{$\frac{9 \hspace{0.1cm} - \hspace{0.1cm} 1}{2}$}}} \vspace{0.2cm}

\hspace{4cm} 3 $=$ {\LARGE{{$\frac{8}{2}$}}} \vspace{0.2cm}

\hspace{2cm} Pero \hspace{0.8cm} 3 $\neq$ 4 \hspace{0.2cm} No se cumple para n $=$ 1 \vspace{0.2cm}

{\textcolor{verde_manzana}{$ii$)  n $=$ 0}} \vspace{0.2cm}

\hspace{4cm} 3$^{0}$ $=$ {\LARGE{{$\frac{3^{(0+1)} \hspace{0.1cm} - \hspace{0.1cm} 1}{2}$}}} \vspace{0.2cm}

\hspace{4cm} 1 $=$ {\LARGE{{$\frac{3^{(1)} \hspace{0.1cm} - \hspace{0.1cm} 1}{2}$}}} \vspace{0.2cm}

\hspace{4cm} 1 $=$ {\LARGE{{$\frac{3 \hspace{0.1cm} - \hspace{0.1cm} 1}{2}$}}} \vspace{0.2cm}

\hspace{4cm} 1 $=$ {\LARGE{{$\frac{2}{2}$}}} \vspace{0.2cm}

\hspace{4.2cm} 1 $=$ 1 \hspace{0.2cm} Sí se cumple para n $=$ 0 \vspace{0.2cm}

{\textcolor{verde_manzana}{$iii$) Hipótesis de inducción (H.I.). }} Suponemos que se cumple para $n$\vspace{0.2cm} 
 
\hspace{4cm} {\fbox{1 $+$ 3 $+$ 3$^{2}$ $+$ $...$ $+$ 3$^{n}$}} $=$  {\fbox{{\LARGE{{$\frac{3^{(n+1)} \hspace{0.1cm} - \hspace{0.1cm} 1}{2}$}}}}} \vspace{0.2cm}
 
{\textcolor{verde_manzana}{Tesis inductiva:}}  \vspace{0.2cm}
 
Ahora debemos probar que se cumple para cuando $n$ $=$ $n$ $+$ $1$ \vspace{0.2cm}
 
\hspace{2cm} 1 $+$ 3 $+$ 3$^{2}$ $+$ $...$ $+$ 3$^{n}$ $+$ 3$^{n+1}$ $=$ {\LARGE{{$\frac{3^{(n+1+1)} \hspace{0.1cm} - \hspace{0.1cm} 1}{2}$}}} \vspace{0.2cm} 

Queremos llegar a: {\LARGE{{$\frac{3^{(n+1+1)} \hspace{0.1cm} - \hspace{0.1cm} 1}{2}$}}} $=$ {\LARGE{{$\frac{3^{(n+2)} \hspace{0.1cm} - \hspace{0.1cm} 1}{2}$}}}. Para ello es importante usar la equivalencia de nuestra  {\textcolor{verde_manzana}{Hipótesis de inducción (H.I.). }}  \vspace{0.2cm}

Tenemos entonces: \vspace{0.5cm}

\hspace{1cm} {\textcolor{verde_manzana}{\fbox{{\black{ 1 $+$ 3 $+$ 3$^{2}$ $+$ $...$ $+$ 3$^{n}$}}}}} $+$ 3$^{n+1}$ $=$ {\textcolor{verde_manzana}{\fbox{{\black{{\LARGE{{$\frac{3^{(n+1)} \hspace{0.1cm} - \hspace{0.1cm} 1}{2}$}}}}}}}} $+$ 3$^{n+1}$ \vspace{0.2cm}

\hspace{7.7cm} $=$ {\LARGE{{$\frac{3^{(n+1)} \hspace{0.1cm} - \hspace{0.1cm} 1}{2}$}}} $+$ 3$^{(n+1)}$ \vspace{0.2cm}

\hspace{7.7cm} $=$  {\LARGE{{$\frac{3^{(n+1)} \hspace{0.1cm} - \hspace{0.1cm} 1 \hspace{0.2cm} + \hspace{0.2cm} 2 \cdot 3^{(n+1)}}{2}$}}} \vspace{0.2cm}

\hspace{7.7cm} $=$ {\LARGE{{$\frac{3^{(n+1)} \hspace{0.1cm} + \hspace{0.1cm} 2 \cdot 3^{(n+1)} \hspace{0.2cm} - \hspace{0.2cm} 1}{2}$}}} \vspace{0.2cm}
 
\hspace{7.7cm} $=$ {\LARGE{{$\frac{3^{(n+1)} (1 \hspace{0.1cm} + \hspace{0.1cm} 2 ) \hspace{0.2cm} - \hspace{0.2cm} 1}{2}$}}} \vspace{0.2cm}

\hspace{7.7cm} $=$  {\LARGE{{$\frac{3^{(n+1)} \cdot (3) \hspace{0.2cm} - \hspace{0.2cm} 1}{2}$}}} \vspace{0.2cm}

\hspace{7.7cm} $=$  {\LARGE{{$\frac{3^{(n+1+1)} \hspace{0.2cm} - \hspace{0.2cm} 1}{2}$}}} \vspace{0.2cm}

\hspace{7.7cm} $=$  {\LARGE{{$\frac{3^{(n+2)} \hspace{0.2cm} - \hspace{0.2cm} 1}{2}$}}} \vspace{0.2cm}
 
\hspace{8cm} \qedsymbol \vspace{0.5cm}


\end{document}
