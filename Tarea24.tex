\documentclass[12pt]{article} 
\usepackage[utf8]{inputenc}
\usepackage[spanish]{babel}
\usepackage{pdfpages}
\usepackage{csquotes}
\usepackage{afterpage}
\usepackage{parskip}
\usepackage{float}
\usepackage{enumitem}
\usepackage{multicol}
\newenvironment{Figura}
  {\par\medskip\noindent\minipage{\linewidth}}
  {\endminipage\par\medskip}
\usepackage{caption}
\usepackage{amsfonts}
\usepackage{amsmath, amsthm, amssymb}
\renewcommand{\qedsymbol}{$\blacksquare$}
\usepackage{graphicx}
\usepackage[left=2.54cm,right=2.54cm,top=2.54cm,bottom=2.54cm]{geometry}
\usepackage{pstricks} 

\usepackage{xcolor}
\definecolor{prussianblue}{RGB}{1, 45, 75} 
\definecolor{brightturquoise}{RGB}{1, 196, 254} 
\definecolor{verde_manzana}{rgb}{0.55, 0.71, 0.0}
\definecolor{Aguamarina}{rgb}{0.5, 1.0, 0.83}
\definecolor{mandarina_atomica}{rgb}{1.0, 0.6, 0.4}
\definecolor{blizzardblue}{rgb}{0.67, 0.9, 0.93}
\definecolor{bluegray}{rgb}{0.4, 0.6, 0.8}
\definecolor{coolgrey}{rgb}{0.55, 0.57, 0.67}
\definecolor{tealgreen}{rgb}{0.0, 0.51, 0.5}
\definecolor{ticklemepink}{rgb}{0.99, 0.54, 0.67}
\definecolor{thulianpink}{rgb}{0.87, 0.44, 0.63}
\definecolor{wildwatermelon}{rgb}{0.99, 0.42, 0.52}
\definecolor{wisteria}{rgb}{0.79, 0.63, 0.86}
\definecolor{yellow(munsell)}{rgb}{0.94, 0.8, 0.0}
\definecolor{trueblue}{rgb}{0.0, 0.45, 0.81}	\definecolor{tropicalrainforest}{rgb}{0.0, 0.46, 0.37}
\definecolor{tearose(rose)}{rgb}{0.96, 0.76, 0.76}
\definecolor{antiquefuchsia}{rgb}{0.57, 0.36, 0.51}	\definecolor{bittersweet}{rgb}{1.0, 0.44, 0.37}	\definecolor{carrotorange}{rgb}{0.93, 0.57, 0.13}
\definecolor{cinereous}{rgb}{0.6, 0.51, 0.48}
\definecolor{darkcoral}{rgb}{0.8, 0.36, 0.27}	\definecolor{orange(colorwheel)}{rgb}{1.0, 0.5, 0.0}
\definecolor{palatinateblue}{rgb}{0.15, 0.23, 0.89} \definecolor{pakistangreen}{rgb}{0.0, 0.4, 0.0} 	\definecolor{vividviolet}{rgb}{0.62, 0.0, 1.0} 
\definecolor{tigre}{rgb}{0.88, 0.55, 0.24} 		\definecolor{plum(traditional)}{rgb}{0.56, 0.27, 0.52} 	\definecolor{persianred}{rgb}{0.8, 0.2, 0.2} 	\definecolor{orange(webcolor)}{rgb}{1.0, 0.65, 0.0} 	\definecolor{onyx}{rgb}{0.06, 0.06, 0.06}
\definecolor{blue-violet}{rgb}{0.54, 0.17, 0.89}
\definecolor{byzantine}{rgb}{0.74, 0.2, 0.64}
\definecolor{byzantium}{rgb}{0.44, 0.16, 0.39}
\definecolor{darkmagenta}{rgb}{0.55, 0.0, 0.55} 	\definecolor{darkviolet}{rgb}{0.58, 0.0, 0.83} 	\definecolor{deepmagenta}{rgb}{0.8, 0.0, 0.8}
\newenvironment{MyColorPar}[1]{%
    \leavevmode\color{#1}\ignorespaces%
}{%
}%

\begin{document}

\begingroup
\begin{titlepage}
	\AddToShipoutPicture*{\put(79,350){\includegraphics[scale=.3]{descarga.png}}}
	\noindent
	\vspace{1mm}
\end{titlepage}
\endgroup

\pagestyle{empty} 
\setlength{\parindent}{0pt}
\sffamily

%%%%%%%%%%%%%%%%%%%%%%%%%%%%%%%%%%%%%%%%%%%%%%%%%%%%%%%%%%%%%%%%%%%
%%%%%%%%%%%%%%%%%%%%%%%%%%%%%%%%%%%%%%%%%%%%%%%%%%%%%%%%%%%%%%%%%%%

\begin{center} 

    \LARGE{\bf{\textsf{Benemérita Universidad Autónoma de Puebla}}} \\[0.5cm]
    
\begin{figure}[htb] \centering

    \includegraphics[scale=.25]{LogoBUAPpng.png} 

\end{figure}

%%%%%%%%%%%%%%%%%%%%%%%%%%%%%%%%%%%%%%%%%%%%%%%%%%%%%%%%%%%%%%%%%%%
%%%%%%%%%%%%%%%%%%%%%%%%%%%%%%%%%%%%%%%%%%%%%%%%%%%%%%%%%%%%%%%%%%%

    \LARGE{Facultad de Ciencias Físico Matemáticas}\\[0.5cm]

\begin{figure}[htb] \centering

    \includegraphics[scale=.4]{LogoFCFMBUAP.png} 
    
\end{figure} 

%%%%%%%%%%%%%%%%%%%%%%%%%%%%%%%%%%%%%%%%%%%%%%%%%%%%%%%%%%%%%%%%%%%
%%%%%%%%%%%%%%%%%%%%%%%%%%%%%%%%%%%%%%%%%%%%%%%%%%%%%%%%%%%%%%%%%%%

    \Large{Licenciatura en Física Teórica}\\[0.5cm]
    \Large{Primer semestre} 

\end{center} \vspace{0.3cm}
%%%%%%%%%%%%%%%%%%%%%%%%%%%%%%%%%%%%%%%%%%%%%%%%%%%%%%%%%%%%%%%%%%%
%%%%%%%%%%%%%%%%%%%%%%%%%%%%%%%%%%%%%%%%%%%%%%%%%%%%%%%%%%%%%%%%%%%

\begin{center}

    {\Large{\bfseries{{\textcolor{carrotorange}{Tarea 23}}}}} \\ 
    
\end{center}

    \large{\bf{\textsf{Curso:}}} {\bfseries{{\textcolor{brightturquoise}{Matemáticas básicas \bfseries{(N.R.C.:25598)}}}}} \\
    \large{\bf{\textsf{Alumno:}}} {\bfseries{{\textcolor{prussianblue}{Julio Alfredo Ballinas García {\large{{$\mid$}}} 202107583}}}}  \\
    \large{\bf{\textsf{Docente:}}} {\bfseries{{\textcolor{wisteria}{Dra. María Araceli Juárez Ramírez}}}}\\
    \large{\bf{\textsf{Grupo:}}} {\bfseries{{\textcolor{verde_manzana}{102}}}}\\

\vfill
    
\begin{center} 

    {\small{\textsf{\underline{Tarea retrasada:} venció 19 de octubre {\red{23:59 PM}}} {\LARGE{ $\mid$ }}\textsf{{\underline{Fecha de hoy:}} 26 de octubre}}}
    
\end{center}

\newpage

%%%%%%%%%%%%%%%%%%%%%%%%%%%%%%%%%%%%%%%%%%%%%%%%%%%%%%%%%%%%%%%%%%%
%%%%%%%%%%%%%%%%%%%%%%%%%%%%%%%%%%%%%%%%%%%%%%%%%%%%%%%%%%%%%%%%%%%

\section{\textsf{Resolver un ejercicio de ecuaciones con valor absoluto}:} \vspace{.5cm}

{\LARGE{ \hspace{.1cm} $\mid 2x + 4 \mid$ $+$ $\mid -2x + 7x -1 \mid$ $=$ $5$}} \vspace{.5cm}

{\Large{Esto es equivalente a}} ({\LARGE{$\equiv$}}) {\Large{:}} \vspace{0.5cm}

{\LARGE{ \hspace{.1cm} $\mid 2x \hspace{0.15cm} + \hspace{0.15cm} 4 \mid$ \hspace{0.15cm} $+$ \hspace{0.15cm} $\mid 5x \hspace{0.15cm} - \hspace{0.15cm}1 \mid$ $=$ $5$}} \hspace{0.2cm} {\Large{$\longleftarrow$}} \hspace{0.0005cm} {\fbox{entonces trabajaremos con la equivalencia.}} \vspace{0.5cm}

{\red{{\underline{Solución:}}}} \vspace{0.5cm} 

\begin{MyColorPar}{palatinateblue}
Iniciamos haciendo un estudio de signos.

Buscamos los ceros de las expresiones {\black{$\mid 2x + 4\mid$}} y {\black{$\mid 5x - 1\mid$}} . \vspace{0.5cm}
\end{MyColorPar}

{\bfseries{1.-}} \hspace{0.2cm} $2x$ $+$ $4$ \hspace{0.2cm} $\Longrightarrow$ \hspace{0.2cm} $2x$ $+$ $4$ $=$ $0$ \hspace{0.2cm} $\Longrightarrow$ \hspace{0.2cm} $2x$ $=$ $-4$ \hspace{0.2cm} $\Longrightarrow$ \hspace{0.2cm} $x$ $=$ {\Large{$-\frac{4}{2}$}} \vspace{0.1cm} 

\hspace{1cm} $\Longrightarrow$ \hspace{0.2cm} \fbox{$x$ $=$ $-2$}  \vspace{0.5cm}

{\bfseries{2.-}} \hspace{0.2cm} $5x$ $-$ $1$ \hspace{0.2cm} $\Longrightarrow$ \hspace{0.2cm} $5x$ $-$ $1$ $=$ $0$ \hspace{0.2cm} $\Longrightarrow$ \hspace{0.2cm} $5x$ $=$ $1$ \hspace{0.2cm} $\Longrightarrow$ \hspace{0.2cm} \fbox{$x$ $=$ {\Large{$\frac{1}{5}$}}} \vspace{0.5cm}

\begin{MyColorPar}{palatinateblue}
Hallamos que los ceros son {\black{\fbox{$x$ $=$ $-2$}}} y  {\black{\fbox{$x$ $=$ {\Large{$\frac{1}{5}$}}}}}
\end{MyColorPar}

\begin{table}[H]
\centering
\begin{tabular}{l|l|l|l|l|l|l|l}
\multicolumn{8}{l}{\hspace{2cm} {\tiny{$-$ 2}} \hspace{.3cm} {\tiny{1/5}}}                                                      \\ 
\cline{2-5}
                        &  $2x + 4$ & {\red{$-$}}  & {\textcolor{verde_manzana}{$+$}}  & {\textcolor{verde_manzana}{$+$}}    \\ 
\cline{2-5}
                        & $5x - 1$ & {\red{$-$}} & {\red{$-$}}  & {\textcolor{verde_manzana}{$+$}}  \\ 
\cline{2-5}
\end{tabular}
\end{table} \vspace{0.5cm}

\begin{MyColorPar}{palatinateblue}
Los dominios encontrados son:
\end{MyColorPar}

{\bfseries{1.-}} \hspace{0.2cm} $\forall$$_{x}$ $\in$ D$_{1}$ $=$ {\textcolor{wisteria}{{\bfseries{($-$ $\infty${\Large{,}} $-$ $2$]}}}} \hspace{0.5cm} {\textcolor{palatinateblue}{{\underline{Nota:}} en este intervalo $2x$ $+$ $4$ es {\red{negativo}} y}} 

{\textcolor{palatinateblue}{$5x$ $-$ $1$ también.}}  \vspace{0.5cm}

\begin{center}
    
$\mid 2x + 4 \mid$ $+$ $\mid 5x - 1\mid$ $=$ $5$ 
\end{center}

\newpage

\begin{MyColorPar}{palatinateblue}
Definición de valor absoluto:
\end{MyColorPar}

\hspace{3cm} $\Longleftrightarrow$ \hspace{0.2cm} $-2x$ $-$ $4$ $-5x$ $+$ $1$ $=$ $5$ \vspace{0.2cm}

\hspace{3cm} $\Longleftrightarrow$ \hspace{0.2cm} $-7x$ $-$ $3$ $=$ $5$ \vspace{0.2cm}

\hspace{3cm} $\Longleftrightarrow$ \hspace{0.2cm} $-7x$ $=$ $8$ \vspace{0.2cm}

\hspace{3cm} $\Longleftrightarrow$ \hspace{0.2cm} \fbox{$x$ $=$ {\LARGE{$-\frac{8}{7}$}}} \hspace{0.2cm} $\neq$ \hspace{0.2cm}  $5$  \vspace{0.2cm}

\begin{MyColorPar}{verde_manzana}
Solución parcial: $S_{1}$ $=$ {\Large{\ $\emptyset$ \ }}. Por lo tanto no es una solución.
\end{MyColorPar} \vspace{0.5cm}

%%%%%%%%%%%%%%%%%%%%%%%%%%%%%%%%%%%%%%%%%%%%%%%%%%%%%%%%%%%%%%%%%%
%%%%%%%%%%%%%%%%%%%%%%%%%%%%%%%%%%%%%%%%%%%%%%%%%%%%%%%%%%%%%%%%%%
%%%%%%%%%%%%%%%%%%%%%%%%%%%%%%%%%%%%%%%%%%%%%%%%%%%%%%%%%%%%%%%%%%%
%%%%%%%%%%%%%%%%%%%%%%%%%%%%%%%%%%%%%%%%%%%%%%%%%%%%%%%%%%%%%%%%%
%%%%%%%%%%%%%%%%%%%%%%%%%%%%%%%%%%%%%%%%%%%%%%%%%%%%%%%%%%%%%%%%%


{\bfseries{2.-}} \hspace{0.2cm} $\forall$$_{x}$ $\in$ D$_{2}$ $=$ {\textcolor{wisteria}{{\bfseries{[$-$ $2${\Large{,}} {\Large{$\frac{1}{5}$}}]}}}}  \vspace{0.5cm}
 \hspace{0.5cm} {\textcolor{palatinateblue}{{\underline{Nota:}} en este intervalo $2x$ $+$ $4$ es {\textcolor{verde_manzana}{positivo}} y}} 

{\textcolor{palatinateblue}{$5x$ $-$ $1$}} {\red{negativo}}.  
\begin{center}
    
$\mid 2x + 4 \mid$ $+$ $\mid 5x - 1\mid$ $=$ $5$ 
\end{center}

\begin{MyColorPar}{palatinateblue}
Definición de valor absoluto:
\end{MyColorPar}

\hspace{3cm} $\Longleftrightarrow$ \hspace{0.2cm} $2x$ $+$ $4$ $-$ ($5x$ $-$ $1$) $=$ $5$ \vspace{0.2cm}

\hspace{3cm} $\Longleftrightarrow$ \hspace{0.2cm} $2x$ $+$ $4$ $-$ $5x$ $+$ $1$ $=$ $5$ \vspace{0.2cm}

\hspace{3cm} $\Longleftrightarrow$ \hspace{0.2cm} $-3x$ $+$ $5$ $=$ $5$ \vspace{0.2cm}

\hspace{3cm} $\Longleftrightarrow$ \hspace{0.2cm} $-3x$ $=$ $5$ $-$ $5$  \vspace{0.2cm}

\hspace{3cm} $\Longleftrightarrow$ \hspace{0.2cm} $-3x$ $=$ $0$ \hspace{0.2cm} $\Longleftrightarrow$ \hspace{0.2cm} \fbox{$x$ $=$ $0$}   \vspace{0.2cm}

\begin{MyColorPar}{verde_manzana}
Solución parcial: $S_{2}$ $=$ {\Large{\{ $0$ \}}}.
\end{MyColorPar} \vspace{0.5cm}

%%%%%%%%%%%%%%%%%%%%%%%%%%%%%%%%%%%%%%%%%%%%%%%%%%%%%%%%%%%%%%%%%%
%%%%%%%%%%%%%%%%%%%%%%%%%%%%%%%%%%%%%%%%%%%%%%%%%%%%%%%%%%%%%%%%%
%%%%%%%%%%%%%%%%%%%%%%%%%%%%%%%%%%%%%%%%%%%%%%%%%%%%%%%%%%%%%%%%%%
%%%%%%%%%%%%%%%%%%%%%%%%%%%%%%%%%%%%%%%%%%%%%%%%%%%%%%%%%%%%%%%%%%

{\bfseries{3.-}} \hspace{0.2cm} $\forall$$_{x}$ $\in$ D$_{3}$ $=$ {\textcolor{wisteria}{{\bfseries{[{\Large{$\frac{1}{5}$}} {\Large{,}} $\infty$]}}}} {\textcolor{palatinateblue}{{\underline{Nota:}} en este intervalo $2x$ $+$ $4$ es {\textcolor{verde_manzana}{positivo}} y}} 

{\textcolor{palatinateblue}{$5x$ $-$ $1$ también.}} \vspace{0.5cm} 

\begin{center}
    
$\mid 2x + 4 \mid$ $+$ $\mid 5x - 1\mid$ $=$ $5$ 
\end{center}

\begin{MyColorPar}{palatinateblue}
Definición de valor absoluto:
\end{MyColorPar}

\hspace{3cm} $\Longleftrightarrow$ \hspace{0.2cm} $2x$ $+$ $4$ $+$ $5x$ $-$ $1$ $=$ $5$ \vspace{0.2cm}

\hspace{3cm} $\Longleftrightarrow$ \hspace{0.2cm} $7x$ $+$ $3$ $=$ $5$ \vspace{0.2cm}

\hspace{3cm} $\Longleftrightarrow$ \hspace{0.2cm} $7x$ $=$ $5$ $-$ $3$ \vspace{0.2cm}

\hspace{3cm} $\Longleftrightarrow$ \hspace{0.2cm} $7x$ $=$ $2$ \hspace{0.2cm} $\Longleftrightarrow$ \hspace{0.2cm} \fbox{$x$ $=$ {\Large{${\frac{2}{7}}$}}} \vspace{0.2cm}

\begin{MyColorPar}{verde_manzana}
Solución parcial: $S_{3}$ $=$ {\Large{\{ {\Large{${\frac{2}{7}}$}} \}}}. 
\end{MyColorPar} \vspace{0.5cm}

\begin{MyColorPar}{verde_manzana}
Solución total: $S_{T}$ $=$ $S_{1} + S_{2} + S_{3}$ . Esto es 
 $S_{T}$ $=$ $\emptyset + \hspace{0.2cm} {0} + \hspace{0.2cm} {\Large{\frac{2}{7}}}$ 

\end{MyColorPar} \vspace{0.5cm}

%%%%%%%%%%%%%%%%%%%%%%%%%%%%%%%%%%%%%%%%%%%%%%%%%%%%%%%%%%%%%%%%%%%%
%%%%%%%%%%%%%%%%%%%%%%%%%%%%%%%%%%%%%%%%%%%%%%%%%%%%%%%%%%%%%%%%%%%%
%%%%%%%%%%%%%%%%%%%%%%%%%%%%%%%%%%%%%%%%%%%%%%%%%%%%%%%%%%%%%%%%%%%%%%%%%%%%%%%%%%%%%%%%%%%%%%%%%%%%%%%%%%%%%%%%%%%%%%%%%%%%%%%%%%%%%%%%


\end{document}
