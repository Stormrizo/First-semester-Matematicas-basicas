\documentclass[12pt]{article} 
\usepackage[utf8]{inputenc}
\usepackage[spanish]{babel}
\usepackage{amsfonts}
\usepackage{amsmath, amsthm, amssymb}
\usepackage{graphicx}
\usepackage[left=2.54cm,right=2.54cm,top=2.54cm,bottom=2.54cm]{geometry}
\usepackage{pstricks}
\begin{document}

\thispagestyle{empty} 
\begin{center} \LARGE{\bf Benemérita Universidad Autónoma de Puebla} \\[0.5cm]
\begin{figure}[htb] \centering \includegraphics[scale=.2]{LogoBUAPpng.png} \end{figure}
\LARGE{Facultad de Ciencias Físico Matemáticas}\\[0.5cm]
\begin{figure}[htb] \centering \includegraphics[scale=.39]{LogoFCFMBUAP.png} \end{figure} 
\Large{Licenciatura en Física Teórica}\\[0.5cm]
\large{Primer semestre} \end{center}
\begin{center} { \Large \bfseries{Tarea 13}} \\ \end{center}
\large{\bf Curso:} Matemáticas básicas \textbf{(N.R.C.:25598)}\\
\large{\bf Alumno:} Julio Alfredo Ballinas García $\left(202107583\right)$ \\
\large{\bf Docente:} Dra. María Araceli Juárez Ramírez\\
\large{\bf Grupo:} 102\\ \begin{center} 
\vfill
\textsc{18 de septiembre de 2021} \end{center}  
\newpage
\sffamily
\section{Demostración de teorema 6.}

\begin{figure}[htb] \centering \includegraphics[scale=.9]{Imagen1.png}
\caption{Axiomas $R_1$ a $R_1_6$}
\end{figure}
\subsection{El elemento neutro para el producto en \mathbb{R} es único (se denota $é=1$)}\\

Antes de iniciar con la demostración del teorema 6. es necesario aclarar que este teorema es consecuencia de los axiomas $R_1$ a $R_1_6$ vistos en la sesión N$^{o}$ 14 de la clase del día 17 de septiembre de 2021. \\

Por lo tanto para poder llegar a una conclusión será necesario emplear algunos axiomas más. 

{\red{Solución: \black{\underline{inicio de la demostración}}}}\\

{\blue{
Supongamos que existe un elemento neutro para el producto.\\

Sea este elemento:

\begin{center}
    \red{$k$}
\end{center}

Aplicamos axioma {\underline{$R_5$}}

\begin{center}
$\forall\left(x,y\right) \in \mathbb{R} \mathrm{ X } \mathbb{R} $ $ xy = yx$
\end{center}

Para el elemento  {\red{$k$}} $=$ $y$ tenemos:
\begin{center}
$ x${\red{$k$}} = {\red{$k$}}$x$
\end{center}

Sabemos por el axioma {\underline{$R_3$}} que existe un único elemento en los reales \mathbb{R} que funciona como neutro para la suma y se denota como $e$. 

\begin{center}
$\exists e\in \mathbb{R} \mid \forall x \in \mathbb{ R }$ $e + x = x$
\end{center}

En consecuencia al tratarse de los números reales $\mathbb{ R }$ existe de manera simultánea un elemento neutro (denotado por $é$) para el producto, este está descrito por medio del axioma $R_7$. 

\begin{center}
$\exists é\in \mathbb{R} \mid é \neq e \wedge \forall x \in \mathbb{ R }$ $éx = x$
\end{center}

Aplicando los axiomas {\underline{$R_5$}} y {\underline{$R_7$}} tenemos:

\begin{equation*}
    \begin{split}
    x {\red{k}}  & = {\red{k}} x \\\\ \textbf{Sea {\red{k}} = {\red{é}} }
         & =  {\red{é}} x\\\\ \textbf{Esto es}
        & = x\\\\
    \end{split}
\end{equation*}

{\hfill \textit{\textbf{Q.E.D.}}}}}
\newpage

\section{Investigar el nombre de la propiedad o axioma de los números reales que dice ($R_1_0$):}

\begin{center}
$\forall x\in \mathbb{R} $ si $  $ x \leq y $ \wedge $ y \leq x $ \Rightarrow x=y$
\end{center}

{\red{Solución:}}\\

\begin{itemize}
    \item Propiedad antisimétrica para la relación de orden. [I]
\end{itemize}

\section{Referencias}
[I] Ferrari, G., \& Tenembaum, S. (s.f.) \textit{RESUMEN DE RELACIONES Y FUNCIONES}. Recuperado de URL: http://www.x.edu.uy/inet/RELACIONES\_FUNCIONES.pdf

\end{document}
