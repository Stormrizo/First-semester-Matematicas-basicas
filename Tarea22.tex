\documentclass[12pt]{article} 
\usepackage[utf8]{inputenc}
\usepackage[spanish]{babel}
\usepackage{pdfpages}
\usepackage{parskip}
\usepackage{float}
\usepackage{enumitem}
\usepackage{multicol}
\newenvironment{Figura}
  {\par\medskip\noindent\minipage{\linewidth}}
  {\endminipage\par\medskip}
\usepackage{caption}
\usepackage{amsfonts}
\usepackage{amsmath, amsthm, amssymb}
\renewcommand{\qedsymbol}{$\blacksquare$}
\usepackage{graphicx}
\usepackage[left=2.54cm,right=2.54cm,top=2.54cm,bottom=2.54cm]{geometry}
\usepackage{pstricks}
\usepackage{xcolor}
\definecolor{prussianblue}{rgb}{0.55, 0.71, 0.0} 
\definecolor{verde_manzana}{rgb}{0.55, 0.71, 0.0}
\definecolor{Aguamarina}{rgb}{0.5, 1.0, 0.83}
\definecolor{mandarina_atomica}{rgb}{1.0, 0.6, 0.4}
\definecolor{blizzardblue}{rgb}{0.67, 0.9, 0.93}
\definecolor{bluegray}{rgb}{0.4, 0.6, 0.8}
\definecolor{coolgrey}{rgb}{0.55, 0.57, 0.67}
\definecolor{tealgreen}{rgb}{0.0, 0.51, 0.5}
\definecolor{ticklemepink}{rgb}{0.99, 0.54, 0.67}
\definecolor{thulianpink}{rgb}{0.87, 0.44, 0.63}
\definecolor{wildwatermelon}{rgb}{0.99, 0.42, 0.52}
\definecolor{wisteria}{rgb}{0.79, 0.63, 0.86}
\definecolor{yellow(munsell)}{rgb}{0.94, 0.8, 0.0}
\definecolor{trueblue}{rgb}{0.0, 0.45, 0.81}	\definecolor{tropicalrainforest}{rgb}{0.0, 0.46, 0.37}
\definecolor{tearose(rose)}{rgb}{0.96, 0.76, 0.76}
\definecolor{antiquefuchsia}{rgb}{0.57, 0.36, 0.51}	\definecolor{bittersweet}{rgb}{1.0, 0.44, 0.37}	\definecolor{carrotorange}{rgb}{0.93, 0.57, 0.13}
\definecolor{cinereous}{rgb}{0.6, 0.51, 0.48}
\definecolor{darkcoral}{rgb}{0.8, 0.36, 0.27}	\definecolor{orange(colorwheel)}{rgb}{1.0, 0.5, 0.0}
\definecolor{palatinateblue}{rgb}{0.15, 0.23, 0.89} \definecolor{pakistangreen}{rgb}{0.0, 0.4, 0.0} 	\definecolor{vividviolet}{rgb}{0.62, 0.0, 1.0} 
\definecolor{tigre}{rgb}{0.88, 0.55, 0.24} 	\definecolor{prussianblue}{rgb}{0.0, 0.19, 0.33} 	\definecolor{plum(traditional)}{rgb}{0.56, 0.27, 0.52} 	\definecolor{persianred}{rgb}{0.8, 0.2, 0.2} 	\definecolor{orange(webcolor)}{rgb}{1.0, 0.65, 0.0} 	\definecolor{onyx}{rgb}{0.06, 0.06, 0.06}
\definecolor{blue-violet}{rgb}{0.54, 0.17, 0.89}
\definecolor{byzantine}{rgb}{0.74, 0.2, 0.64}
\definecolor{byzantium}{rgb}{0.44, 0.16, 0.39}
\definecolor{darkmagenta}{rgb}{0.55, 0.0, 0.55} 	\definecolor{darkviolet}{rgb}{0.58, 0.0, 0.83} 	\definecolor{deepmagenta}{rgb}{0.8, 0.0, 0.8}

\begin{document}
\pagestyle{empty} 
\setlength{\parindent}{0pt}
\sffamily
\begin{center} \LARGE{\bf Benemérita Universidad Autónoma de Puebla} \\[0.5cm]
\begin{figure}[htb] \centering \includegraphics[scale=.2]{LogoBUAPpng.png} \end{figure}
\LARGE{Facultad de Ciencias Físico Matemáticas}\\[0.5cm]
\begin{figure}[htb] \centering \includegraphics[scale=.39]{LogoFCFMBUAP.png} \end{figure} 
\Large{Licenciatura en Física Teórica}\\[0.5cm]
\large{Primer semestre} \end{center}
\begin{center} { \Large \bfseries{Tarea 22}: (Teorema 25)} \\ \end{center}
\large{\bf Curso:} Matemáticas básicas \textbf{(N.R.C.:25598)}\\
\large{\bf Alumno:} Julio Alfredo Ballinas García $\left(202107583\right)$ \\
\large{\bf Docente:} Dra. María Araceli Juárez Ramírez\\
\large{\bf Grupo:} 102\\ \begin{center} 
\vfill
\textsc{\underline{Tarea retrasada:} venció 12 de octubre} \end{center}
\begin{center}
\textsc{Fecha de hoy: 13 de octubre}
\end{center}
\newpage

\section*{\sffamily{Mostrar {\red{Teorema 25}} {\blue{ii)}}: Se tiene en $\mathbb{R}$ }} \vspace{.5cm}

{\LARGE{{\blue{ii)}} \hspace{.1cm} $x$ $\leq$ $0$ $\wedge$ $y$ $\leq$ $0$ $\Longrightarrow$ $xy$ $\geq$ $0$}} \vspace{.5cm}


{\red{{\underline{Demostración directa}}}}  \vspace{0.5cm}

{\red{{\underline{Solución:}}}} \vspace{0.5cm} 

{\textcolor{palatinateblue}{Suponemos}} {\Large{$x$ $\leq$ $0$ $\wedge$ $y$ $\leq$ $0$ }} {\textcolor{pakistangreen}{verdadera.}} \vspace{0.5cm}

{\textcolor{carrotorange}{Aplicamos axioma 14}} \textcolor{pakistangreen}{($\forall$ $(x,y)$ si [$e$ $\leq$ $x$ $\wedge$ $e$ $\leq$ $y$] $\Rightarrow$ $e$ $\leq$ $xy$):} {\vspace{0.5cm}}

{\textcolor{palatinateblue}{Para aplicar el}}  {\textcolor{carrotorange}{axioma 14 (R$_{14}$)}} {\textcolor{palatinateblue}{ debemos transformar el antecedente}} {\textcolor{pakistangreen}{($x$ $\leq$ $0$ $\wedge$ $y$ $\leq$ $0$)}} \vspace{0.5cm}

{\textcolor{carrotorange}{Por teorema 19 (T$_{19}$) $i$) 4}} {\textcolor{pakistangreen}{[$-y$ $\leq$ $-x$ $\Longrightarrow$ $x$ $\leq$ $y$]}} {\textcolor{palatinateblue}{reescribimos al antecedente}} \vspace{0.1cm}
 
{\textcolor{pakistangreen}{$x$ $\leq$ $0$ $\wedge$ $y$ $\leq$ $0$}} {\textcolor{palatinateblue}{como}}: \vspace{0.5cm}

1. \hspace{2.1cm} $x$ $\leq$ $0$ $\Longrightarrow$ {{\textcolor{darkviolet}{\fbox{{\black{$0$ $\leq$ $-x$}}}}}}\vspace{0.5cm}

2. \hspace{2.1cm} $y$ $\leq$ $0$ $\Longrightarrow$ {{\textcolor{darkviolet}{\fbox{{\black{$0$ $\leq$ $-y$}}}}}}\vspace{0.5cm}

{\textcolor{palatinateblue}{Tenemos entonces por el }} {\textcolor{carrotorange}{teorema 19 (T$_{19}$) $i$) 4}}: \vspace{0.5cm}

\hspace{1.4cm} $x$ $\leq$ $0$ \hspace{0.1cm} $\wedge$ \hspace{0.1cm} $y$ $\leq$ $0$ \hspace{0.1cm} $\Longrightarrow$ \hspace{0.1cm} $0$ $\leq$ $-x$ \hspace{0.1cm} $\wedge$ \hspace{0.1cm} $-y$ $\geq$ $0$ \vspace{0.5cm}

\hspace{1cm}{\textcolor{carrotorange}{Por teorema 12 (T$_{12}$)}}\hspace{0.1cm} $\Longrightarrow$ \hspace{0.1cm} $-x(-y)$ $=$ $-(-xy)$ \vspace{0.5cm}

{\textcolor{palatinateblue}{Así:}}

\hspace{5.6cm}$0$ $\leq$ $-(-xy)$ \vspace{0.5cm}

\hspace{1.5cm}{\textcolor{carrotorange}{Por teorema 4 (T$_{4}$)}}\hspace{0.1cm} $\Longrightarrow$ \hspace{0.1cm} $0$ $\leq$ $xy$ \vspace{0.5cm}

\newpage

\begin{equation*}
    \begin{split}
      & \Longrightarrow -xy + 0 \leq xy + (-xy)  \quad \textup{{\textcolor{carrotorange}{Por axioma 4 (R$_{4}$)}}} \\\\ 
      & \Longrightarrow -xy \leq 0 \hspace{3.6cm} \textup{{\textcolor{carrotorange}{{\textcolor{carrotorange}{Por ax. 3 y ax. 4 (R$_{3}$) y (R$_{4}$)}}}}} \\\\
       & \Longrightarrow -(-xy) \geq 0 \hspace{2.8564cm} \textup{{\textcolor{carrotorange}{Por teorema 19 $i$) 4 (T$_{19}$)}}} \\\\
      & \Longrightarrow {{\textcolor{darkviolet}{\fbox{{\black{$xy$ \geq 0}}}}}} \hspace{4.01cm} {\textup{{\textcolor{carrotorange}{Por teorema 4 (T$_{4}$)}} \hspace{0.1cm} {\textcolor{carrotorange}\qedsymbol}}} \\\\
    \end{split}
\end{equation*}
\end{document}

