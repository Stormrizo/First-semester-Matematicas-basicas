\documentclass[12pt]{article} 
\usepackage[utf8]{inputenc}
\usepackage[spanish]{babel}
\usepackage{amsfonts}
\usepackage{amsmath, amsthm, amssymb}
\renewcommand{\qedsymbol}{$\blacksquare$}
\usepackage{graphicx}
\usepackage[left=2.54cm,right=2.54cm,top=2.54cm,bottom=2.54cm]{geometry}
\usepackage{pstricks}
\begin{document}

\thispagestyle{empty} 
\begin{center} \LARGE{\bf Benemérita Universidad Autónoma de Puebla} \\[0.5cm]
\begin{figure}[htb] \centering \includegraphics[scale=.2]{LogoBUAPpng.png} \end{figure}
\LARGE{Facultad de Ciencias Físico Matemáticas}\\[0.5cm]
\begin{figure}[htb] \centering \includegraphics[scale=.39]{LogoFCFMBUAP.png} \end{figure} 
\Large{Licenciatura en Física Teórica}\\[0.5cm]
\large{Primer semestre} \end{center}
\begin{center} { \Large \bfseries{Tarea 16}} \\ \end{center}
\large{\bf Curso:} Matemáticas básicas \textbf{(N.R.C.:25598)}\\
\large{\bf Alumno:} Julio Alfredo Ballinas García $\left(202107583\right)$ \\
\large{\bf Docente:} Dra. María Araceli Juárez Ramírez\\
\large{\bf Grupo:} 102\\ \begin{center} 
\vfill
\textsc{28 de septiembre de 2021} \end{center}  
\newpage
\sffamily
\section*{Lista de axiomas.}

\begin{figure}[htb] \centering \includegraphics[scale=.95]{Lista de axiomas 1.png} 
\caption{Axiomas $R_1$ a $R_1_4$}
\end{figure} 
\newpage
\begin{figure}[htb] \centering \includegraphics[scale=.95]{Lista de axiomas 2.png} 
\caption{Axiomas $R_5$ a $R_1_5$}
\end{figure} 
\newpage
\begin{figure}[htb] \centering \includegraphics[scale=.95]{Lista de axiomas 3.png} 
\caption{Axioma $R_1_6$}
\end{figure} 

\vspace{1cm}

\section*{\blue{Teorema 12:} \red{Caso 2}. \black{$\textbf{x(-y) = -xy}$}}\\

{\textit{Para mostrar esto sabemos que si al producto $\textbf{xy}$ le sumamos su inverso aditivo $(\textbf{-xy})$ el resultado es $\textbf{e}$ o $\textbf{0}$}:}\\

\begin{center}
    $xy + (- xy) = 0$ \hspace{1cm} \textup{Por axioma 4 (\textbf{R}$_4)$}
\end{center} \vspace{0.8cm}

{\red{\underline{Solución:}}} Para mostrar 2., bastará mostrar $xy + x(-y)=0$. Aquí asumimos que $x(-y)$ es el elemento simétrico ($x^{\prime}$) de $xy$ ($x$). \\\\

{\red{\underline{Demostración:}}}
\begin{equation*}
    \begin{split}
       xy + x(-y) & = \blue x(y + (-y)) \hspace{1cm} \textup{\black{Por axioma 9 (R$_9)$}} \\\\ 
        & = \blue x(0)\hspace{2.7cm} \textup{\black{Por axioma 4 (R$_4)$}}\\\\ 
         & = \blue 0 \hspace{3.4cm} \textup{\black{Por teorema 5 $(xy=0 \hspace{0.3cm} \Leftrightarrow \hspace{0.3cm} x = 0\hspace{0.3cm} \vee \hspace{0.3cm} y=0)$}}\\\\
          xy + x(-y) & = \blue 0\hspace{3.4cm} \textup{\black{Por transitividad de la igualdad}}\\\\ 
           x(-y) & = \blue -xy\hspace{2.8cm} \textup{\black{Por teorema 3 $(x^{\prime}=-x)$}} \hspace{.5cm} \textup{\black{\qedsymbol}}\\\\ 
    \end{split}
\end{equation*}
\newpage

\section*{\blue{Teorema 13:}}\\

{\textit{Para todo $x$ en $\mathbb{R}$ se tiene \hspace{.5cm} $-x=-1(x)$}:}\\

{\red{\underline{Demostración:}}}
\begin{equation*}
    \begin{split}
       -x& = \blue é(-x) \hspace{1cm} \textup{\black{Por axioma 7 (R$_7)$}} \\\\ 
        & = \blue -1x \hspace{1.4cm} \textup{\black{Por axioma 5 (R$_5)$}} \\\\
         & = \blue -1(x) \hspace{1cm} \textup{\black{Por axioma 6 (R$_6)$}} \hspace{.5cm} \textup{\black{\qedsymbol}}\\\\ 
    \end{split}
\end{equation*}

\section*{\blue{Teorema 14: \black{ida \Rightarrow}}}\\

{\textit{Se tiene siempre en $\mathbb{R}$:\hspace{.5cm} $x\hspace{.2cm}\leq\hspace{.2cm} y \hspace{.2cm}\Leftrightarrow\hspace{.2cm} x \hspace{.2cm}<\hspace{.2cm} y \hspace{.4cm}\vee\hspace{.4cm} x\hspace{.2cm}=\hspace{.2cm}y$ }:}\\

Mostrar ida\hspace{.2cm} $``\Rightarrow"$, es decir \hspace{0.4cm} $x\hspace{.2cm}\leq \hspace{.2cm} y \hspace{.2cm}\Rightarrow\hspace{.2cm} x\hspace{.2cm}<\hspace{.2cm}y \hspace{.4cm}\vee \hspace{.4cm}x\hspace{.2cm}=\hspace{.2cm}y$ \\

Para la solución de esta demostración vamos a usar la siguiente tautología: \\

\begin{center}
    $P\hspace{0.3cm}\Rightarrow \hspace{0.3cm}Q \hspace{0.3cm}\vee\hspace{0.3cm} R \hspace{0.3cm}\Rightarrow\hspace{0.3cm} P \hspace{0.3cm}\wedge\hspace{0.3cm} \neg \hspace{0.1cm} R\hspace{0.3cm} \Rightarrow \hspace{0.3cm}Q$
\end{center}
\vspace{0.2cm} 

\hspace{0.7cm} $\textbf{P}=x \hspace{0.3cm} \leq \hspace{0.3cm} y$\hspace{0.4cm} $\textbf{Q}=x \hspace{0.3cm} < \hspace{0.3cm} y$ \hspace{0.4cm} $\textbf{R}=x \hspace{0.3cm} = \hspace{0.3cm} y$\\

Nos queda: 

\begin{center}
    $x \hspace{0.3cm} \leq \hspace{0.3cm} y \hspace{0.3cm} \wedge \hspace{0.3cm}  \neg \hspace{0.1cm} (\hspace{0.1cm}x\hspace{0.2cm}=\hspace{0.2cm}y\hspace{0.1cm}) \hspace{0.3cm} \Rightarrow \hspace{0.3cm} x\hspace{0.3cm}<\hspace{0.3cm}y$
\end{center} \vspace{2cm} 

{\red{\underline{Demostración:}}}
\newpage
Vamos a demostrar por método directo. Suponemos verdadero el antecedente: 

\begin{center}
    $x \hspace{0.3cm} \leq \hspace{0.3cm} y \hspace{0.3cm} \wedge \hspace{0.3cm}  \neg \hspace{0.1cm} (\hspace{0.1cm}x\hspace{0.2cm}=\hspace{0.2cm}y\hspace{0.1cm}) \hspace{0.3cm} \Leftrightarrow \hspace{0.3cm} x \hspace{0.3cm} \leq \hspace{0.3cm} y \hspace{0.3cm} \wedge \hspace{0.3cm} (\hspace{0.1cm}x\hspace{0.2cm}<\hspace{0.2cm}y\hspace{0.1cm})$ 
    
    \vspace{1cm}
    
    \hspace{9cm}Por definición de $\neg \hspace{0.1cm}(=)$
\end{center}
\vspace{1cm}

Por transitiva de $\Leftrightarrow$:\\

\begin{center}
    $x \hspace{0.3cm} \leq \hspace{0.3cm} y \hspace{0.3cm} \wedge \hspace{0.3cm}  \neg \hspace{0.1cm} (\hspace{0.1cm}x\hspace{0.2cm}=\hspace{0.2cm}y\hspace{0.1cm}) \hspace{0.3cm} \Leftrightarrow \hspace{0.3cm} x \hspace{0.3cm} \leq \hspace{0.3cm} y \hspace{0.3cm} \wedge \hspace{0.3cm} (\hspace{0.1cm}x\hspace{0.2cm}<\hspace{0.2cm}y\hspace{0.1cm})$ 
\end{center}\vspace{0.1cm}

\begin{center}
    $x \hspace{0.3cm} \hspace{0.3cm} \leq \hspace{0.3cm} y \hspace{0.3cm} \wedge \hspace{0.3cm} (\hspace{0.1cm}x\hspace{0.2cm}<\hspace{0.2cm}y\hspace{0.1cm}) \Leftrightarrow \hspace{0.3cm} (\hspace{0.1cm}x\hspace{0.2cm}<\hspace{0.2cm}y\hspace{0.1cm})$ \hspace{.5cm} \textup{\black{\qedsymbol}} 
\end{center}\vspace{1cm}

\end{document} 
