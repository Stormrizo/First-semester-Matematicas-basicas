\documentclass[12pt]{article} 
\usepackage[utf8]{inputenc}
\usepackage[spanish]{babel}
\usepackage{amsfonts}
\usepackage{amsmath, amsthm, amssymb}
\usepackage{graphicx}
\usepackage[left=2.54cm,right=2.54cm,top=2.54cm,bottom=2.54cm]{geometry}
\usepackage{pstricks}
\begin{document}

\thispagestyle{empty} 
\begin{center} \LARGE{\bf Benemérita Universidad Autónoma de Puebla} \\[0.5cm]
\begin{figure}[htb] \centering \includegraphics[scale=.2]{LogoBUAPpng.png} \end{figure}
\LARGE{Facultad de Ciencias Físico Matemáticas}\\[0.5cm]
\begin{figure}[htb] \centering \includegraphics[scale=.39]{LogoFCFMBUAP.png} \end{figure} 
\Large{Licenciatura en Física Teórica}\\[0.5cm]
\large{Primer semestre} \end{center}
\begin{center} { \Large \bfseries{Tarea 12 (repaso)}} \\ \end{center}
\large{\bf Curso:} Matemáticas básicas \textbf{(N.R.C.:25598)}\\
\large{\bf Alumno:} Julio Alfredo Ballinas García $\left(202107583\right)$ \\
\large{\bf Docente:} Dra. María Araceli Juárez Ramírez\\
\large{\bf Grupo:} 102\\ \begin{center} 
\vfill
\textsc{16 de septiembre de 2021} \end{center}  

\newpage
\sffamily

{\LARGE{ Resolver los ejercicios indicados en clase \\ }}

\sffamily
Sean los conjuntos: 

\begin{enumerate}
        \item [I.] $P= \left\{x \in \mathbb{Z} \mid 2x^2+5x-3=0\right\}$
\end{enumerate}

\begin{enumerate}
        \item [II.] $M= \left\{x \in \mathbb{N} \mid x=4k \wedge -4<x<21\right\}$
\end{enumerate}
\begin{enumerate}
        \item [III.]  $T= \left\{x \in \mathbb{R} \mid (x^2-9)(x-4)=0\right\}$ \\
\end{enumerate}

{{\red{{\underline{Calcular:\\}}}}}

\begin{enumerate}
        \item [\blue{a)}] $M-(T-P)$
\end{enumerate}
\begin{enumerate}
        \item [\blue{b)}] $P(M-T)$ $\leftarrow$ $Partes$ $de$  $M-T$
\end{enumerate}
\begin{enumerate}
        \item [\blue{c)}] $(M \cup T)$ $-$ $P$\\
\end{enumerate}
{\textbf{Antes de calcular las operaciones debemos expresar los conjuntos en su forma extensiva:\\}}

\underline{Para el conjunto I. tenemos:}

\begin{center}
     $P= \left\{x \in \mathbb{Z} \mid 2x^2+5x-3=0\right\}$
\end{center}

Resolviendo $2x^2+5x-3=0$ para hallar las raíces que hacen que la expresión sea igual a $0$.

\begin{equation}
    \begin{split}
        2x^2+5x-3=0 & =  \frac{2}{2}\cdot(2x^2+5x-3=0) \\\\
         & =  \frac{4x^2+5(2x)-6=0}{2}\\\\ \textbf{Factorización}
        & = \frac{(2x+6)(2x-1)}{2}\\\\
         & = (x+3)(2x-1)=0\\\\ \textbf{\blue{Esto es}}
          & = (x+3=0) \vee (2x-1=0)\\\\ \textbf{Despejando x}
          & = (x=-3) \vee (x=\frac{1}{2})\\\\
    \end{split}
\end{equation}
Como $\frac{1}{2}$ $\notin$ a los números enteros $(\mathbb{Z})$ entonces, el único elemento que pertence a $P= \left\{x \in \mathbb{Z} \mid 2x^2+5x-3=0\right\}$ es $-3$.\\

En forma extensiva el conjunto $P= \left\{x \in \mathbb{Z} \mid 2x^2+5x-3=0\right\}$ es:

\begin{center}
     $P= \left\{-3\right\}$
\end{center}

\underline{Para el conjunto II. tenemos:}

\begin{center}
   $M= \left\{x \in \mathbb{N} \mid x=4k \wedge -4<x<21\right\}$
\end{center}

Calulemos $x=4k$ para hallar los elementos del conjunto M:\\

$x=4k$ está definida en el intervalo $(-4,21)$, también debe pertencer al conjunto de los naturales:

\begin{center}
   $\mathbb{N}= \left\{1,2,3,4,5,6,7...+\infty \right\}$
\end{center}
\newpage
Los únicos valores que cumplen con las condiciones son los siguientes:
\begin{center}
   $8, 12, 16$ y $20$
\end{center}

En forma extensiva el conjunto  $M= \left\{x \in \mathbb{N} \mid x=4k \wedge -4<x<21\right\}$ es:
\begin{center}
     $M= \left\{8, 12, 16, 20\right\}$ 
\end{center}

\underline{Para el conjunto III. tenemos:}

\begin{center}
    $T= \left\{x \in \mathbb{R} \mid (x^2-9)(x-4)=0\right\}$ 
\end{center}
 
Resolviendo $(x^2-9)(x-4)=0$ para hallar las raíces que hacen que la expresión sea igual a $0$.

\begin{equation}
    \begin{split}
   (x^2-9)(x-4)=0 & = (x^2-9=0) \vee (x-4=0) \\\\ \textbf{Diferencia de cuadrados}
         & =  (x+3=0)\vee(x-3=0) \vee (x-4=0)\\\\ \textbf{Despejando x}
        & = (x=-3)\vee(x=3) \vee (x=4)\\\\
    \end{split}
\end{equation}

En forma extensiva el conjunto  $T= \left\{x \in \mathbb{R} \mid (x^2-9)(x-4)=0\right\}$  es:
\begin{center}
    $T= \left\{-3,3,4\right\}$ 
\end{center}

Ahora ponemos u organizamos los conjuntos en forma extensiva hallados con anterioridad. 

\begin{center}
     $P= \left\{-3\right\}$
\end{center}
\begin{center}
     $M= \left\{8, 12, 16, 20\right\}$ 
\end{center}

\begin{center}
    $T= \left\{-3,3,4\right\}$ 
\end{center}

En este punto ya podemos realizar el cálculo de $\blue{a)}$, $\blue{b)}$ y $\blue{c)}$

\newpage

{\red{Solución:}}

\begin{enumerate}
        \item [\blue{a)}] $M-(T-P)$
\end{enumerate}

Definición de diferencia de conjuntos:

\begin{center}
     $A-B= \left\{x\mid x \in A \wedge x \notin  B \right\}$
\end{center}
\begin{center}
    Esto es: 
\end{center}
\begin{center}
    $A-B= \left\{x\mid x \in A \wedge x \in  B^{C} \right\}$ 
\end{center}

Sean los conjuntos:

\begin{center}
     $P= \left\{-3\right\}$ 
\end{center}
\begin{center}
     $M= \left\{8, 12, 16, 20\right\}$ 
\end{center}
\begin{center}
     y
\end{center}
\begin{center}
     $T= \left\{-3,3,4\right\}$  
\end{center}
La operación es $M-(T-P)$:

\begin{equation}
    \begin{split}
        M-(T-P) & = \{8, 12, 16, 20\} - (\{-3,3,4\}-\{-3\}) \\\\ \textbf{Def. de diferencia}
         & =  \{8, 12, 16, 20\} - (\{3,4\})\\\\ \textbf{Def. de diferencia}
        & = \{8, 12, 16, 20\} \\\\
    \end{split}
\end{equation}
Finalmente la operación $M-(T-P)$ es:

\begin{center}
   $M-(T-P)=\{8, 12, 16, 20\}$
\end{center}
\newpage
\begin{enumerate}
        \item [\blue{b)}]$P(M-T)$ $\leftarrow$ $Partes$ $de$  $M-T$
\end{enumerate}

Sean:

\begin{center}
   $M=\{8, 12, 16, 20\}$ 
\end{center}
\begin{center}
    y
\end{center}
\begin{center}
    $T = \{-3, 3, 4\}$
\end{center}

Realizamos la operación $M-T$:

\begin{equation}
    \begin{split}
        M-T & = \{8, 12, 16, 20\} - \{-3,3,4\} \\\\ \textbf{Def. de diferencia}
         & =  \{8, 12, 16, 20\}\\\\ 
    \end{split}
\end{equation}

Entonces $P(M-T)$ es:

\begin{center}
    $P(M-T) = P(\{8, 12, 16, 20\})$
\end{center}

Cardinalidad de $P$ es: 

\begin{center}
    $Card(P) = Card(\{8, 12, 16, 20\})= 2^4 = 16$
\end{center}

El conjunto potencia tendrá 16 elementos, los cuales son:

\begin{enumerate}
    \item [1)] $(M-T)_1 = \{\varnothing\} $  
    \item [2)] $(M-T)_2 = \{8\} $
    \item [3)] $(M-T)_3 = \{12\} $
    \item [4)] $(M-T)_4 = \{16\} $
    \item [5)] $(M-T)_5 = \{20\} $
    \item [6)] $(M-T)_6 = \{8, 12\} $
    \item [7)] $(M-T)_7 = \{8, 16\} $
    \item [8)] $(M-T)_8 = \{8, 20\} $
    \item [9)] $(M-T)_9 = \{12,16\} $
    \item [10)] $(M-T)_1_0 = \{12, 20\} $
    \item [11)] $(M-T)_1_1 = \{16, 20\} $
    \item [12)] $(M-T)_1_2 = \{8, 12, 16\} $
    \item [13)] $(M-T)_1_3 = \{8, 12,20\} $
    \item [14)] $(M-T)_1_4 = \{8, 16, 20\} $
    \item [15)] $(M-T)_1_5 = \{12, 16, 20\} $
    \item [16)] $(M-T)_1_6 = \{(M-T)\}$\\ \\
\end{enumerate}

\begin{enumerate}
        \item [\blue{c)}]$(M \cup T)$ $-$ $P$
\end{enumerate}

Definición de unión de conjuntos: 
\begin{center}
    $A \cup B = \left\{x \in A \vee x \in B\right\}$
\end{center}

Sean los conjuntos:

\begin{center}
    $P= \left\{-3\right\}$
\end{center}

\begin{center}
   $M=\{8, 12, 16, 20\}$
\end{center}

\begin{center}
    y
\end{center}

\begin{center}
   $T = \{-3, 3, 4\}$
\end{center}

Entonces la operación $(M \cup T)$ $-$ $P$ es:

\begin{equation}
    \begin{split}
        (M \cup T) - P & = (\{8, 12, 16, 20\} \cup \{-3, 3, 4\})-\{-3\}  \\\\ \textbf{Def. de unión}
         & =  \{-3, 3, 4, 8, 12, 16, 20\} - \{-3\}\\\\ \textbf{Def. de diferencia}
        & =  \{3, 4, 8, 12, 16, 20\} \\\\
    \end{split}
\end{equation}

Entonces la operación $(M \cup T)$ $-$ $P$ es:

\begin{center}
    $(M \cup T)$ $-$ $P$ $=$  $\{3, 4, 8, 12, 16, 20\}$
\end{center}

\newpage
Hola Doctora María Araceli, espero no afecte en la calificación de esta tarea el haberlo realizado en digital, lo que sucede es que estaba fuera de casa y llevé mi laptop conmigo y como no tenía mis útiles escolares decidí hacerlo por este medio.\\

Saludos profesora... ¡Feliz día de la Independencia! ¡Viva México! :) \\

\begin{figure}[htb] \centering \includegraphics[scale=.5]{mexico.jpeg} \end{figure}


\end{document}
