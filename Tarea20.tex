\documentclass[12pt]{article} 
\usepackage[utf8]{inputenc}
\usepackage[spanish]{babel}
\usepackage{pdfpages}
\usepackage{parskip}
\usepackage{float}
\usepackage{enumitem}
\usepackage{multicol}
\newenvironment{Figura}
  {\par\medskip\noindent\minipage{\linewidth}}
  {\endminipage\par\medskip}
\usepackage{caption}
\usepackage{amsfonts}
\usepackage{amsmath, amsthm, amssymb}
\renewcommand{\qedsymbol}{$\blacksquare$}
\usepackage{graphicx}
\usepackage[left=2.54cm,right=2.54cm,top=2.54cm,bottom=2.54cm]{geometry}
\usepackage{pstricks}
\usepackage{xcolor}
\definecolor{verde_manzana}{rgb}{0.55, 0.71, 0.0}
\definecolor{Aguamarina}{rgb}{0.5, 1.0, 0.83}
\definecolor{mandarina_atomica}{rgb}{1.0, 0.6, 0.4}
\definecolor{blizzardblue}{rgb}{0.67, 0.9, 0.93}
\definecolor{bluegray}{rgb}{0.4, 0.6, 0.8}
\definecolor{coolgrey}{rgb}{0.55, 0.57, 0.67}
\definecolor{tealgreen}{rgb}{0.0, 0.51, 0.5}
\definecolor{ticklemepink}{rgb}{0.99, 0.54, 0.67}
\definecolor{thulianpink}{rgb}{0.87, 0.44, 0.63}
\definecolor{wildwatermelon}{rgb}{0.99, 0.42, 0.52}
\definecolor{wisteria}{rgb}{0.79, 0.63, 0.86}
\definecolor{yellow(munsell)}{rgb}{0.94, 0.8, 0.0}
\definecolor{trueblue}{rgb}{0.0, 0.45, 0.81}	\definecolor{tropicalrainforest}{rgb}{0.0, 0.46, 0.37}
\definecolor{tearose(rose)}{rgb}{0.96, 0.76, 0.76}
\definecolor{antiquefuchsia}{rgb}{0.57, 0.36, 0.51}	\definecolor{bittersweet}{rgb}{1.0, 0.44, 0.37}	\definecolor{carrotorange}{rgb}{0.93, 0.57, 0.13}
\definecolor{cinereous}{rgb}{0.6, 0.51, 0.48}\definecolor{darkcoral}{rgb}{0.8, 0.36, 0.27}	\definecolor{orange(colorwheel)}{rgb}{1.0, 0.5, 0.0}
\definecolor{palatinateblue}{rgb}{0.15, 0.23, 0.89} \definecolor{pakistangreen}{rgb}{0.0, 0.4, 0.0}

\begin{document}
\pagestyle{empty} 
\setlength{\parindent}{0pt}
\sffamily
\begin{center} \LARGE{\bf Benemérita Universidad Autónoma de Puebla} \\[0.5cm]
\begin{figure}[htb] \centering \includegraphics[scale=.2]{LogoBUAPpng.png} \end{figure}
\LARGE{Facultad de Ciencias Físico Matemáticas}\\[0.5cm]
\begin{figure}[htb] \centering \includegraphics[scale=.39]{LogoFCFMBUAP.png} \end{figure} 
\Large{Licenciatura en Física Teórica}\\[0.5cm]
\large{Primer semestre} \end{center}
\begin{center} { \Large \bfseries{Tarea 20}: (Soluciones de inecuaciones)} \\ \end{center}
\large{\bf Curso:} Matemáticas básicas \textbf{(N.R.C.:25598)}\\
\large{\bf Alumno:} Julio Alfredo Ballinas García $\left(202107583\right)$ \\
\large{\bf Docente:} Dra. María Araceli Juárez Ramírez\\
\large{\bf Grupo:} 102\\ \begin{center} 
\vfill
\textsc{7 de octubre de 2021} \end{center}

\newpage
\begin{center}

{\section*{\LARGE{\textcolor{black}{Resolver los incisos {\textcolor{pakistangreen}{c)}} y {\textcolor{palatinateblue}{f)}} del {\textcolor{wildwatermelon}{\underline{Ejercicio 3.}}}}}}}\end{center} \vspace{2.2cm}

\section{{\textcolor{wildwatermelon}{{\underline{Ejercicio 3.}}}} Inciso {\textcolor{pakistangreen}{c)}} {\LARGE{$\frac{(3-2x)^{3} (x-5)}{(7x-1) (3x+4)^{2}} \geq 0$ {\blue{$^{+}$}}}}} \vspace{.6cm}

\begin{enumerate}
\item {\textcolor{coolgrey}{Hallar el dominio de la inecuación ($D_i$).}}
\item {\textcolor{coolgrey}{Hallar las x que satisfacen la inecuación.}}
\end{enumerate} \vspace{.6cm}

1. {\textcolor{coolgrey}{Hallar el dominio de la inecuación ($D_i$).}}\vspace{0.5cm}

Por {\textcolor{carrotorange}{teorema 5}}: hallaremos los valores que no están definidos para el cociente de la desigualdad. \vspace{0.5cm}

$(7x-1)(3x+4)^{2}$ $=$ $0$ $\Longleftrightarrow$ $(7x-1)$ $=$ $0$ $\vee$ $(3x+4)^{2}$ $=$ $0$\vspace{0.3cm}

$\Longleftrightarrow$ $7x-1$ $=$ $0$ $\Longleftrightarrow$ $7x$ $=$ $1$ $\Longleftrightarrow$ \fbox{$x$ $=$ $\frac{1}{7}$} $\longleftarrow$ \hspace{0.1cm} {\footnotesize{No está definida para ese valor}} \vspace{0.3cm}

$\vee$  $3x+4$ $=$ $0$ $\Longleftrightarrow$ $3x$ $=$ $-4$ $\Longleftrightarrow$ \fbox{$x$ $=$ $\frac{-4}{3}$} $\longleftarrow$ \hspace{0.1cm} {\footnotesize{No está definida para ese valor}} \vspace{0.3cm}

El dominio de la inecuación denotado por $D_i:$ es \vspace{0.5cm}

\hspace{4cm} $\mathbb{R}$ $-$ {\LARGE{$\left\{\frac{1}{7}, \frac{-4}{3}\right\}$}} 
2. {\textcolor{coolgrey}{Hallar las x que satisfacen la inecuación.}}\vspace{0.5cm}

Determinado el dominio de la desigualdad, iniciamos el estudio de signos de cada factor que aparece en la inecuación.

Estos factores o raíces son: 

\begin{center}
    {\LARGE{$\frac{3}{2}$}}, $5$, {\LARGE{$\frac{1}{7}$}}, {\LARGE{$-\frac{4}{3}$}}
\end{center} \newpage

\begin{table}
\centering
\begin{tabular}{l|l|l|l|l|l|l|l}
\multicolumn{8}{l}{\hspace{2cm} x \hspace{2.4cm} {\tiny{-4/3}} \hspace{.2cm} {\tiny{1/7}} \hspace{.05cm} {\tiny{3/2}} \hspace{.2cm} {\tiny{5}} }                                                      \\ 
\cline{2-7}
                        & 1. $3-2x$                & {\blue{+}} & {\blue{+}} & {\blue{+}} & {\red{$-$}} & {\red{$-$}} &   \\ 
\cline{2-7}
                        & 2. $(3-2x)^{3}$          & {\blue{+}} & {\blue{+}} & {\blue{+}} & {\red{$-$}} & {\red{$-$}} &   \\ 
\cline{2-7}
                        & 3. $x-5$                 & {\red{$-$}} & {\red{$-$}} & {\red{$-$}} & {\red{$-$}} & {\blue{+}} &   \\ 
\cline{2-7}
                        & 4. $7x-1$                & {\red{$-$}} & {\red{$-$}} & {\blue{+}} & {\blue{+}} & {\blue{+}} &   \\ 
\cline{2-7}
                        & 5. $3x+4$                & {\red{$-$}} & {\blue{+}} & {\blue{+}} & {\blue{+}} & {\blue{+}} &   \\ 
\cline{2-7}
                        & 6. $(3x+4)^{2}$          & {\blue{+}} & {\blue{+}} & {\blue{+}} & {\blue{+}} & {\blue{+}} &   \\ 
\cline{1-7}
\multicolumn{1}{|l|}{N} & 7. $(3-2x)^{3}$ $(x-5)$  & {\red{$-$}} & {\red{$-$}} & {\red{$-$}} & {\blue{+}} & {\red{$-$}} &   \\ 
\cline{1-7}
\multicolumn{1}{|l|}{D} & 8. $(7x-1)$ $(3x+4)^{2}$ & {\red{$-$}}& {\red{$-$}} & {\blue{+}} & {\blue{+}} & {\blue{+}} &   \\ 
\cline{1-7}
                        & N/D                      & {\blue{+}} & {\blue{+}} & {\red{$-$}} & {\blue{+}} & {\red{$-$}} &   \\
\cline{2-7}
\end{tabular}
\end{table} \vspace{1cm}

Solución: 

\begin{center}
    $ (-\infty, -\frac{4}{3}\hspace{0.2cm}[\hspace{0.2cm}\cup$ $]-\frac{4}{3}, \frac{1}{7}\hspace{0.2cm}[$ $\cup\hspace{0.1cm} [\hspace{0.2cm}\frac{3}{2}, 5 \hspace{0.2cm}]$
\end{center} \vspace{.5cm}

\section{{\textcolor{wildwatermelon}{{\underline{Ejercicio 3.}}}} Inciso {\textcolor{palatinateblue}{f)}} {\LARGE{$\frac{(2-5x) (2x-4)}{(x-7) (2-3x)} \leq 0$ {\red{$^{-}$}}}}} \vspace{0.1cm}

\begin{enumerate}
\item {\textcolor{coolgrey}{Hallar el dominio de la inecuación ($D_i$).}}
\item {\textcolor{coolgrey}{Hallar las x que satisfacen la inecuación.}}
\end{enumerate} \vspace{.6cm}

1. {\textcolor{coolgrey}{Hallar el dominio de la inecuación ($D_i$).}}\vspace{0.5cm}

Por {\textcolor{carrotorange}{teorema 5}}: hallaremos los valores que no están definidos para el cociente de la desigualdad. \vspace{0.5cm}

$(x-7)(2-3x)$ $=$ $0$ $\Longleftrightarrow$ $(x-7)$ $=$ $0$ $\vee$ $(2-3x)$ $=$ $0$\vspace{0.3cm}

$\Longleftrightarrow$ $x-7$ $=$ $0$ $\Longleftrightarrow$ \fbox{$x$ $=$ $7$} $\longleftarrow$ \hspace{0.1cm} {\footnotesize{No está definida para ese valor}} \vspace{0.3cm}

$\vee$  $2-3x$ $=$ $0$ $\Longleftrightarrow$ $2$ $=$ $3x$ $\Longleftrightarrow$ \fbox{ $\frac{2}{3}$ $=$ $x$} $\longleftarrow$ \hspace{0.1cm} {\footnotesize{No está definida para ese valor}} \vspace{0.3cm}

El dominio de la inecuación denotado por $D_i:$ es \vspace{0.5cm}

\hspace{4cm} $\mathbb{R}$ $-$ {\LARGE{$\left\{\frac{2}{3}, 7\right\}$}} \vspace{0.5cm} 

\newpage

2. {\textcolor{coolgrey}{Hallar las x que satisfacen la inecuación.}}

\vspace{0.5cm}

Determinado el dominio de la desigualdad, iniciamos el estudio de signos de cada factor que aparece en la inecuación. 

\vspace{0.1cm}

Estos factores o raíces son: 

\vspace{0.7cm}

\begin{center}
    {\LARGE{$\frac{2}{5}$}}, $2$, $7$, {\LARGE{$\frac{2}{3}$}}
\end{center} 

\vspace{0.7cm}


\begin{table}[H]
\centering
\begin{tabular}{l|l|l|l|l|l|l|l}
\multicolumn{8}{l}{\hspace{2cm} x \hspace{2.3cm} {\tiny{2/5}} \hspace{.1cm} {\tiny{2/3}} \hspace{.19cm} {\tiny{2}} \hspace{.35cm} {\tiny{7}} }                                                      \\ 
\cline{2-7}
                        & 1. $2-5x$ & {\blue{+}}  & {\red{$-$}} & {\red{$-$}} & {\red{$-$}} & {\red{$-$}} &   \\ 
\cline{2-7}
                        & 2. $2x-4$ & {\red{$-$}} & {\red{$-$}}  & {\red{$-$}} & {\blue{+}}  & {\blue{+}} &   \\ 
\cline{2-7}
                        & 3. $x-7$  & {\red{$-$}}  & {\red{$-$}}& {\red{$-$}} & {\red{$-$}}& {\blue{+}} &   \\ 
\cline{2-7}
                        & 4. $2-3x$ & {\blue{+}}   & {\blue{+}} & {\red{$-$}} & {\red{$-$}} & {\red{$-$}} &   \\ 
\cline{1-7}
\multicolumn{1}{|l|}{N} & 5. $(2-5x)(2x-4)$    & {\red{$-$}} & {\blue{+}} & {\blue{+}} & {\red{$-$}} & {\red{$-$}} &   \\ 
\cline{1-7}
\multicolumn{1}{|l|}{D} & 6. $(x-7)(2-3x)$   & {\red{$-$}} & {\red{$-$}}  & {\blue{+}}  & {\blue{+}}  & {\red{$-$}} &   \\ 
\cline{1-7}
                        & 7. N/D & {\blue{+}}  & {\red{$-$}}  & {\blue{+}}  & {\red{$-$}} & {\blue{+}}  &   \\ 
\cline{2-7}
\multicolumn{1}{l}{}    & \multicolumn{1}{l}{} & \multicolumn{1}{l}{} & \multicolumn{1}{l}{} & \multicolumn{1}{l}{} & \multicolumn{1}{l}{} & \multicolumn{1}{l}{} &   \\
\multicolumn{1}{l}{}    & \multicolumn{1}{l}{} & \multicolumn{1}{l}{} & \multicolumn{1}{l}{} & \multicolumn{1}{l}{} & \multicolumn{1}{l}{} & \multicolumn{1}{l}{} &  
\end{tabular}
\end{table} \vspace{1cm}

Solución: 

\begin{center}
    $ (\frac{2}{5}, \frac{2}{3}\hspace{0.2cm}[\hspace{0.2cm}\cup$ $[\hspace{0.2cm} 2, 7\hspace{0.2cm}[$
\end{center} 
\end{document}

