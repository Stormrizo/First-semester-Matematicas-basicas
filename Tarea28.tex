\documentclass[12pt]{article} 
\usepackage[utf8]{inputenc}
\usepackage[spanish]{babel}
\usepackage{pdfpages}
\usepackage{csquotes}
\usepackage{afterpage}
\usepackage{parskip}
\usepackage{float}
\usepackage{enumitem}
\usepackage{multicol}
\newenvironment{Figura}
  {\par\medskip\noindent\minipage{\linewidth}}
  {\endminipage\par\medskip}
\usepackage{caption}
\usepackage{amsfonts}
\usepackage{amsmath, amsthm, amssymb}
\renewcommand{\qedsymbol}{$\blacksquare$}
\usepackage{graphicx}
\usepackage[left=2.54cm,right=2.54cm,top=2.54cm,bottom=2.54cm]{geometry}
\usepackage{pstricks} 
\usepackage{xcolor}
\definecolor{prussianblue}{RGB}{1, 45, 75} 
\definecolor{brightturquoise}{RGB}{1, 196, 254} 
\definecolor{verde_manzana}{rgb}{0.55, 0.71, 0.0}
\definecolor{Aguamarina}{rgb}{0.5, 1.0, 0.83}
\definecolor{mandarina_atomica}{rgb}{1.0, 0.6, 0.4}
\definecolor{blizzardblue}{rgb}{0.67, 0.9, 0.93}
\definecolor{bluegray}{rgb}{0.4, 0.6, 0.8}
\definecolor{coolgrey}{rgb}{0.55, 0.57, 0.67}
\definecolor{tealgreen}{rgb}{0.0, 0.51, 0.5}
\definecolor{ticklemepink}{rgb}{0.99, 0.54, 0.67}
\definecolor{thulianpink}{rgb}{0.87, 0.44, 0.63}
\definecolor{wildwatermelon}{rgb}{0.99, 0.42, 0.52}
\definecolor{wisteria}{rgb}{0.79, 0.63, 0.86}
\definecolor{yellow(munsell)}{rgb}{0.94, 0.8, 0.0}
\definecolor{trueblue}{rgb}{0.0, 0.45, 0.81}	\definecolor{tropicalrainforest}{rgb}{0.0, 0.46, 0.37}
\definecolor{tearose(rose)}{rgb}{0.96, 0.76, 0.76}
\definecolor{antiquefuchsia}{rgb}{0.57, 0.36, 0.51}	\definecolor{bittersweet}{rgb}{1.0, 0.44, 0.37}	\definecolor{carrotorange}{rgb}{0.93, 0.57, 0.13}
\definecolor{cinereous}{rgb}{0.6, 0.51, 0.48}
\definecolor{darkcoral}{rgb}{0.8, 0.36, 0.27}	\definecolor{orange(colorwheel)}{rgb}{1.0, 0.5, 0.0}
\definecolor{palatinateblue}{rgb}{0.15, 0.23, 0.89} \definecolor{pakistangreen}{rgb}{0.0, 0.4, 0.0} 	\definecolor{vividviolet}{rgb}{0.62, 0.0, 1.0} 
\definecolor{tigre}{rgb}{0.88, 0.55, 0.24} 		\definecolor{plum(traditional)}{rgb}{0.56, 0.27, 0.52} 	\definecolor{persianred}{rgb}{0.8, 0.2, 0.2} 	\definecolor{orange(webcolor)}{rgb}{1.0, 0.65, 0.0} 	\definecolor{onyx}{rgb}{0.06, 0.06, 0.06}
\definecolor{blue-violet}{rgb}{0.54, 0.17, 0.89}
\definecolor{byzantine}{rgb}{0.74, 0.2, 0.64}
\definecolor{byzantium}{rgb}{0.44, 0.16, 0.39}
\definecolor{darkmagenta}{rgb}{0.55, 0.0, 0.55} 	\definecolor{darkviolet}{rgb}{0.58, 0.0, 0.83} 	\definecolor{deepmagenta}{rgb}{0.8, 0.0, 0.8}
\newenvironment{MyColorPar}[1]{%
    \leavevmode\color{#1}\ignorespaces%
}{%
}%

\begin{document}

\begingroup
\begin{titlepage}
	\AddToShipoutPicture*{\put(79,350){\includegraphics[scale=.3]{descarga.png}}}
	\noindent
	\vspace{1mm}
\end{titlepage}
\endgroup

\pagestyle{empty} 
\setlength{\parindent}{0pt}
\sffamily

%%%%%%%%%%%%%%%%%%%%%%%%%%%%%%%%%%%%%%%%%%%%%%%%%%%%%%%%%%%%%%%%%%%
%%%%%%%%%%%%%%%%%%%%%%%%%%%%%%%%%%%%%%%%%%%%%%%%%%%%%%%%%%%%%%%%%%%

\begin{center} 

    \LARGE{\bf{\textsf{Benemérita Universidad Autónoma de Puebla}}} \\[0.5cm]
    
\begin{figure}[htb] \centering

    \includegraphics[scale=.25]{LogoBUAPpng.png} 

\end{figure}

%%%%%%%%%%%%%%%%%%%%%%%%%%%%%%%%%%%%%%%%%%%%%%%%%%%%%%%%%%%%%%%%%%%
%%%%%%%%%%%%%%%%%%%%%%%%%%%%%%%%%%%%%%%%%%%%%%%%%%%%%%%%%%%%%%%%%%%

    \LARGE{Facultad de Ciencias Físico Matemáticas}\\[0.5cm]

\begin{figure}[htb] \centering

    \includegraphics[scale=.4]{LogoFCFMBUAP.png} 
    
\end{figure} 

%%%%%%%%%%%%%%%%%%%%%%%%%%%%%%%%%%%%%%%%%%%%%%%%%%%%%%%%%%%%%%%%%%%
%%%%%%%%%%%%%%%%%%%%%%%%%%%%%%%%%%%%%%%%%%%%%%%%%%%%%%%%%%%%%%%%%%%

    \Large{Licenciatura en Física Teórica}\\[0.5cm]
    \Large{Primer semestre} 

\end{center} \vspace{0.3cm}
%%%%%%%%%%%%%%%%%%%%%%%%%%%%%%%%%%%%%%%%%%%%%%%%%%%%%%%%%%%%%%%%%%%
%%%%%%%%%%%%%%%%%%%%%%%%%%%%%%%%%%%%%%%%%%%%%%%%%%%%%%%%%%%%%%%%%%%

\begin{center}

    {\Large{\bfseries{{\textcolor{carrotorange}{Tarea 28 (Inducción)}}}}} \\ 
    
\end{center}

    \large{\bf{\textsf{Curso:}}} {\bfseries{{\textcolor{brightturquoise}{Matemáticas básicas \bfseries{(N.R.C.:25598)}}}}} \\
    \large{\bf{\textsf{Alumno:}}} {\bfseries{{\textcolor{prussianblue}{Julio Alfredo Ballinas García {\large{{$\mid$}}} 202107583}}}}  \\
    \large{\bf{\textsf{Docente:}}} {\bfseries{{\textcolor{wisteria}{Dra. María Araceli Juárez Ramírez}}}}\\
    \large{\bf{\textsf{Grupo:}}} {\bfseries{{\textcolor{verde_manzana}{102}}}}\\

\vfill
    
\begin{center} 

    {\small{\textsf{\underline{Tarea retrasada:} venció 28 de octubre {\red{23:59 p.m.}}} {\LARGE{ $\mid$ }}\textsf{{\underline{Fecha de hoy:}} 29 de octubre}}}
    
\end{center}

\newpage

%%%%%%%%%%%%%%%%%%%%%%%%%%%%%%%%%%%%%%%%%%%%%%%%%%%%%%%%%%%%%%%%%%%
%%%%%%%%%%%%%%%%%%%%%%%%%%%%%%%%%%%%%%%%%%%%%%%%%%%%%%%%%%%%%%%%%%%

\section{\textsf{Mostrar que [1, $\infty$) satisface a) y b)}} \vspace{0.5cm}

{\LARGE{[1, $\infty$) $=$ $\left\{ x \in \mathbb{R} \mid x \geq 1  \right\}$ }} \vspace{0.5cm}

{\red{Solución:}} \begin{MyColorPar}{verde_manzana}
Debemos probar que se cumplen las siguientes propiedades: 
\end{MyColorPar} \vspace{0.5cm}
\begin{center}
\begin{MyColorPar}{palatinateblue}
{\underline{a) 1 $\in$ $\mathbb{N}$}} \hspace{0.3cm} y \hspace{0.3cm} {\underline{b) si $x$ $\in$ $\mathbb{N}$ $\Longrightarrow$ $x$ $+$ $1$ $\in$ $\mathbb{N}$}}
\end{MyColorPar} \vspace{0.5cm}
\end{center}

Del intervalo [1, $\infty$) $=$ $\left\{ x \in \mathbb{R} \mid x \geq 1 \right\}$ podemos observar claramente que $0$ $\notin$ [1, $\infty$) pues $0$ $<$ $1$ \hspace{0.2cm} $\wedge$ \hspace{0.2cm} $0$ $\neq$ $1$. \vspace{0.2cm}

Por otro lado 1 sí $\in$ [1, $\infty$) ya que \hspace{0.2cm} 1 $\geq$ 1 \hspace{0.2cm} $\Longleftrightarrow$ \hspace{0.2cm} 1 $>$ 1 \hspace{0.2cm} $\vee$ \hspace{0.2cm} 1 $=$ 1 \mbox{{\textcolor{carrotorange}{por teorema 14 (T$_{14}$)}}} y al ser una disyunción 1 $=$ 1 es verdadera. \vspace{0.5cm}

Supongamos que $x$ $\in$ [1, $\infty$) \hspace{0.2cm} $\Longrightarrow$ \hspace{0.2cm} $x$ $+$ $1$ $\in$ [1, $\infty$] \vspace{0.5cm} 

Como $x$ $\in$ [1, $\infty$) es verdadera por hipótesis. \vspace{0.5cm}

$x$ $\in$ [1, $\infty$) \hspace{0.2cm} $\Longleftrightarrow$ \hspace{0.2cm} $x$ $\geq$ $1$ \hspace{0.2cm} $\wedge$ \hspace{0.2cm} $1$ $=$ $1$ {\textcolor{carrotorange}{por teorema 18 i) (T$_{18}$)}} partiendo de \mbox{$x$ $\geq$ $1$} y sumando $1$ a ambos lados de la desigualdad tenemos: \vspace{0.5cm}

\hspace{4cm} $x$ $\geq$ $1$  \hspace{0.2cm} $\Longleftrightarrow$  \hspace{0.2cm} $x$ $+$ $1$ $\geq$ $1$ $+$ $1$ \vspace{0.5cm}

\hspace{5.8cm} $\Longleftrightarrow$  \hspace{0.2cm} $x$ $+$ $1$ $\geq$ $2$ $>$ $0$ \hspace{0.5cm} {\textcolor{carrotorange}{Por teorema 27 ($T_{27}$)}}     \vspace{0.5cm} 

\hspace{5.8cm} $\Longleftrightarrow$  \hspace{0.2cm} $x$ $+$ $1$ $\geq$ $2$ \hspace{0.2cm} $\wedge$ \hspace{0.2cm} $2$ $>$ $0$      \vspace{0.5cm} 

\hspace{5.8cm} $\Longleftrightarrow$  \hspace{0.2cm} $x$ $+$ $1$  $>$ $0$ \hspace{0.5cm} {\textcolor{carrotorange}{Por teorema 16 ($T_{16}$)}}     \vspace{0.5cm} 

Es decir  $x$ $+$ $1$ $\in$ [1, $\infty$]  \vspace{0.5cm}

Los reactivos a) $\wedge$ b) se satisfacen para $x$ $\in$ [1, $\infty$] \hspace{1cm} \qedsymbol

\newpage


\section{\textsf{Mostrar que si $A$ $\subseteq$ $\mathbb{R}$ es un conjunto inductivo entonces \mbox{$\mathbb{N}$ $\subseteq$ $A$}}} \vspace{0.5cm}

Definición de conjunto inductivo: $A$ $\subseteq$ $\mathbb{R}$ es inductivo si: \vspace{0.5cm}

a) $1$ $\in$ $A$ 

b) Si $x$ $\in$ $A$ \hspace{0.2cm} $\Longrightarrow$ \hspace{0.2cm} $x$ $+$ $1$ $\in$ $A$.

Definimos $\mathbb{N}$ $=$ $\left\{ x \in  \mathbb{R} \mid  x \hspace{0.2cm} es \hspace{0.1cm} elemento \hspace{0.1cm} de \hspace{0.1cm} cualquier \hspace{0.1cm} conjunto \hspace{0.1cm} inductivo \right\}$

Teorema: $\mathbb{N}$ es un conjunto inductivo. (Probado en clases.)

{\red{Demostración:}} \begin{MyColorPar}{verde_manzana}
Debemos probar que $1$ $\in$ $A$, pero por definición de $\mathbb{N}$, $1$ será elemento de $A$, si $1$ es elemento de cualquier conjunto inductivo, pero eso es garantizado por la definición de {\bfseries{conjunto inductivo}}. Probaremos ahora \mbox{b) si $x$ $\in$ $\mathbb{R}$} \hspace{0.2cm} $\Longrightarrow$ \hspace{0.2cm} $x$ $+$ $1$ $\in$ $\mathbb{R}$, para cualquier conjunto inductivo $\mathbb{R}$. Así que $x$ $+$ $1$ $\in$ $\mathbb{R}$ por definición de número real... Por lo tanto $\mathbb{N}$ $\subseteq$ $A$ \vspace{0.1cm} 

\hspace{12cm}\qedsymbol
\end{MyColorPar} \vspace{1cm}

%%%%%%%%%%%%%%%%%%%%%%%%%%%%%%%%%%%%%%%%%%%%%%%%%%%%%%%%%%%%%%%%%%%%%%%%%%%%%%%%%%%%%%%%%%%%%%%%%%%%%%%%%%%%%%%%%%%%%%%%%%%%%%%%%%%%%%%%%%%%%%%%%%%%%%%%%%%%%%%%%%%%%%%%%%%%%%%%%%%%%%%%%%%%%%%%%%%%%%%%%%%%%%%%%%%%%%%%%%%%%%%%%%%%%%%%%%%%%%%%%%%%%%%%%%%%%%%%%%%%%%%%%%%%%%%%

\section{\textsf{Mostrar por inducción que:}} \vspace{0.5cm}

\hspace{3cm} {\LARGE{$1$ $+$ $2$ $+$ ... $+$ $n$ $=$}} {\huge{{$\frac{n(n+1)}{2}$}}} \vspace{0.5cm}

{\red{Solución:}} \begin{MyColorPar}{verde_manzana}
Tenemos que garantizar que se cumplan las siguientes propiedades:

$i$) $n$ $=$ $1$

$ii$) $k$ $=$ $\left\{ n \in \mathbb{N} \mid 1 + 2 + ... + n = {\Large{\frac{n(n+1)}{2}}} \right\}$ y $n$ $\in$ $k$, es decir que si $n$ satisface $1$ $+$ $2$ $+$ ... $n$ $=$ {\Large{$\frac{n(n+1)}{2}$}} su consecutivo ($n$ $+$ $1$) también lo garantizará.  
\end{MyColorPar} \newpage

$i$) $n$ $=$ $1$ \vspace{0.5cm }

\hspace{5cm} $1$ $=$ {\LARGE{$\frac{1(1+1)}{2}$}} \vspace{0.5cm}

\hspace{5.4cm} $=$ {\LARGE{$\frac{1(1+1)}{2}$}} \vspace{0.5cm}

\hspace{5.4cm} $=$ {\LARGE{$\frac{1(2)}{2}$}} \vspace{0.5cm}

\hspace{5.4cm} $=$ {\LARGE{$\frac{(2)}{2}$}} \vspace{0.5cm}

\hspace{5cm} $1$ $=$ $1$ \vspace{0.5cm}

$ii$) Hipótesis de inducción (H.I.). Suponemos que se cumple para $n$ \vspace{0.5cm }

\hspace{5cm} \fbox{$1$ $+$ $2$ $+$ ... $+$ $n$} $=$ \fbox{{\LARGE{$\frac{n(n+1)}{2}$}}} \vspace{0.5cm }

Tesis inductiva: \vspace{0.5cm }

Ahora debemos probar que se cumple para cuando $n$ $=$ $n$ $+$ $1$ \vspace{0.5cm}

\hspace{3cm} $1$ $+$ $2$ $+$ ... $+$ $n$ $+$ $n$ $+$ $1$ $=$ {\LARGE{$\frac{(n \hspace{0.1cm}+\hspace{0.1cm} 1) (n\hspace{0.1cm} + \hspace{0.1cm}1\hspace{0.1cm} +\hspace{0.1cm} 1)}{2}$}} \vspace{0.5cm }

Tenemos entonces: \vspace{0.5cm}

\hspace{3cm} $1$ $+$ $2$ $+$ ... $+$ $n$ $+$ ($n$ $+$ $1$) $=$ {\LARGE{$\frac{n(n\hspace{0.1cm} + \hspace{0.1cm}1)}{2}$}} $+$ ($n$ $+$ $1$) \vspace{0.5cm }

\hspace{8.8cm} $=$ {\LARGE{$\frac{n^{2}\hspace{0.2cm}+\hspace{0.2cm}n}{2}$}} $+$ ($n$ $+$ $1$) \vspace{0.5cm }

\hspace{8.8cm} $=$ {\LARGE{{$\frac{n^{2} \hspace{0.1cm} + \hspace{0.1cm} n \hspace{0.1cm} + \hspace{0.1cm} 2(n\hspace{0.1cm}+\hspace{0.1cm}1)}{2}$}}} \vspace{0.5cm }

\hspace{8.8cm} $=$ {\LARGE{{$\frac{n^{2} \hspace{0.1cm} + \hspace{0.1cm} n \hspace{0.1cm} + \hspace{0.1cm} 2n\hspace{0.1cm}+\hspace{0.1cm}2}{2}$}}} \vspace{0.5cm }

\hspace{8.8cm} $=$ {\LARGE{{$\frac{n^{2} \hspace{0.1cm} + \hspace{0.1cm} 3n \hspace{0.1cm}+\hspace{0.1cm}2}{2}$}}} \vspace{0.5cm }

\hspace{8.8cm} $=$ {\LARGE{{$\frac{(n + 1)(n+2)}{2}$}}} \vspace{0.5cm }

\hspace{8.8cm} $=$ \fbox{{\LARGE{{$\frac{(n + 1)(n+1+1)}{2}$}}}} \vspace{0.2cm }

\hspace{9cm} \qedsymbol
\end{document}
