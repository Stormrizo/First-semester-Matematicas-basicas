\documentclass[12pt]{article} 
\usepackage[utf8]{inputenc}
\usepackage[spanish]{babel}
\usepackage{pdfpages}
\usepackage{parskip}
\usepackage{float}
\usepackage{enumitem}
\usepackage{multicol}
\newenvironment{Figura}
  {\par\medskip\noindent\minipage{\linewidth}}
  {\endminipage\par\medskip}
\usepackage{caption}
\usepackage{amsfonts}
\usepackage{amsmath, amsthm, amssymb}
\renewcommand{\qedsymbol}{$\blacksquare$}
\usepackage{graphicx}
\usepackage[left=2.54cm,right=2.54cm,top=2.54cm,bottom=2.54cm]{geometry}
\usepackage{pstricks}
\usepackage{xcolor}
\definecolor{verde_manzana}{rgb}{0.55, 0.71, 0.0}
\definecolor{Aguamarina}{rgb}{0.5, 1.0, 0.83}
\definecolor{mandarina_atomica}{rgb}{1.0, 0.6, 0.4}
\definecolor{blizzardblue}{rgb}{0.67, 0.9, 0.93}
\definecolor{bluegray}{rgb}{0.4, 0.6, 0.8}
\definecolor{coolgrey}{rgb}{0.55, 0.57, 0.67}
\definecolor{tealgreen}{rgb}{0.0, 0.51, 0.5}
\definecolor{ticklemepink}{rgb}{0.99, 0.54, 0.67}
\definecolor{thulianpink}{rgb}{0.87, 0.44, 0.63}
\definecolor{wildwatermelon}{rgb}{0.99, 0.42, 0.52}
\definecolor{wisteria}{rgb}{0.79, 0.63, 0.86}
\definecolor{yellow(munsell)}{rgb}{0.94, 0.8, 0.0}
\definecolor{trueblue}{rgb}{0.0, 0.45, 0.81}	\definecolor{tropicalrainforest}{rgb}{0.0, 0.46, 0.37}
\definecolor{tearose(rose)}{rgb}{0.96, 0.76, 0.76}
\definecolor{antiquefuchsia}{rgb}{0.57, 0.36, 0.51}	\definecolor{bittersweet}{rgb}{1.0, 0.44, 0.37}	\definecolor{carrotorange}{rgb}{0.93, 0.57, 0.13}
\definecolor{cinereous}{rgb}{0.6, 0.51, 0.48}
\definecolor{darkcoral}{rgb}{0.8, 0.36, 0.27}	\definecolor{orange(colorwheel)}{rgb}{1.0, 0.5, 0.0}
\definecolor{palatinateblue}{rgb}{0.15, 0.23, 0.89} \definecolor{pakistangreen}{rgb}{0.0, 0.4, 0.0} 	\definecolor{vividviolet}{rgb}{0.62, 0.0, 1.0} 
\definecolor{tigre}{rgb}{0.88, 0.55, 0.24} 	\definecolor{prussianblue}{rgb}{0.0, 0.19, 0.33} 	\definecolor{plum(traditional)}{rgb}{0.56, 0.27, 0.52} 	\definecolor{persianred}{rgb}{0.8, 0.2, 0.2} 	\definecolor{orange(webcolor)}{rgb}{1.0, 0.65, 0.0} 	\definecolor{onyx}{rgb}{0.06, 0.06, 0.06}
\definecolor{blue-violet}{rgb}{0.54, 0.17, 0.89}
\definecolor{byzantine}{rgb}{0.74, 0.2, 0.64}
\definecolor{byzantium}{rgb}{0.44, 0.16, 0.39}
\definecolor{darkmagenta}{rgb}{0.55, 0.0, 0.55} 	\definecolor{darkviolet}{rgb}{0.58, 0.0, 0.83} 	\definecolor{deepmagenta}{rgb}{0.8, 0.0, 0.8}

\begin{document}
\pagestyle{empty} 
\setlength{\parindent}{0pt}
\sffamily
\begin{center} \LARGE{\bf Benemérita Universidad Autónoma de Puebla} \\[0.5cm]
\begin{figure}[htb] \centering \includegraphics[scale=.2]{LogoBUAPpng.png} \end{figure}
\LARGE{Facultad de Ciencias Físico Matemáticas}\\[0.5cm]
\begin{figure}[htb] \centering \includegraphics[scale=.39]{LogoFCFMBUAP.png} \end{figure} 
\Large{Licenciatura en Física Teórica}\\[0.5cm]
\large{Primer semestre} \end{center}
\begin{center} { \Large \bfseries{Tarea 19}: (Teorema 21 {\blue{$ii)$}} y {\blue{$iii)$}} )} \\ \end{center}
\large{\bf Curso:} Matemáticas básicas \textbf{(N.R.C.:25598)}\\
\large{\bf Alumno:} Julio Alfredo Ballinas García $\left(202107583\right)$ \\
\large{\bf Docente:} Dra. María Araceli Juárez Ramírez\\
\large{\bf Grupo:} 102\\ \begin{center} 
\vfill
\textsc{\underline{Tarea retrasada:} venció {\red{{\underline{5 de octubre}}}} de 2021} \end{center}
\begin{center}
\textsc{Fecha de hoy: 14 de octubre de 2021}
\end{center}
\newpage

\section*{\sffamily{Mostrar {\red{Teorema 21}} {\blue{ii)}}:}}  \vspace{.5cm}

{\LARGE{{\blue{ii)}} \hspace{.1cm} $\mid x \mid$ $=$ $-x$}} \vspace{.5cm}


{\red{{\underline{Demostración por casos}}}}  \vspace{0.5cm}

{\red{{\underline{Solución:}}}} \vspace{0.5cm} 

{\textcolor{palatinateblue}{{\underline{Caso 1.}} }} si {\Large{$x$ $=$ $0$}}\vspace{0.5cm}

{\textcolor{palatinateblue}{Entonces:}} \vspace{0.5cm}

{\textcolor{carrotorange}{Por la definición de valor absoluto (1)}} \vspace{0.5cm}

\begin{center}
    
{\underline{DEFINICIÓN.}} Para todo $x$ en $\mathbb{R}$ se define $\mid x\mid \hspace{0.1cm} = x$ \hspace{0.45cm} si \hspace{0.2cm} $x\hspace{0.2cm} \geq \hspace{0.2cm}0$. (1) \vspace{0.1cm}

\hspace{8.63cm} $\mid x\mid = -\hspace{0.1cm}  x$ \hspace{0.2cm} si \hspace{0.1cm} $x\hspace{0.2cm} < \hspace{0.2cm}0$. (2) 
\vspace{0.3cm}

\end{center} \vspace{0.5cm}

{\textcolor{palatinateblue}{Así que, como } $x$ $=$ $0$} \vspace{0.5cm}

\hspace{4cm} $x$ $=$ $0$ $\Longrightarrow$ $0$ $\geq$ $0$ \vspace{0.5cm}

\hspace{5.3cm} $\Longleftrightarrow$ $0$ $<$ $0$ $\vee$ $0$ $=$ $0$ \hspace{0.1cm} {\textcolor{carrotorange}{Por teorema 14 (T$_{14})$}}\vspace{0.5cm} 

\hspace{5.35cm} $\Longleftrightarrow$ $\mid 0 \mid$ $=$ $0$ \hspace{1.43cm} {\textcolor{carrotorange}{Por definición de v.a. (1)}}\vspace{0.5cm} 

{\textcolor{palatinateblue}{Entonces el simétrico de} $x$} {\textcolor{palatinateblue}{es:}} \vspace{0.5cm}

$-x$ $=$ {\textcolor{vividviolet}{{\fbox{{\black{$-0$ $=$ $0$}}}}}} $=$ $x$ $\Longrightarrow$ {\textcolor{vividviolet}{{\fbox{{\black{$-x$ $=$ $x$}}}}}} \vspace{0.5cm}

\hspace{4.5cm} $\mid -x \mid$ $=$ $-x$ \hspace{2.5cm} {\textcolor{carrotorange}{Por definición de v.a (1)}} \vspace{0.5cm}

{\textcolor{palatinateblue}{En conclusión:}} \vspace{0.5cm}

\hspace{4.8cm} $\mid x \mid$ $=$ $-x$ \hspace{0.3cm} {\textcolor{carrotorange}{\qedsymbol}} \hspace{0.1cm} {\textcolor{palatinateblue}{\underline{Caso 1.}}} \vspace{1cm}
%%%%%%%%%%%%%%%%%%%%%%%%%%%%%%%%%%%%%%%%%%%%%%%%%%%%%%%%%%%
%%%%%%%%%%%%%%%%%%%%%%%%%%%%%%%%%%%%%%%%%%%%%%%%%%%%%%%%%%%%
%%%%%%%%%%%%%%%%%%%%%%%%%%%%%%%%%%%%%%%%%%%%%%%%%%%%%%%%%%
%%%%%%%%%%%%%%%%%%%%%%%%%%%%%%%%%%%%%%%%%%%%%%%%%%%%%%%%%%%%

{\textcolor{palatinateblue}{{\underline{Caso 2.}} }} si {\Large{$x$ $>$ $0$}}\vspace{0.5cm}

{\textcolor{palatinateblue}{Entonces:}} \vspace{0.5cm}

{\textcolor{carrotorange}{Por la definición de valor absoluto (1)}} \vspace{0.5cm}

{\textcolor{palatinateblue}{Se tiene que:}} \vspace{0.5cm}

\hspace{4.8cm} {\textcolor{vividviolet}{{\fbox{{\black{$\mid x \mid$ $=$ $x$}}}}}} \vspace{0.5cm}  

{\textcolor{palatinateblue}{Entonces:}} \vspace{0.5cm}

\hspace{5cm} $-x$ $<$ $0$ \hspace{1.9cm} {\textcolor{carrotorange}{Por teorema 19 $ii$) 4 (T$_{19}$)}} \vspace{0.5cm}

{\textcolor{palatinateblue}{Luego:}} \vspace{0.5cm}

\hspace{4.4cm} $\mid -x \mid$ $=$ $-(-x)$ \hspace{0.8cm} {\textcolor{carrotorange}{Por definición de v.a (2)}} \vspace{0.5cm}

\hspace{5.8cm} $=$ $x$ \hspace{2cm} {\textcolor{carrotorange}{Por axioma 4 (R$_{4}$)}} \vspace{0.5cm}

\hspace{2.8cm} $\Longrightarrow$ \hspace{0.5cm} {\textcolor{vividviolet}{{\fbox{{\black{$\mid -x \mid$ $=$ $x$}}}}}} \vspace{0.5cm}

{\textcolor{palatinateblue}{En conclusión:}} \vspace{0.5cm}

\hspace{2.8cm} $\Longrightarrow$ \hspace{0.9cm} {\textcolor{vividviolet}{{\fbox{{\black{$\mid x \mid$ $=$ $ -x $}}}}}}  \hspace{0.5cm} {\textcolor{carrotorange}{\qedsymbol}} \hspace{0.1cm} {\textcolor{palatinateblue}{\underline{Caso 2.}}} \vspace{1cm}

\newpage

%%%%%%%%%%%%%%%%%%%%%%%%%%%%%%%%%%%%%%%%%%%%%%%%%%%%%%%%%%%
%%%%%%%%%%%%%%%%%%%%%%%%%%%%%%%%%%%%%%%%%%%%%%%%%%%%%%%%%%
%%%%%%%%%%%%%%%%%%%%%%%%%%%%%%%%%%%%%%%%%%%%%%%%%%%%%%%%
%%%%%%%%%%%%%%%%%%%%%%%%%%%%%%%%%%%%%%%%%%%%%%%%%%%%%%%%%%%


{\textcolor{palatinateblue}{{\underline{Caso 3.}} }} si {\Large{$x$ $<$ $0$}}\vspace{0.5cm}

{\textcolor{palatinateblue}{Entonces:}} \vspace{0.5cm}

{\textcolor{carrotorange}{Por la definición de valor absoluto (2)}} \vspace{0.5cm}

{\textcolor{palatinateblue}{Se tiene que:}} \vspace{0.5cm}

\hspace{4.8cm} {\textcolor{vividviolet}{{\fbox{{\black{$\mid x \mid$ $=$ $-x$}}}}}} \vspace{0.5cm}  

{\textcolor{palatinateblue}{Entonces:}} \vspace{0.5cm}

\hspace{5cm} $-x$ $>$ $0$ \hspace{1.9cm} {\textcolor{carrotorange}{Por teorema 19 $ii$) 4 (T$_{19}$)}} \vspace{0.5cm}

{\textcolor{palatinateblue}{Luego:}} \vspace{0.5cm}

\hspace{4.4cm} $\mid -x \mid$ $=$ $-x$ \hspace{0.8cm} {\textcolor{carrotorange}{Por definición de v.a (1)}} \vspace{0.5cm}

{\textcolor{palatinateblue}{En conclusión:}} \vspace{0.5cm}

\hspace{2.8cm} $\Longrightarrow$ \hspace{0.9cm} {\textcolor{vividviolet}{{\fbox{{\black{$\mid x \mid$ $=$ $ -x $}}}}}}  \hspace{0.5cm} {\textcolor{carrotorange}{\qedsymbol}} \hspace{0.1cm} {\textcolor{palatinateblue}{\underline{Caso 3.}}} \vspace{1cm}

%%%%%%%%%%%%%%%%%%%%%%%%%%%%%%%%%%%%%%%%%%%%%%%%%%%%%%%%%%%%
%%%%%%%%%%%%%%%%%%%%%%%%%%%%%%%%%%%%%%%%%%%%%%%%%%%%%%%%%%%
%%%%%%%%%%%%%%%%%%%%%%%%%%%%%%%%%%%%%%%%%%%%%%%%%%%%%%%%%%
%%%%%%%%%%%%%%%%%%%%%%%%%%%%%%%%%%%%%%%%%%%%%%%%%%%%%%%%%%%%

\section*{\sffamily{Mostrar {\red{Teorema 21}} {\blue{iii)}}:}}  \vspace{.5cm}

{\red{{\underline{Demostración por casos}}}} \vspace{0.5cm}

{\red{{\underline{Solución:}}}} \vspace{0.5cm}

{\textcolor{palatinateblue}{sea} $\mid x \mid $ $=$ $0$ } {\textcolor{verde_manzana}{{\underline{es verdadera}}}} \vspace{0.5cm}

\newpage

{\textcolor{yellow(munsell)}{1.-}} {\textcolor{palatinateblue}{Si}} $x$ $\geq$ $0$ \vspace{0.5cm}

\hspace{1.4cm} $\Longrightarrow$ $\mid x \mid$ $=$ $x$ \hspace{0.3cm} {\textcolor{carrotorange}{Por definición de v.a (1)}} \vspace{0.5cm}

\hspace{1.4cm} $\Longrightarrow$ $0$ $=$  $\mid x \mid$ $=$ $x$ \vspace{0.5cm}

\hspace{1.4cm} $\Longrightarrow$ {\textcolor{vividviolet}{{\fbox{{\black{$x$ $=$  $0$}}}}}} \vspace{0.5cm}

{\textcolor{yellow(munsell)}{2.-}} {\textcolor{palatinateblue}{ Si}} $x$ $<$ $0$ \vspace{0.5cm}

\hspace{1.4cm} $\Longrightarrow$ $\mid x \mid$ $=$ $-x$ \vspace{0.5cm} \hspace{0.3cm}  {\textcolor{carrotorange}{Por definición de v.a (2)}}

\hspace{1.4cm} $\Longrightarrow$ $0$ $=$  $\mid x \mid$ $=$ $-x$ \vspace{0.5cm}

\hspace{1.4cm} $\Longrightarrow$ $-x$ $=$  $0$ \vspace{0.5cm}

\hspace{1.4cm} $\Longrightarrow$ $-x$ $=$  $x$ \vspace{0.5cm}

\hspace{1.4cm} $\Longrightarrow$ {\textcolor{vividviolet}{{\fbox{{\black{$x$ $=$  $0$}}}}}} \hspace{0.4cm} {\textcolor{carrotorange}{\qedsymbol}} \vspace{0.5cm}

\end{document}

