\documentclass[12pt]{article} 
\usepackage[utf8]{inputenc}
\usepackage[spanish]{babel}
\usepackage{amsfonts}
\usepackage{amsmath, amsthm, amssymb}
\renewcommand{\qedsymbol}{$\blacksquare$}
\usepackage{graphicx}
\usepackage[left=2.54cm,right=2.54cm,top=2.54cm,bottom=2.54cm]{geometry}
\usepackage{pstricks}
\begin{document}

\thispagestyle{empty} 
\begin{center} \LARGE{\bf Benemérita Universidad Autónoma de Puebla} \\[0.5cm]
\begin{figure}[htb] \centering \includegraphics[scale=.2]{LogoBUAPpng.png} \end{figure}
\LARGE{Facultad de Ciencias Físico Matemáticas}\\[0.5cm]
\begin{figure}[htb] \centering \includegraphics[scale=.39]{LogoFCFMBUAP.png} \end{figure} 
\Large{Licenciatura en Física Teórica}\\[0.5cm]
\large{Primer semestre} \end{center}
\begin{center} { \Large \bfseries{Tarea 14}} \\ \end{center}
\large{\bf Curso:} Matemáticas básicas \textbf{(N.R.C.:25598)}\\
\large{\bf Alumno:} Julio Alfredo Ballinas García $\left(202107583\right)$ \\
\large{\bf Docente:} Dra. María Araceli Juárez Ramírez\\
\large{\bf Grupo:} 102\\ \begin{center} 
\vfill
\textsc{21 de septiembre de 2021} \end{center}  
\newpage
\sffamily
\section*{Lista de axiomas.}

\begin{figure}[htb] \centering \includegraphics[scale=.9]{Imagen1.png} 
\caption{Axiomas $R_1$ a $R_1_6$}
\end{figure} 

\newpage

\section*{{\blue{Teorema 3: Unicidad del simétrico aditivo.}}}\\

{\textit{Para cada $x$ de $\mathbb{R}$, su simétrico $x^{\prime}$ es único (se denota $x^{\prime}=-x)$.}}\\

{\red{\underline{Solución:}}} Sean $x^{\prime}$ y $x^{{\prime}{\prime}}$ $\in$  $\mathbb{R}$ $\Rightarrow$ $x^{\prime}= x^{{\prime}{\prime}}$

\begin{center}
    {{\textbf{Hipótesis 1:}}} $x^{\prime} + x = 0$
\end{center}

\begin{center}
{{\textbf{Hipótesis 2:}}} $x^{{\prime}{\prime}} + x = 0$
\end{center}

\begin{center}
    {{\textbf{Tesis:}}} $ x^{\prime} = x^{{\prime}{\prime}}$
\end{center}



{\red{\underline{Demostración:}}}
\begin{align*}
\blue
  x^{\prime} =  x^{\prime} + e & \qquad \textup{Por axioma 3 (R$_3$)}\\
  \blue
  x^{\prime} =  x^{\prime} + 0 & \qquad \textup{Sabemos que e = 0}\\
  \blue
  x^{\prime} =  x^{\prime} + (x^{{\prime}{\prime}}+x) & \qquad \textup{Por \textbf{Hipótesis 2}}\\
  \blue
 x^{\prime} =  (x^{{\prime}{\prime}}+x)+x^{\prime} & \qquad \textup{Por axioma 1 (R$_1$)}\\
 \blue
  x^{\prime} = x^{{\prime}{\prime}}+(x+x^{\prime}) & \qquad \textup{Por axioma 2 (R$_2$)}\\
  \blue
  x^{\prime} =  x^{{\prime}{\prime}} + e & \qquad \textup{Por axioma 3 (R$_3$)}\\
  \blue
  x^{\prime} =  x^{{\prime}{\prime}} + 0 & \qquad \\
  \blue
  x^{\prime} =  x^{{\prime}{\prime}} & \qquad \textup{\qedsymbol}\\
\end{align*}

Concluimos que existe un sólo elemento simétrico con respecto a la operación suma debido a que $x^{\prime}$ $=$ $x^{{\prime}{\prime}}$ y lo denotamos como $-x$. 
\newpage
\section*{{\blue{Teorema 7: Ley de cancelación para el producto.}}}\\
{\textit{Se tiene siempre en $\mathbb{R}:$ $ x\neq 0 $ $ xy = xz$ $\Rightarrow$ $y=z$}}.\\

{\red{\underline{Demostración directa:}}} \\

{{\green{Suponemos $xy = xz$}}}
\begin{align*}
\blue
  xy\cdot x^{\prime} =  xz\cdot x^{\prime} & \qquad \textup{Por axioma 8 (R$_8$)}\\
  \blue
  x^{\prime}\cdot xy = x^{\prime}\cdot xz & \qquad \textup{Por axioma 5 (R$_5$)}\\
  \blue
  (x^{\prime}x)\cdot y = (x^{\prime}x)\cdot z & \qquad \textup{Por axioma 6 (R$_6$)}\\
  \blue
  (\'e)\cdot y = (\'e)\cdot z & \qquad \textup{Por axioma 8 (R$_8$)}\\
  \blue
  y = z & \qquad \textup{Por axioma 7 (R$_7$)}\\
  & \qquad \textup{\qedsymbol}\\
\end{align*}

Podemos concluir que si multiplicamos un número por su inverso, este será igual a multiplicar el mismo número y su inverso, aunque suene absurdo, por tanto $y$ y $z$ son iguales.\\\\


\section*{{\blue{Teorema 8: Unicidad del simétrico multiplicativo.}}}\\
{\textit{Para cada $x\neq 0$ en $\mathbb{R}$,  su simétrico $x^{{\prime}{\prime}}$ es único.}}\\

{\red{\underline{Demostración:}}} \\

{\red{\underline{Solución:}}} {{\green{Suponemos que existen dos elementos simétricos con respecto a la operación prodcuto; sean $a^{-1}$ y $b^{-1}$.}}}

\begin{center}
    {{\textbf{Hipótesis 1:}}} $x \cdot a^{-1} = \'e$
\end{center}

\begin{center}
{{\textbf{Hipótesis 2:}}} $x$ $\cdot$ $b^{-1}$ $=$ $\'e$
\end{center}

\begin{center}
    {{\textbf{Tesis:}}} $ a^{-1} = b^{-1}$ $ = x^{{\prime}{\prime}}$
\end{center}

\begin{align*}
\blue
  x \cdot a^{-1} =  \'e & \qquad \\
  \blue
  x \cdot a^{-1} =  x \cdot b^{-1} & \qquad \textup{Por \textbf{Hipótesis 2}}\\
  \blue
 x \cdot a^{-1} \cdot x^{\prime} =  x \cdot b^{-1} \cdot x^{\prime} & \qquad \textup{Por axioma 8 (R$_8$)}\\
 \blue
  x^{\prime} \cdot x \cdot a^{-1} = x^{\prime} \cdot  x \cdot b^{-1} & \qquad \textup{Por axioma 5 (R$_5$)}\\
  \blue
    (x^{\prime} \cdot x) \cdot a^{-1} = (x^{\prime} \cdot  x) \cdot b^{-1} & \qquad \textup{Por axioma 6 (R$_6$)}\\
  \blue
   (\'e ) \cdot a^{-1} = (\'e ) \cdot b^{-1} & \qquad \textup{Por axioma 8 (R$_8$)}\\ 
  \blue
   a^{-1} = b^{-1} & \qquad \textup{Por axioma 7 (R$_7$)}\\
    & \qquad \textup{\qedsymbol}
\end{align*}

Podemos concluir que existe un sólo elemento simétrico con respecto al producto debido a que $a^{-1} = b^{-1}$ y lo tenotamos como $x^{{\prime}{\prime}}$ $ = $ $x^{-1}$ $ = $ $\frac{1}{x}$.

\section*{{\blue{Teorema 9.}}}\\
{\textit{Para cada $x\neq 0$ en $\mathbb{R}$, se tiene siempre $(x^{-1})^{-1} = x$. (Notación $\frac{y}{x}$ designa $yx^{-1}$.}}\\

{\red{\underline{Demostración:}}} \\

Por axioma 8 $(R_8)$: 
\begin{center}
\blue
    $ x^{-1} \cdot x = \'e $  $ \vee $ $ x\cdot x^{-1}  = \'e $ $\longleftarrow$ \qquad \textup{{\black{Por axioma 5 (R$_5$)}}}
\end{center}

Si
\begin{center}
\blue
    $x^{-1}$ $\in$ $\mathbb{R}$ $\Longrightarrow$ $(x^{-1})^{-1}$ también $\in$ $\mathbb{R}$
\end{center}

De nuevo por axioma 8 $(R_8)$: 
\begin{center}
\blue
     $ x^{-1} \cdot (x^{-1})^{-1} = \'e $  $ \vee $ $ (x^{-1})^{-1} \cdot x^{-1}  = \'e $ $\longleftarrow$ \qquad \textup{\black{Por axioma 5 (R$_5$)}}
\end{center}
\newpage
Entonces por transitividad de la igualdad tenemos:
\begin{center}
\blue
 $(x^{-1})^{-1} \cdot x^{-1}$ $=$ $\'e$ $=$ $x \cdot x^{-1}$
 \end{center}
 \begin{center}
\blue
 $\Rightarrow$ $x \cdot  (x^{-1})^{-1}\cdot x^{-1} = x \cdot x^{-1} \cdot x$ $\leftarrow$ \qquad \textup{\black{Por axioma 8 (R$_8$)}}
 \end{center}
 Como anteriormente se menciona por transitividad de la igualdad tenemos:
\begin{center}
\blue
    $x \cdot  (x^{-1})^{-1} \cdot x^{-1} = x \cdot x^{-1} \cdot x$
\end{center}

\begin{align*}
\blue
 x^{-1} \cdot x \cdot  (x^{-1})^{-1} = x^{-1} \cdot x \cdot x & \qquad \textup{Por axioma 5 (R$_5$)}\\
  \blue
   (x^{-1} \cdot x) \cdot  (x^{-1})^{-1} = (x^{-1} \cdot x) \cdot x & \qquad \textup{Por axioma 6 (R$_6$)}\\
  \blue
  (\'e) \cdot  (x^{-1})^{-1} = (\'e) \cdot x & \qquad \textup{Por axioma 8 (R$_8$)}\\
 \blue
 (x^{-1})^{-1} =  x & \qquad \textup{Por axioma 7 (R$_7$)}\\
    & \qquad \textup{\qedsymbol}
\end{align*}
 
Hemos demostrado que $(x^{-1})^{-1}$ $=$ $x$.  \\\\

\section*{{\blue{Teorema 5:}} Caso 2.}

{\textit{Se tiene siempre en $\mathbb{R}$ la equivalencia:}} 

\begin{center}
    $ xy = 0 $ $ \Leftrightarrow $ $ x = 0 $ $\vee$ $ y = 0 $
\end{center}

Caso 2.- $y = 0$, mostraremos que $x \cdot 0 = 0$.\\

{\red{\underline{Demostración:}}}\\

Por axioma 3 (R$_3$):

\begin{center}
\blue
   si  $ 0 + x = 0 $ $ \Rightarrow $ $ 0 + 0 = 0$
\end{center}
\newpage

Por lo tanto $x\cdot 0$ puede expresarse así:
\begin{center}
\blue
    $x\cdot (0+0)$
\end{center}

Por axioma 9 ($R_9$):

\begin{center}
\blue
    $x\cdot 0 + x \cdot 0$
\end{center}

Ahora por transitividad de la igualdad:

\begin{equation*}
    \begin{split}
    \blue
        x \cdot 0 & = \blue x \cdot (0 + 0) \\\\ 
        & = \blue x \cdot 0 + x \cdot 0\\\\ 
         & =\blue  0 + x \cdot 0\\\\ 
       \qquad \textup{Por axioma 1 (R$_1$)}  & =\blue   x \cdot 0 + 0 
    \end{split}
\end{equation*}
De nuevo, por transitividad de la igualdad:

\begin{equation*}
    \begin{split}
    \blue
        x \cdot 0 +  x \cdot 0 & = \blue x \cdot 0 + 0 \\\\ 
       \qquad \textup{Por teorema 2.} \qquad \blue x \cdot 0 & =\blue 0 \qquad \textup{{\black{teorema 2: $x+y=x+z\Rightarrow y=z$}}}\\\\
       & \qquad \textup{\qedsymbol}
    \end{split}
\end{equation*}
Hemos demostrado el segundo caso del teorema 5. 

\newpage

\section*{\blue{Tabla de verdad para tautología:}}

\begin{center}
\section*{$P\Rightarrow Q \vee R \Leftrightarrow P \wedge \lnot  Q\Rightarrow R$ } 
\end{center}\\

Se concluye que en efecto se trata de una tautología.
\begin{table}
\centering
\begin{tabular}{|l|l|l|l|l|l|l|l|l|} 
\hline
P & Q & R & $\lnot  Q$ & $Q \vee R$ & $P \wedge \lnot  Q$ & ($P\Rightarrow Q \vee R$) & ($P \wedge \lnot  Q\Rightarrow R$) & ($P\Rightarrow Q \vee R) \Leftrightarrow (P \wedge \lnot  Q\Rightarrow R$)  \\ 
\hline
V & V & V &  F  &  V     &  F      &        V          &        V           &             V                                          \\ 
\hline
V & V & F &  F  &  V     &  F      &        V          &        V           &           V                                            \\ 
\hline
V & F & V &  V  &  V     &   V     &       V           &         V          &            V                                           \\ 
\hline
V & F & F &   V & F      &  V      &        F          &        F           &           V                                            \\ 
\hline
F & V & V & F   & V      &  F      &        V          &        V           &           V                                            \\ 
\hline
F & V & F & F   & V      & F       &        V          &        V           &           V                                            \\ 
\hline
F & F & V & V   & V      & F       &        V          &    V               &           V                                            \\ 
\hline
F & F & F &  V  &  F     & F       &        V          &    V               &           V                                            \\
\hline
\end{tabular}
\end{table}


\end{document}


