\documentclass[12pt]{article} 
\usepackage[utf8]{inputenc}
\usepackage[spanish]{babel}
\usepackage{amsfonts}
\usepackage{amsmath, amsthm, amssymb}
\renewcommand{\qedsymbol}{$\blacksquare$}
\usepackage{graphicx}
\usepackage[left=2.54cm,right=2.54cm,top=2.54cm,bottom=2.54cm]{geometry}
\usepackage{pstricks}
\begin{document}
\sffamily
\thispagestyle{empty} 
\begin{center} \LARGE{\bf Benemérita Universidad Autónoma de Puebla} \\[0.5cm]
\begin{figure}[htb] \centering \includegraphics[scale=.2]{LogoBUAPpng.png} \end{figure}
\LARGE{Facultad de Ciencias Físico Matemáticas}\\[0.5cm]
\begin{figure}[htb] \centering \includegraphics[scale=.39]{LogoFCFMBUAP.png} \end{figure} 
\Large{Licenciatura en Física Teórica}\\[0.5cm]
\large{Primer semestre} \end{center}
\begin{center} { \Large \bfseries{Tarea 17}} \\ \end{center}
\large{\bf Curso:} Matemáticas básicas \textbf{(N.R.C.:25598)}\\
\large{\bf Alumno:} Julio Alfredo Ballinas García $\left(202107583\right)$ \\
\large{\bf Docente:} Dra. María Araceli Juárez Ramírez\\
\large{\bf Grupo:} 102\\ \begin{center} 
\vfill
\textsc{30 de septiembre de 2021} \end{center}  
\newpage
\section*{Lista de axiomas.}

\begin{figure}[htb] \centering \includegraphics[scale=.95]{Lista de axiomas 1.png} 
\caption{Axiomas $R_1$ a $R_1_4$}
\end{figure} 
\newpage
\begin{figure}[htb] \centering \includegraphics[scale=.95]{Lista de axiomas 2.png} 
\caption{Axiomas $R_5$ a $R_1_5$}
\end{figure} 
\newpage
\begin{figure}[htb] \centering \includegraphics[scale=.95]{Lista de axiomas 3.png} 
\caption{Axioma $R_1_6$}
\end{figure} 

\vspace{1cm}

\section*{\blue{Teorema 15:} \red{Caso 3}.} 

$x,y$ siendo elementos cualesquiera de $\mathbb{R}$, se cumple siempre una y sólo \par una de las tres relaciones: $x\hspace{0.3cm}<\hspace{0.3cm}y$; \hspace{0.3cm}$y\hspace{0.3cm}<\hspace{0.3cm}x$;\hspace{0.3cm} $x=y$. \vspace{0.2cm} \vspace{0.2cm}

Ley de tricotomía. \vspace{0.3cm}

III) $\neg \hspace{0.1cm}(y\hspace{0.1cm}<\hspace{0.1cm}x) \hspace{0.2cm}\wedge\hspace{0.2cm}\neg \hspace{0.1cm}(x\hspace{0.1cm}=\hspace{0.1cm}y)\hspace{0.1cm}\Longrightarrow \hspace{0.1cm} x\hspace{0.1cm}<y$ \vspace{0.5cm}

{\red{\underline{Solución:}}} Para mostrar III., Lo haremos por {\blue{\underline{contrarrecíproca}}}. Recordemos que \par dada la proposición \hspace{0.2cm} $P\hspace{0.1cm}\Rightarrow \hspace{0.1cm}Q$ esto es equivalente a $\neg\hspace{0.1cm} Q\hspace{0.1cm}\Rightarrow\hspace{0.1cm}\neg\hspace{0.1cm}P$ \vspace{0.5cm}

Así la {\blue{\underline{contrarrecíproca}}} de III) \hspace{0.1cm} $\neg\hspace{0.1cm}(x\hspace{0.1cm}<\hspace{0.1cm}y)\hspace{0.1cm}\Rightarrow\hspace{0.1cm}(y\hspace{0.1cm}<\hspace{0.1cm}x)\hspace{0.1cm}\vee\hspace{0.1cm}(x\hspace{0.1cm}=\hspace{0.1cm}y)$ \vspace{0.5cm}

{\red{\underline{Demostración:}}}\vspace{0.5cm}

\begin{center}
    $\neg\hspace{0.1cm}(x\hspace{0.1cm}<\hspace{0.1cm}y)  \Longleftrightarrow \blue  \neg\hspace{0.1cm}(x\hspace{0.1cm}\leq\hspace{0.1cm}y\wedge\hspace{0.1cm}x\hspace{0.1cm}\neq\hspace{0.1cm}y)$\hspace{1cm}\vspace{1cm} 
    
   \hspace{10cm} \textup{\black{Por definición de ($<)$}}
\end{center}\vspace{0.5cm}

\begin{center}
    \hspace{5cm}\Longleftrightarrow \hspace{0.2cm} $\blue  \neg\hspace{0.1cm}(x\hspace{0.1cm}\leq\hspace{0.1cm}y)\vee\neg \hspace{0.1cm}(x\hspace{0.1cm}\neq\hspace{0.1cm}y) \hspace{1cm}$ \hspace{1cm} \vspace{1cm}
    
   \hspace{10cm} \textup{\black{Por ley de De Morgan}}
\end{center}\vspace{1cm}

\begin{center}
   \hspace{3.3cm} \Longleftrightarrow \hspace{0.2cm}$\blue \neg\hspace{0.1cm}(x\hspace{0.1cm}\leq\hspace{0.1cm}y)\vee \hspace{0.1cm}(x\hspace{0.1cm}=\hspace{0.1cm}y) \hspace{1cm}$ \hspace{1cm}
\end{center}\vspace{0.5cm}

\begin{center}
    \hspace{2.6cm}\Longleftrightarrow \hspace{0.2cm}$\blue \hspace{0.1cm}(x\hspace{0.1cm}$>$\hspace{0.1cm}y)\hspace{0.1cm}\vee \hspace{0.1cm}(x\hspace{0.1cm}=\hspace{0.1cm}y) \hspace{1cm}$ 
\end{center}\vspace{1cm}


\begin{center}
     \hspace{2cm}\Longleftrightarrow \hspace{0.2cm}$\blue x\hspace{0.1cm}=\hspace{0.1cm}y \hspace{1cm}$ \hspace{1cm}
    \hspace{1cm} \vspace{1cm}
    
    \hspace{10cm} \textup{\black{\qedsymbol}}\hspace{0.2cm}\textup{\black{Por disyunción}} 
\end{center}\vspace{0.5cm}

\section*{\blue{Teorema 3:} \red{Argumento completo de la unicidad por contradicción}.} 

Para cada $x$ de $\mathbb{R}$, su simétrico $x^{\prime}$ es único. (Se denota $x^{\prime}\hspace{0.1cm}=\hspace{0.1cm}-x)$. \vspace{0.2cm} \vspace{0.2cm}

{\red{\underline{Solución:}}}  {\blue{\underline{contradicción}}}.Suponemos que no existen 2 elementos simétricos de \par $x$ con respecto a la operación suma, tal que $x_1^{\prime}\hspace{0.1cm}\neq\hspace{0.1cm}x_2^{\prime}$\vspace{1cm}

{\red{\underline{Demostración:}}}\vspace{0.5cm}

{\red{\underline{Solución:}}} Sean $x^{\prime}$ y $x^{{\prime}{\prime}}$ $\in$  $\mathbb{R}$ $\Rightarrow$ $x_1^{\prime}\neq x_2^{\prime}$

\begin{center}
    {{\textbf{Hipótesis 1:}}} $x_1^{\prime} + x = 0$
\end{center}

\begin{center}
{{\textbf{Hipótesis 2:}}} $x_2^{\prime} + x = 0$
\end{center}

\begin{center}
    {{\textbf{Tesis:}}} $ x_1^{\prime} \neq x_2^{\prime}$
\end{center}



{\red{\underline{Demostración:}}}
\begin{align*}
\blue
  x_1^{\prime} =  x_1^{\prime} + e & \qquad \textup{Por axioma 3 (R$_3$)}\\
  \blue
  x_1^{\prime} =  x_1^{\prime} + 0 & \qquad \textup{Sabemos por Teorema 1. e = 0}\\
  \blue
  x_1^{\prime} =  x_1^{\prime} + (x_2^{\prime}+x) & \qquad \textup{Por \textbf{Hipótesis 2}}\\
  \blue
 x_1^{\prime} =  (x_2^{\prime}+x)+x_1^{\prime} & \qquad \textup{Por axioma 1 (R$_1$)}\\
 \blue
  x_1^{\prime} = x_2^{\prime}{\prime}+(x+x^{\prime}) & \qquad \textup{Por axioma 2 (R$_2$)}\\
  \blue
  x^{\prime} =  x^{{\prime}{\prime}} + e & \qquad \textup{Por axioma 3 (R$_3$)}\\
  \blue
  x^{\prime} =  x^{{\prime}{\prime}} + 0 & \qquad \textup{Por Teorema 1. e = 0} \\
  \blue
  x^{\prime} =  x^{{\prime}{\prime}} & \qquad \textup{\qedsymbol}\\
\end{align*}

Pero esto es una contradicción $x_1^{\prime}$ $=$ $x_2^{\prime}$ .Entonces concluimos que $x_1^{\prime}$ $\neq$ $x_2^{\prime}$ es falsa, por lo tanto es cierto que $x_1^{\prime}$ $=$ $x_2^{\prime}$.
\vspace{2cm}

\section*{\blue{Teorema 16:}\red{ Caso 2}} 

\hspace{2cm} $x\hspace{0.2cm}<\hspace{0.2cm}y\hspace{0.2cm} \wedge\hspace{0.2cm} y\hspace{0.2cm}\leq\hspace{0.2cm} z\hspace{0.2cm} \Longrightarrow\hspace{0.2cm} x\hspace{0.2cm}<\hspace{0.2cm}z$.
\vspace{0.3cm}

{\red{\underline{Demostración:}}} 

\vspace{1cm}

Supongamos verdadero:

\begin{center}
    $\hspace{2cm} x\hspace{0.2cm}<\hspace{0.2cm}y\hspace{0.2cm} \wedge\hspace{0.2cm} y\hspace{0.2cm}\leq\hspace{0.2cm} z\hspace{0.2cm}$
\end{center}

Tenemos que llegar a \hspace{0.2cm}$x\hspace{0.1cm}<\hspace{0.1cm}z:$

\begin{center}
    $\hspace{2cm} x\hspace{0.2cm}<\hspace{0.2cm}y\hspace{0.2cm} \wedge\hspace{0.2cm} y\hspace{0.2cm}\leq\hspace{0.2cm} z\hspace{0.2cm}\Longleftrightarrow\hspace{0.2cm} x\hspace{0.2cm}<\hspace{0.2cm}z$ 
\end{center}

Por la definición de relación entre elementos ($x\hspace{0.2cm}<\hspace{0.2cm}y$) tenemos:

\begin{center}
    $\hspace{2cm} x\hspace{0.2cm}<\hspace{0.2cm}y\hspace{0.2cm} \wedge\hspace{0.2cm} y\hspace{0.2cm}\leq\hspace{0.2cm} z\hspace{0.2cm}\Longleftrightarrow\hspace{0.2cm} (x\hspace{0.2cm}\leq\hspace{0.2cm}y\hspace{0.2cm} \wedge\hspace{0.2cm} x\hspace{0.2cm}\neq\hspace{0.2cm}y)\hspace{0.2cm}\wedge\hspace{0.2cm} y \hspace{0.2cm} \leq \hspace{0.2cm} z$ 
\end{center}
\newpage
\begin{center}
    $\hspace{2cm} x\hspace{0.2cm}<\hspace{0.2cm}y\hspace{0.2cm} \wedge\hspace{0.2cm} y\hspace{0.2cm}\leq\hspace{0.2cm} z\hspace{0.2cm}\Longleftrightarrow\hspace{0.2cm} (x\hspace{0.2cm}\leq\hspace{0.2cm}y\hspace{0.2cm} \wedge\hspace{0.2cm} x\hspace{0.2cm}\neq\hspace{0.2cm}y)\hspace{0.2cm}\wedge\hspace{0.2cm} y \hspace{0.2cm} \leq \hspace{0.2cm} z$ 
\end{center}\vspace{0.2cm}

Por axioma 6
\begin{center}
$\hspace{6.5cm}\Longleftrightarrow\hspace{0.2cm} (x\hspace{0.2cm}\leq\hspace{0.2cm}y\hspace{0.2cm} \wedge\hspace{0.2cm} y \hspace{0.2cm} \leq \hspace{0.2cm} z)\hspace{0.2cm}\wedge\hspace{0.2cm} x\hspace{0.2cm}\neq\hspace{0.2cm}y$ 
\end{center}\vspace{0.2cm}

Por axioma 10
\begin{center}
$\hspace{6.5cm}\Longleftrightarrow\hspace{0.2cm} x\hspace{0.2cm}\leq\hspace{0.2cm}z\hspace{0.2cm}\wedge\hspace{0.2cm} x \hspace{0.2cm} \neq \hspace{0.2cm} y$ 
\end{center}\vspace{0.2cm}

Para concluir basta probar $y\hspace{0.2cm} \neq z$. \vspace{0.4cm}

Por {\red{\underline{contradicción.}}} Supongamos $y\hspace{0.1cm}=\hspace{0.1cm} z$. Entonces tendríamos:

\begin{center}
    $x\hspace{0.1cm}<z\hspace{0.2cm} \wedge \hspace{0.2cm} y\hspace{0.1cm} \leq \hspace{0.1cm} z$
\end{center}\vspace{0.2cm}

Por teorema 14:
\begin{center}
    $x\hspace{0.1cm}<z\hspace{0.2cm} \wedge \hspace{0.2cm} (y\hspace{0.1cm} < \hspace{0.1cm} z\hspace{0.2cm} \vee \hspace{0.1cm}y\hspace{0.1cm} = \hspace{0.1cm} z)$
\end{center}
Llegamos a:
\begin{center}
    $x\hspace{0.1cm}<z$ \hspace{02cm}\qedsymbol
\end{center}
\end{document}
