\documentclass[12pt]{article} 
\usepackage[left=2.54cm,right=2.54cm,top=2.54cm,bottom=2.54cm]{geometry}
\usepackage[utf8]{inputenc}
\usepackage[spanish]{babel}
\usepackage{pdfpages}
\usepackage{csquotes}
\usepackage{afterpage}
\usepackage{parskip}
\usepackage{float}
\usepackage{enumitem}
\usepackage{multicol}
\newenvironment{Figura}
  {\par\medskip\noindent\minipage{\linewidth}}
  {\endminipage\par\medskip}
\usepackage{caption}
\usepackage{amsfonts}
\usepackage{amsmath, amsthm, amssymb}
\renewcommand{\qedsymbol}{$\blacksquare$}
\usepackage{graphicx}
\usepackage{pstricks} 

\usepackage{xcolor}

\definecolor{prussianblue}{RGB}{1, 45, 75} 
\definecolor{brightturquoise}{RGB}{1, 196, 254} 
\definecolor{Aguamarina}{rgb}{0.5, 1.0, 0.83}
\definecolor{mandarina_atomica}{rgb}{1.0, 0.6, 0.4}
\definecolor{blizzardblue}{rgb}{0.67, 0.9, 0.93}
\definecolor{Ebony Clay}{RGB}{35, 44, 67}
\definecolor{Tuscany}{RGB}{205, 111, 52}
\definecolor{prussianblue}{RGB}{1, 45, 75} 
\definecolor{brightturquoise}{RGB}{1, 196, 254} 
\definecolor{verde_manzana}{rgb}{0.55, 0.71, 0.0}
\definecolor{Aguamarina}{rgb}{0.5, 1.0, 0.83}
\definecolor{mandarina_atomica}{rgb}{1.0, 0.6, 0.4}
\definecolor{blizzardblue}{rgb}{0.67, 0.9, 0.93}
\definecolor{bluegray}{rgb}{0.4, 0.6, 0.8}
\definecolor{coolgrey}{rgb}{0.55, 0.57, 0.67}
\definecolor{tealgreen}{rgb}{0.0, 0.51, 0.5}
\definecolor{ticklemepink}{rgb}{0.99, 0.54, 0.67}
\definecolor{thulianpink}{rgb}{0.87, 0.44, 0.63}
\definecolor{wildwatermelon}{rgb}{0.99, 0.42, 0.52}
\definecolor{wisteria}{rgb}{0.79, 0.63, 0.86}
\definecolor{yellow(munsell)}{rgb}{0.94, 0.8, 0.0}
\definecolor{trueblue}{rgb}{0.0, 0.45, 0.81}	\definecolor{tropicalrainforest}{rgb}{0.0, 0.46, 0.37}
\definecolor{tearose(rose)}{rgb}{0.96, 0.76, 0.76}
\definecolor{antiquefuchsia}{rgb}{0.57, 0.36, 0.51}	\definecolor{bittersweet}{rgb}{1.0, 0.44, 0.37}	\definecolor{carrotorange}{rgb}{0.93, 0.57, 0.13}
\definecolor{cinereous}{rgb}{0.6, 0.51, 0.48}
\definecolor{darkcoral}{rgb}{0.8, 0.36, 0.27}	\definecolor{orange(colorwheel)}{rgb}{1.0, 0.5, 0.0}
\definecolor{palatinateblue}{rgb}{0.15, 0.23, 0.89} \definecolor{pakistangreen}{rgb}{0.0, 0.4, 0.0} 	\definecolor{vividviolet}{rgb}{0.62, 0.0, 1.0} 
\definecolor{tigre}{rgb}{0.88, 0.55, 0.24} 		\definecolor{plum(traditional)}{rgb}{0.56, 0.27, 0.52} 	\definecolor{persianred}{rgb}{0.8, 0.2, 0.2} 	\definecolor{orange(webcolor)}{rgb}{1.0, 0.65, 0.0} 	\definecolor{onyx}{rgb}{0.06, 0.06, 0.06}
\definecolor{blue-violet}{rgb}{0.54, 0.17, 0.89}
\definecolor{byzantine}{rgb}{0.74, 0.2, 0.64}
\definecolor{byzantium}{rgb}{0.44, 0.16, 0.39}
\definecolor{darkmagenta}{rgb}{0.55, 0.0, 0.55} 
\definecolor{Gallery}{RGB}{236, 236, 236} 
\definecolor{darkviolet}{rgb}{0.58, 0.0, 0.83} 	\definecolor{deepmagenta}{rgb}{0.8, 0.0, 0.8}
\definecolor{Mercury}{RGB}{228, 228, 228} 
\definecolor{Alto}{RGB}{220, 220, 220}
\definecolor{Woodsmoke}{RGB}{4, 4, 5} 
\definecolor{Iron}{RGB}{227, 227, 228} 
\definecolor{Bluechill}{RGB}{11, 150, 144}
\definecolor{Deep Sea Green}{RGB}{8, 83, 94}
\definecolor{Sun}{RGB}{251, 175, 17} 
\definecolor{Lochmara}{RGB}{9, 116, 189}  
\definecolor{Green vogue}{RGB}{4, 40, 85}  
\definecolor{Hippie Blue}{RGB}{92, 148, 179}  
\definecolor{Saratoga}{RGB}{85, 100, 19}  
\definecolor{Earls Green}{RGB}{177, 196, 56}  
\definecolor{Cavern Pink}{RGB}{231, 190, 194} 
\definecolor{Tamarillo}{RGB}{155, 23, 33} 
\definecolor{Cinnabar}{RGB}{225, 71, 53} 
\definecolor{Horizon}{RGB}{88, 132, 169} 
\definecolor{Tarawera}{RGB}{6, 48, 70}
\definecolor{Fiery Orange}{RGB}{180, 92, 22}
\definecolor{Lemon Ginger}{RGB}{170, 164, 40}
\definecolor{Burnt Sienna}{RGB}{236, 119, 88}
\definecolor{Milano Red}{RGB}{184, 12, 11}
\newenvironment{MyColorPar}[1]{%
    \leavevmode\color{#1}\ignorespaces%
}{%
}%


\begin{document}

\begingroup
\begin{titlepage}
	\AddToShipoutPicture*{\put(79,350){\includegraphics[scale=.3]{descarga.png}}}
	\noindent
	\vspace{1mm}
\end{titlepage}
\endgroup

\pagestyle{empty} 
\setlength{\parindent}{0pt}
\sffamily

%%%%%%%%%%%%%%%%%%%%%%%%%%%%%%%%%%%%%%%%%%%%%%%%%%%%%%%%%%%%%%%%%%%
%%%%%%%%%%%%%%%%%%%%%%%%%%%%%%%%%%%%%%%%%%%%%%%%%%%%%%%%%%%%%%%%%%%

\begin{center} 

    \LARGE{\bf{\textsf{Benemérita Universidad Autónoma de Puebla}}} \\[0.5cm]
    
\begin{figure}[htb] \centering

    \includegraphics[scale=.25]{LogoBUAPpng.png} 

\end{figure}

%%%%%%%%%%%%%%%%%%%%%%%%%%%%%%%%%%%%%%%%%%%%%%%%%%%%%%%%%%%%%%%%%%%
%%%%%%%%%%%%%%%%%%%%%%%%%%%%%%%%%%%%%%%%%%%%%%%%%%%%%%%%%%%%%%%%%%%

    \LARGE{Facultad de Ciencias Físico Matemáticas}\\[0.5cm]

\begin{figure}[htb] \centering

    \includegraphics[scale=.4]{LogoFCFMBUAP.png} 
    
\end{figure} 

%%%%%%%%%%%%%%%%%%%%%%%%%%%%%%%%%%%%%%%%%%%%%%%%%%%%%%%%%%%%%%%%%%%
%%%%%%%%%%%%%%%%%%%%%%%%%%%%%%%%%%%%%%%%%%%%%%%%%%%%%%%%%%%%%%%%%%%

    \Large{Licenciatura en Física Teórica}\\[0.5cm]
    \Large{Primer semestre} 

\end{center} \vspace{0.3cm}
%%%%%%%%%%%%%%%%%%%%%%%%%%%%%%%%%%%%%%%%%%%%%%%%%%%%%%%%%%%%%%%%%%%
%%%%%%%%%%%%%%%%%%%%%%%%%%%%%%%%%%%%%%%%%%%%%%%%%%%%%%%%%%%%%%%%%%%

\begin{center}

    {\Large{\bfseries{{\textcolor{carrotorange}{Tarea 30}}}}} \\ 
    
\end{center}

    \large{\bf{\textsf{Curso:}}} {\bfseries{{\textcolor{brightturquoise}{Matemáticas básicas \bfseries{(N.R.C.:25598)}}}}} \\
    \large{\bf{\textsf{Alumno:}}} {\bfseries{{\textcolor{prussianblue}{Julio Alfredo Ballinas García {\large{{$\mid$}}} 202107583}}}}  \\
    \large{\bf{\textsf{Docente:}}} {\bfseries{{\textcolor{wisteria}{Dra. María Araceli Juárez Ramírez}}}}\\
    \large{\bf{\textsf{Grupo:}}} {\bfseries{{\textcolor{verde_manzana}{102}}}}\\

\vfill
    
\begin{center} 

    {\small{\textsf{\underline{\bfseries Venció:} el 9 de noviembre de 2021 {{\textcolor{Cinnabar}{\bfseries 23:59 PM}}}} {\LARGE{ $\mid$ }}\textsf{{\underline{\bfseries Se entregó:}} el 10 de noviembre de 2021}}}
    
\end{center}

\newpage

%%%%%%%%%%%%%%%%%%%%%%%%%%%%%%%%%%%%%%%%%%%%%%%%%%%%%%%%%%%%%%%%%%%
%%%%%%%%%%%%%%%%%%%%%%%%%%%%%%%%%%%%%%%%%%%%%%%%%%%%%%%%%%%%%%%%%%%
\section*{\textsf{Mostrar si} $n \hspace{0.2cm} \bigg |\ \bigg  m$ \hspace{0.2cm} $\Longrightarrow$ \hspace{0.2cm } $n \hspace{0.2cm} \bigg |\ \bigg  m \cdot s$ \hspace{0.1cm } $\forall_{s}$ $\in$ $\mathbb{Z}$   }

\begin{MyColorPar}{Cinnabar} \bfseries
\underline {Solución:}
\end{MyColorPar} \vspace{0.5cm}

\begin{MyColorPar}{Tarawera}\bfseries
{\textcolor{Cinnabar}{\bfseries{Demostración directa}}}. Suponemos {\black{$n \hspace{0.2cm} \bigg |\ \bigg  m$}} {\textcolor{verde_manzana}{\bfseries{verdadera}}}. \vspace{0.5cm}

{\black{$\Longrightarrow$}} \hspace{0.2cm} {\black{$\exists$ $k, t$ $\in$ $\mathbb{Z}$ $\mid$ $m$ $=$ $kn$ \hspace{0.1cm} $\wedge$ \hspace{0.1cm} $s$ $=$ $tn$}} \vspace{0.5cm}

{\black{$\Longrightarrow$}} \hspace{0.2cm} {\black{$m$ $\cdot$ $s$   $=$ $kn$ $\cdot$ $tn$ }} \vspace{0.5cm}

{\black{$\Longrightarrow$}} \hspace{0.2cm} {\black{ ($k$ $\cdot$ $t$) $\cdot$ $n$ }} \vspace{0.5cm}

Esto demuestra que {\black{$m$ $\cdot$ $s$}} es múltiplo de {\black{$n$}} o {\black{$n \hspace{0.2cm} \bigg |\ \bigg  m \cdot s$}}  \vspace{0.5cm}

\hspace{1cm} {\black{ \qedsymbol }} \vspace{1cm}
\end{MyColorPar}

\section*{\textsf{Mostrar que si} $a_{0}$ es par $\Longrightarrow$ $2 \hspace{0.2cm} \bigg |\ \bigg m$ }

\begin{MyColorPar}{Tarawera}\bfseries
{\textcolor{Cinnabar}{\bfseries{Demostración directa}}}. Suponemos {\black{$a_{0}$}} es par  {\textcolor{verde_manzana}{\bfseries{($a_{0}$ $=$ $2k$)}}}. \vspace{0.5cm}

\hspace{2cm}{\black{\fbox{$a$ $=$ $a_{n}$ $\times$ $10^{n}$ $+$ ... $+$ $a_{1}$ $\times$ $10$ $+$ $a_{0}$}}} {\Large$*$} \vspace{0.5cm}

Pero en clase se mostró que {\black{\mbox{$a$ $=$ $a_{n}$ $\times$ $10^{n}$ $+$ ... $+$ $a_{1}$ $\times$ $10$ $+$ $a_{0}$ $=$ $2s$}}} \vspace{0.5cm}

{\black{$\Longrightarrow$}} $-$ $\big($ {\black{\mbox{$a$ $=$ $a_{n}$ $\times$ $10^{n}$ $+$ ... $+$ $a_{1}$ $\times$ $10$ $+$ $a_{0}$}}} $\big)$ \vspace{0.5cm}

Ahora, sumando este simétrico a la expresión {\black{{\Large$*$}}} tenemos...  \vspace{0.5cm}

{\black{$a_{0}$ $=$ $2k$ $-$ $2s$ $=$ $2(k-s)$}}  Lo cual muestra que...  
\vspace{0.5cm}

{\black{$2 \hspace{0.2cm} \bigg |\ \bigg  m $}}  \vspace{0.5cm}

\hspace{1cm} {\black{ \qedsymbol }} \vspace{1cm}
\end{MyColorPar}

%%%%%%%%%%%%%%%%%%%%%%%%%%%%%%%%%%%%%%%%%%%%%%%%%%%%%%%%%%%%%%%%%%%%%%%%%%%%%%%%%%%%%%%%%%%%%%%%%%%%%%%%%%%%%%%%%%%%%%%%%%%%%%%%%%%%%%%%%%%%%%%%%%%%%%%%%%%%%%%%%%%%%%%%%%%%%%%%%%%%%%%%%%%%%%%%%%%%%%%%%%%%%%%

\section*{$3 \hspace{0.2cm} \bigg |\ \bigg m$ $\Longleftrightarrow$ {\textsf{la suma de sus dígitos es múltiplo de 3}}  }

\begin{MyColorPar}{Tarawera}\bfseries
{\textcolor{Cinnabar}{\bfseries{Demostración directa}}}. Suponemos {\black{$m$}} es impar {\textcolor{verde_manzana}{\bfseries{($m$ $=$ $2k-1$)}}} \vspace{0.5cm}

\hspace{2cm}{\black{\fbox{$m$ $=$ $a_{n}$ $\times$ $10^{n}$ $=$ $a_{n}$ $\big($ $33$...$3$ $+$ $1$) $=$ $3$ $a_n$ $\big($ $11$...$1$ $\big)$ $+$ $a_{n}$}}} {\Large$*$} \vspace{0.5cm}

Ahora, sumando este simétrico a la expresión {\black{{\Large$*$}}} tenemos...  \vspace{0.5cm}

{\black{$m$ $=$ $2k-1$ $-$ $2s$ $=$ $2k$ $-$ $2s$ $-$ $1$ $=$ $2(k-s)-1$ $=$ $2(n)-1$}} \vspace{0.5cm} 

Lo cual muestra que...  
\vspace{0.5cm}

{\black{$3 \hspace{0.2cm} \bigg |\ \bigg  m $}}  \vspace{0.5cm}

\hspace{1cm} {\black{ \qedsymbol }} \vspace{1cm}
\end{MyColorPar}


\end{document}
