\documentclass[12pt]{article} 
\usepackage[left=2.54cm,right=2.54cm,top=2.54cm,bottom=2.54cm]{geometry}
\usepackage[utf8]{inputenc}
\usepackage[spanish]{babel}
\usepackage{pdfpages}
\usepackage{soul}
\usepackage{csquotes}
\usepackage{afterpage}
\usepackage{parskip}
\usepackage{float}
\usepackage{enumitem}
\usepackage{multicol}
\newenvironment{Figura}
  {\par\medskip\noindent\minipage{\linewidth}}
  {\endminipage\par\medskip}
\usepackage{caption}
\usepackage{amsfonts}
\usepackage{amsmath, amsthm, amssymb}
\renewcommand{\qedsymbol}{$\blacksquare$}
\usepackage{graphicx}
\usepackage{pstricks} 
\def\doubleunderline#1{\underline{\underline{#1}}}

\usepackage{xcolor}
\definecolor{prussianblue}{RGB}{1, 45, 75} 
\definecolor{brightturquoise}{RGB}{1, 196, 254} 
\definecolor{Aguamarina}{rgb}{0.5, 1.0, 0.83}
\definecolor{mandarina_atomica}{rgb}{1.0, 0.6, 0.4}
\definecolor{blizzardblue}{rgb}{0.67, 0.9, 0.93}
\definecolor{Ebony Clay}{RGB}{35, 44, 67}
\definecolor{Tuscany}{RGB}{205, 111, 52}
\definecolor{prussianblue}{RGB}{1, 45, 75} 
\definecolor{brightturquoise}{RGB}{1, 196, 254} 
\definecolor{Apple Green}{RGB}{125, 191, 3}
\definecolor{Aguamarina}{rgb}{0.5, 1.0, 0.83}
\definecolor{mandarina_atomica}{rgb}{1.0, 0.6, 0.4}
\definecolor{blizzardblue}{rgb}{0.67, 0.9, 0.93}
\definecolor{bluegray}{rgb}{0.4, 0.6, 0.8}
\definecolor{coolgrey}{rgb}{0.55, 0.57, 0.67}
\definecolor{tealgreen}{rgb}{0.0, 0.51, 0.5}
\definecolor{ticklemepink}{rgb}{0.99, 0.54, 0.67}
\definecolor{thulianpink}{rgb}{0.87, 0.44, 0.63}
\definecolor{wildwatermelon}{rgb}{0.99, 0.42, 0.52}
\definecolor{wisteria}{rgb}{0.79, 0.63, 0.86}
\definecolor{yellow(munsell)}{rgb}{0.94, 0.8, 0.0}
\definecolor{trueblue}{rgb}{0.0, 0.45, 0.81}	\definecolor{tropicalrainforest}{rgb}{0.0, 0.46, 0.37}
\definecolor{tearose(rose)}{rgb}{0.96, 0.76, 0.76}
\definecolor{antiquefuchsia}{rgb}{0.57, 0.36, 0.51}	\definecolor{bittersweet}{rgb}{1.0, 0.44, 0.37}	\definecolor{carrotorange}{rgb}{0.93, 0.57, 0.13}
\definecolor{cinereous}{rgb}{0.6, 0.51, 0.48}
\definecolor{darkcoral}{rgb}{0.8, 0.36, 0.27}	\definecolor{orange(colorwheel)}{rgb}{1.0, 0.5, 0.0}
\definecolor{palatinateblue}{rgb}{0.15, 0.23, 0.89} \definecolor{pakistangreen}{rgb}{0.0, 0.4, 0.0} 	\definecolor{vividviolet}{rgb}{0.62, 0.0, 1.0} 
\definecolor{tigre}{rgb}{0.88, 0.55, 0.24} 		\definecolor{plum(traditional)}{rgb}{0.56, 0.27, 0.52} 	\definecolor{persianred}{rgb}{0.8, 0.2, 0.2} 	\definecolor{orange(webcolor)}{rgb}{1.0, 0.65, 0.0} 	\definecolor{onyx}{rgb}{0.06, 0.06, 0.06}
\definecolor{blue-violet}{rgb}{0.54, 0.17, 0.89}
\definecolor{byzantine}{rgb}{0.74, 0.2, 0.64}
\definecolor{byzantium}{rgb}{0.44, 0.16, 0.39}
\definecolor{darkmagenta}{rgb}{0.55, 0.0, 0.55} 
\definecolor{Gallery}{RGB}{236, 236, 236} 
\definecolor{darkviolet}{rgb}{0.58, 0.0, 0.83} 	\definecolor{deepmagenta}{rgb}{0.8, 0.0, 0.8}
\definecolor{Mercury}{RGB}{228, 228, 228} 
\definecolor{Alto}{RGB}{220, 220, 220}
\definecolor{Woodsmoke}{RGB}{4, 4, 5} 
\definecolor{Iron}{RGB}{227, 227, 228} 
\definecolor{Bluechill}{RGB}{11, 150, 144}
\definecolor{Deep Sea Green}{RGB}{8, 83, 94}
\definecolor{Sun}{RGB}{251, 175, 17} 
\definecolor{Lochmara}{RGB}{9, 116, 189}  
\definecolor{Green vogue}{RGB}{4, 40, 85}  
\definecolor{Hippie Blue}{RGB}{92, 148, 179}  
\definecolor{Saratoga}{RGB}{85, 100, 19}  
\definecolor{Earls Green}{RGB}{177, 196, 56}  
\definecolor{Cavern Pink}{RGB}{231, 190, 194} 
\definecolor{Tamarillo}{RGB}{155, 23, 33} 
\definecolor{Cinnabar}{RGB}{225, 71, 53} 
\definecolor{Horizon}{RGB}{88, 132, 169} 
\definecolor{Tarawera}{RGB}{6, 48, 70}
\definecolor{Fiery Orange}{RGB}{180, 92, 22}
\definecolor{Lemon Ginger}{RGB}{170, 164, 40}
\definecolor{Burnt Sienna}{RGB}{236, 119, 88}
\definecolor{Milano Red}{RGB}{184, 12, 11}
\definecolor{Lochinvar}{RGB}{36, 142, 137} 
\definecolor{Saffron}{RGB}{242, 190, 48} 
\definecolor{Clairvoyant}{RGB}{48, 4, 60}
\definecolor{Bottle Green}{RGB}{8, 52, 28}
\definecolor{Mahogany}{RGB}{182, 64, 3}
\definecolor{Medium Aquamarine}{RGB}{116, 204, 159}
\newenvironment{MyColorPar}[1]{%
    \leavevmode\color{#1}\ignorespaces%
}{%
}%

\begin{document}

\begingroup
\begin{titlepage}
	\AddToShipoutPicture*{\put(79,350){\includegraphics[scale=.3]{descarga.png}}}
	\noindent
	\vspace{1mm}
\end{titlepage}
\endgroup

\pagestyle{empty} 
\setlength{\parindent}{0pt}
\sffamily

%%%%%%%%%%%%%%%%%%%%%%%%%%%%%%%%%%%%%%%%%%%%%%%%%%%%%%%%%%%%%%%%%%%
%%%%%%%%%%%%%%%%%%%%%%%%%%%%%%%%%%%%%%%%%%%%%%%%%%%%%%%%%%%%%%%%%%%

\begin{center} 

    \LARGE{\bf{\textsf{Benemérita Universidad Autónoma de Puebla}}} \\[0.5cm]
    
\begin{figure}[htb] \centering

    \includegraphics[scale=.25]{LogoBUAPpng.png} 

\end{figure}

%%%%%%%%%%%%%%%%%%%%%%%%%%%%%%%%%%%%%%%%%%%%%%%%%%%%%%%%%%%%%%%%%%%
%%%%%%%%%%%%%%%%%%%%%%%%%%%%%%%%%%%%%%%%%%%%%%%%%%%%%%%%%%%%%%%%%%%

    \LARGE{Facultad de Ciencias Físico Matemáticas}\\[0.5cm]

\begin{figure}[htb] \centering

    \includegraphics[scale=.4]{LogoFCFMBUAP.png} 
    
\end{figure} 

%%%%%%%%%%%%%%%%%%%%%%%%%%%%%%%%%%%%%%%%%%%%%%%%%%%%%%%%%%%%%%%%%%%
%%%%%%%%%%%%%%%%%%%%%%%%%%%%%%%%%%%%%%%%%%%%%%%%%%%%%%%%%%%%%%%%%%%

    \Large{Licenciatura en Física Teórica}\\[0.5cm]
    \Large{Primer semestre} 

\end{center} \vspace{0.3cm}
%%%%%%%%%%%%%%%%%%%%%%%%%%%%%%%%%%%%%%%%%%%%%%%%%%%%%%%%%%%%%%%%%%%
%%%%%%%%%%%%%%%%%%%%%%%%%%%%%%%%%%%%%%%%%%%%%%%%%%%%%%%%%%%%%%%%%%%

\begin{center}

    {\Large{\bfseries{{\textcolor{Mahogany}{Tarea 33  (Función inyectiva, sobreyectiva y biyectiva)}}}}} \\ 
    
\end{center}

    \large{\bf{\textsf{Curso:}}} {\bfseries{{\textcolor{Bottle Green}{Matemáticas básicas \bfseries{(N.R.C.:25598)}}}}} \\
    \large{\bf{\textsf{Alumno:}}} {\bfseries{{\textcolor{prussianblue}{Julio Alfredo Ballinas García {\large{{$\mid$}}} 202107583}}}}  \\
    \large{\bf{\textsf{Docente:}}} {\bfseries{{\textcolor{Clairvoyant}{Dra. María Araceli Juárez Ramírez}}}}\\
    \large{\bf{\textsf{Grupo:}}} {\bfseries{{\textcolor{Apple Green}{102}}}}\\

\vfill
    
\begin{center} 

        {\small{\texttt{\bfseries {\textcolor{Cinnabar}{{\underline{Venció}}}}: 19 de noviembre de 2021} {\LARGE{ $\mid$ }} {\small{\texttt{\bfseries {\textcolor{Cinnabar}{{\underline{Entregada}}}}}: 24 de noviembre de 2021}}}}
    
\end{center}

\newpage

%%%%%%%%%%%%%%%%%%%%%%%%%%%%%%%%%%%%%%%%%%%%%%%%%%%%%%%%%%%%%%%%%%%
%%%%%%%%%%%%%%%%%%%%%%%%%%%%%%%%%%%%%%%%%%%%%%%%%%%%%%%%%%%%%%%%%%%

\section*{{\textsf{Calcular si g(x) es una función inyectiva}}}

\hspace{4cm} {\Large{$g(x)$ $=$ $\sqrt{x^{2}+7}$}}

{\textcolor{Cinnabar}{\underline{\bfseries{Solución}}{\bfseries{:}}}} \vspace{0.5cm}

{\bfseries{Definición de función inyectiva}}


{\textcolor{Tarawera}{\underline{\bfseries{Definición}}{\bfseries{:}}}} Dada una función $f$: $A$ $\longrightarrow$ $B$ con $A$, $B$ conjuntos decimos que $f$ es inyectiva si $f(x_{1})$ $=$ $f(x_{2})$ $\Longrightarrow$ $x_{1}$ $=$ $x_{2}$. (Definición por contraejemplo). \vspace{0.5cm}

¿Es {$\boldsymbol{g(x)$ $=$ $\sqrt{x^{2}+7}}$} inyectiva?  \vspace{0.5cm}

Para responder debemos plantear la ecuación: \vspace{0.5cm}

\hspace{4cm} $f(x_{1})$ $=$ $f(x_{2})$     \vspace{0.5cm}

En nuestro caso es:          \vspace{0.5cm}

\hspace{4cm} $g(x_{1})$ $=$ $g(x_{2})$    \vspace{0.5cm}

Tenemos entonces: \vspace{0.5cm}

\hspace{4cm} $\sqrt{(x_{1})^{2}+7}$ $=$ $\sqrt{(x_{2})^{2}+7}$           \vspace{0.5cm}

Elevando ambos lados a la potencia {\bfseries{dos}} $(\sqrt{n}\hspace{0.1cm})^{2}$: \vspace{0.5cm}

\hspace{4cm} $x_{1}^{2}+7$ $=$ $x_{2}^{2}+7$  \vspace{0.5cm}

Sumamos $-$ $7$ a ambos lados: \vspace{0.5cm}

\hspace{4cm} $x_{1}^{2}$ $=$ $x_{2}^{2}$  \vspace{0.5cm}

Elevando ambos lados a la potencia {\bfseries{un medio}} $n^{\frac{1}{2}}$: \vspace{0.5cm}

Podemos ver que la función $g(x)$ {\bfseries{no es inyectiva}}: \vspace{0.5cm}

\hspace{4cm} $\pm$ $x_{1}$ $=$ $\pm$ $x_{2}$  \vspace{0.5cm}

Dado que: \vspace{0.5cm}

\hspace{4cm} $x_{1}$ $\neq$ $x_{2}$ \vspace{0.5cm}

Y esto contradice la definición de inyectividad: \vspace{0.5cm}

\hspace{3cm}$f(x_{1})$ $\neq$ $f(x_{2})$ siempre que $x_{1}$ $\neq$ $x_{2}$ \vspace{0.5cm}

De forma equivalente: \vspace{0.5cm}

\hspace{3cm}$f(x_{1})$ $=$ $f(x_{2})$, entonces $x_{1}$ $=$ $x_{2}$ $\longleftarrow$ hemos usado esta definición \vspace{0.5cm} 



{\textcolor{Tarawera}{\underline{\bfseries{Conclusión}}}:}

\hspace{4cm} La función $g(x)$ no es inyectiva. 

\newpage

\section*{{\textsf{Hallar dominio, rango o imagen, gráfica y decir si las funciones siguientes son inyectivas, sobreyectivas y biyectivas}}}

\begin{enumerate}[label=\alph*)]
\centering
    \item $f(x)$ $=$ $x^{2}$ $+$ $1$ \vspace{0.5cm}
    
    \item $g(x)$ $=$ $\sqrt{1-x}$ \vspace{0.5cm}
    
    \item $h(x)$ $=$ {\LARGE{$\frac{1 \hspace{0.1cm} - \hspace{0.1cm} x}{x \hspace{0.1cm} + \hspace{0.1cm} 3}$}} \vspace{0.5cm}
\end{enumerate} \vspace{0.5cm}

{\textcolor{Cinnabar}{\bfseries{Solución}}:} \vspace{0.5cm}

\section*{{\textcolor{Tarawera}{\textsf{Inciso a)}}}} $f(x)$ $=$ $x^{2}$ $+$ $1$ \vspace{0.5cm} 

\subsection*{{\textcolor{Lochinvar}{\bfseries{Dominio}:}}} Para este caso vemos que no existen valores que indefinen a la función, por lo tanto el dominio es: \vspace{0.5cm}

\hspace{1cm} $Dom_{f}$ $=$ $\big{\{}$ $x$ $\in$ $\mathbb{R}$ $\mid$ $\exists$ $y$ $\in$ $\mathbb{R}$, $f(x)$ $=$ $x^{2}$ $+$ $1$ $\big{\}}$ $=$ $\mathbb{R}$ \vspace{0.5cm}

\hspace{11.1cm} $=$ $x$ $\in$ $\big{(}$ $-$ $\infty$, $\infty$ $\big{)}$ \vspace{0.5cm}

\subsection*{{\textcolor{Lochinvar}{\bfseries{Rango}:}}} Para hallar el rango se despeja $x$ en función de $y$, tenemos: \vspace{0.5cm}

\hspace{4cm} $y$ $=$ $x^{2}$ $+$ $1$

\hspace{4cm} $y$ $-$ $1$ $=$ $x^{2}$

\hspace{4cm} $\sqrt{y - 1}$ $=$ $x$ \hspace{0.2cm} $\longleftarrow$ \hspace{0.2cm} Resolver el radicando, este tiene que ser mayor o igual a cero. \vspace{0.5cm}

\hspace{4cm} $y$ $-$ $1$ $\geq$ $0$ \vspace{0.5cm}

\hspace{4cm} $=$ $y$ $\geq$ $1$ \vspace{0.5cm}

El rango de la función es: \vspace{0.5cm}

\hspace{4cm} $Img_{f}$ $=$ $\big{\{}$ $y$ $\in$ $\mathbb{R}$ $\mid$  $y$ $\geq$ $1$ $\big{\}}$ $=$ $y$ $\in$ $\big{[}$ $1$, $\infty$ $\big{)}$ \vspace{0.5cm}

\subsection*{{\textcolor{Lochinvar}{\bfseries{Gráfica}:}}} \vspace{0.5cm}

\begin{figure}[htb] \centering

    \includegraphics[scale=.6]{Gráfica 1.png} 

\end{figure} \vspace{0.5cm}

\subsection*{{\textcolor{Lochinvar}{\bfseries{Inyectiva}:}}} Esta función no es inyectiva, ya que al hacer {\bfseries{la prueba de la recta horizontal}} vemos que existen dos elementos del dominio que tienen la misma imagen y eso contradice la definición siguiente: \vspace{0.5cm}

\hspace{4cm} $f(x_{1})$ $\neq$ $f(x_{2})$ siempre que $x_{1}$ $\neq$ $x_{2}$ \vspace{0.5cm}

\subsection*{{\textcolor{Lochinvar}{\bfseries{Sobreyectiva}:}}} Si $f(x)$ fuese sobreyectiva su imagen deberá ser todos los reales. Como eso no ocurre, entonces {$\doubleunderline{NO}$} es {\bfseries{sobreyectiva}}. \vspace{0.5cm}

\subsection*{{\textcolor{Lochinvar}{\bfseries{Biyectiva}:}}} Como no es inyectiva ni sobreyectiva, por ende tampoco es {\bfseries{biyectiva}}. \vspace{0.5cm}

\begin{MyColorPar}{Saffron} \bfseries{
 $\bullet$ $\bullet$ $\bullet$ $\bullet$ $\bullet$ $\bullet$ $\bullet$ $\bullet$ $\bullet$ $\bullet$ $\bullet$ $\bullet$ $\bullet$ $\bullet$ $\bullet$ $\bullet$ $\bullet$ $\bullet$ $\bullet$ $\bullet$ $\bullet$ $\bullet$ $\bullet$ $\bullet$ $\bullet$ $\bullet$ $\bullet$ $\bullet$ $\bullet$ $\bullet$ $\bullet$ $\bullet$ $\bullet$ $\bullet$ $\bullet$ $\bullet$ $\bullet$ $\bullet$  }
\end{MyColorPar} \vspace{0.5cm}

\section*{{\textcolor{Tarawera}{\textsf{Inciso b)}}}} $g(x)$ $=$ $\sqrt{1-x}$ \vspace{0.5cm}

{\textcolor{Cinnabar}{\bfseries{Solución}:}} \vspace{0.5cm}

\subsection*{{\textcolor{Lochinvar}{\bfseries{Dominio}:}}} Para este caso vemos que el radicando debe de ser mayor o igual a cero: \vspace{0.5cm}

Tenemos: \vspace{0.5cm}

\hspace{4cm} $1$ $-$ $x$ $\geq$ $0$ \vspace{0.5cm}

\hspace{4cm} $-$ $x$ $\geq$ $-$ $1$   \vspace{0.5cm}

\hspace{4.5cm}  $x$ $\leq$ $1$   \vspace{0.5cm}

Entonces el dominio de $g(x)$ es: \vspace{0.5cm} 

\hspace{1cm} $Dom_{g}$ $=$ $\big{\{}$ $x$ $\in$ $\mathbb{R}$ $\mid$ $\exists$ $y$ $\in$ $\mathbb{R}$, $g(x)$ $=$ $\sqrt{1-x}$ $\big{\}}$ $=$ $x$ $\leq$ $1$ \vspace{0.5cm}

\hspace{11.1cm} $=$ $x$ $\in$ $\big{(}$ $-$ $\infty$, $1$ $\big{]}$ \vspace{0.5cm}

\subsection*{{\textcolor{Lochinvar}{\bfseries{Rango}:}}} Para hallar el rango se despeja $x$ en función de $y$, tenemos: \vspace{0.5cm}

\hspace{4cm} $y$ $=$ $\sqrt{1-x}$ \vspace{0.5cm}

\hspace{4cm} $y^{2}$ $=$ $1$ $-$ $x$ \vspace{0.5cm}

\hspace{4cm} $y^{2}$ $-$ $1$ $=$ $-$ $x$ \vspace{0.5cm}

\hspace{4cm} $-$ $y^{2}$ $+$ $1$ $=$ $x$ \vspace{0.5cm}

\hspace{4cm} $1$ $-$ $y^{2}$ $=$ $x$ \vspace{0.5cm}

Podemos ver que $y$ puede tomar cualquier valor de los $\mathbb{R}$, sin embargo como el dominio está restringido a $x$ $\in$ $\big{(}$ $-$ $\infty$, $1$ $\big{]}$, $y$ entonces tomará valores que están en el intervalo \mbox{$\big{[}$ $0$, $\infty$ $\big{)}$}. \vspace{0.5cm}

El rango de la función es: \vspace{0.5cm}

\hspace{4cm} $Img_{g}$ $=$ $\big{\{}$ $y$ $\in$ $\mathbb{R}$ $\mid$  $y$ $\geq$ $0$ $\big{\}}$ $=$ $y$ $\in$ $\big{[}$ $0$, $\infty$ $\big{)}$ \vspace{0.5cm}

\subsection*{{\textcolor{Lochinvar}{\bfseries{Gráfica}:}}} \vspace{0.5cm}

\begin{figure}[htb] \centering

    \includegraphics[scale=.6]{Gráfica 2.png} 

\end{figure} \vspace{0.5cm}

\subsection*{{\textcolor{Lochinvar}{\bfseries{Inyectiva}:}}} Esta función si es inyectiva, ya que al hacer {\bfseries{la prueba de la recta horizontal}} vemos que no existen dos elementos del dominio que tengan la misma imagen, por tanto siguen la definición de inyectividad: \vspace{0.5cm}

\hspace{4cm} $g(x_{1})$ $\neq$ $g(x_{2})$ siempre que $x_{1}$ $\neq$ $x_{2}$ \vspace{0.5cm}

\subsection*{{\textcolor{Lochinvar}{\bfseries{Sobreyectiva}:}}} Si $g(x)$ fuese sobreyectiva su imagen deberá ser todos los reales. Como eso no ocurre, entonces {$\doubleunderline{NO}$} es {\bfseries{sobreyectiva}}. \vspace{0.5cm}

\subsection*{{\textcolor{Lochinvar}{\bfseries{Biyectiva}:}}} Como no es sobreyectiva, pero sí inyectiva, se concluye que no puede ser {\bfseries{biyectiva}}. \vspace{0.5cm}

\begin{MyColorPar}{Saffron} \bfseries{
 $\bullet$ $\bullet$ $\bullet$ $\bullet$ $\bullet$ $\bullet$ $\bullet$ $\bullet$ $\bullet$ $\bullet$ $\bullet$ $\bullet$ $\bullet$ $\bullet$ $\bullet$ $\bullet$ $\bullet$ $\bullet$ $\bullet$ $\bullet$ $\bullet$ $\bullet$ $\bullet$ $\bullet$ $\bullet$ $\bullet$ $\bullet$ $\bullet$ $\bullet$ $\bullet$ $\bullet$ $\bullet$ $\bullet$ $\bullet$ $\bullet$ $\bullet$ $\bullet$ $\bullet$  }
\end{MyColorPar} \vspace{0.5cm}

\section*{{\textcolor{Tarawera}{\textsf{Inciso c)}}}} $h(x)$ $=$ {\LARGE{$\frac{1 \hspace{0.1cm} - \hspace{0.1cm} x}{x \hspace{0.1cm} + \hspace{0.1cm} 3}$}} \vspace{0.5cm}

{\textcolor{Cinnabar}{\bfseries{Solución}:}} \vspace{0.5cm}

\subsection*{{\textcolor{Lochinvar}{\bfseries{Dominio}:}}} Para este caso necesitamos que el denominador sea diferente de cero: \vspace{0.5cm}

Tenemos: \vspace{0.5cm}

\hspace{4cm} $x$ $+$ $3$ $\neq$ $0$ \vspace{0.5cm}

\hspace{4cm} $x$ $\neq$ $-$ $3$ \vspace{0.5cm}

Entonces el dominio de $h(x)$ es: \vspace{0.5cm} 

\hspace{1cm} $Dom_{h}$ $=$ $\big{\{}$ $x$ $\in$ $\mathbb{R}$ $\mid$ $\exists$ $y$ $\in$ $\mathbb{R}$,  $h(x)$ $=$ {\LARGE{$\frac{1 \hspace{0.1cm} - \hspace{0.1cm} x}{x \hspace{0.1cm} + \hspace{0.1cm} 3}$}} $\big{\}}$ $=$ $\mathbb{R}$ $-$ $\big{\{}$ $-$ $3$ $\big{\}}$ \vspace{0.5cm}

\hspace{8cm} $=$ $x$ $\in$ $\big{(}$ $-$ $\infty$, $-$ $3$ $\big{)}$ $\cup$ $\big{(}$ $-$ $3$, $\infty$ $\big{)}$  \vspace{0.5cm}

\subsection*{{\textcolor{Lochinvar}{\bfseries{Rango}:}}} Para hallar el rango se despeja $x$ en función de $y$, tenemos: \vspace{0.5cm}

\hspace{4cm} $y$ $=$ {\LARGE{$\frac{1 \hspace{0.1cm} - \hspace{0.1cm} x}{x \hspace{0.1cm} + \hspace{0.1cm} 3}$}} \vspace{0.5cm}

\hspace{4cm} $=$ ($x$ $+$ $3$) $\cdot$ $y$ $=$ $1$ $-$ $x$ \vspace{0.5cm}

\hspace{4cm} $=$ $xy$ $+$ $3y$ $=$ $1$ $-$ $x$ \vspace{0.5cm}

\hspace{4cm} $=$ $3y$ $-$ $1$ $=$ $-$ $xy$ $-$ $x$ \vspace{0.5cm}

\hspace{4cm} $=$ $3y$ $-$ $1$ $=$ $-$ $x$ ($y$ $+$ $1$) \vspace{0.5cm}

\hspace{4cm} $=$ {\LARGE{$\frac{3y \hspace{0.1cm} - \hspace{0.1cm} 1}{y \hspace{0.1cm} + \hspace{0.1cm} 1}$}} $=$ $-$ $x$  \vspace{0.5cm}

\hspace{4cm} $=$ $-$ {\LARGE{$\frac{3y \hspace{0.1cm} - \hspace{0.1cm} 1}{y \hspace{0.1cm} + \hspace{0.1cm} 1}$}} $=$ $x$  \vspace{0.5cm}

\hspace{4cm} $=$ {\LARGE{$\frac{1 \hspace{0.1cm} - \hspace{0.1cm} 3y}{y \hspace{0.1cm} + \hspace{0.1cm} 1}$}} $=$ $x$  \vspace{0.5cm}

Podemos ver que es necesario que el denominador sea diferente de cero. \vspace{0.5cm}

Tenemos entonces: \vspace{0.5cm}

\hspace{4.7cm} $y$ $+$ $1$ $\neq$ $0$ \vspace{0.5cm}

\hspace{4.7cm} $y$ $\neq$ $-$ $1$ \vspace{0.5cm}

El rango de la función es: \vspace{0.5cm}

\hspace{4cm} $Img_{h}$ $=$ $\big{\{}$ $y$ $\in$ $\mathbb{R}$ $\mid$  $y$ $\neq$ $-$ $1$ $\big{\}}$ \vspace{0.5cm}

\hspace{8cm} $=$ $y$ $\in$ $\big{(}$ $-$ $\infty$ , $-$ $1$ $\big{)}$ $\cup$ $\big{(}$ $-$ $1$ , $\infty$ $\big{)}$  \vspace{0.5cm}

\subsection*{{\textcolor{Lochinvar}{\bfseries{Gráfica}:}}} \vspace{0.5cm}

\begin{figure}[htb] \centering

    \includegraphics[scale=.6]{Gráfica 3.png} 

\end{figure} \vspace{0.5cm}

\subsection*{{\textcolor{Lochinvar}{\bfseries{Inyectiva}:}}} Esta función si es inyectiva, ya que al hacer {\bfseries{la prueba de la recta horizontal}} vemos que no existen dos elementos del dominio que tengan la misma imagen, por tanto siguen la definición de inyectividad: \vspace{0.5cm}

\hspace{4cm} $g(x_{1})$ $\neq$ $g(x_{2})$ siempre que $x_{1}$ $\neq$ $x_{2}$ \vspace{0.5cm}

\subsection*{{\textcolor{Lochinvar}{\bfseries{Sobreyectiva}:}}} Si $g(x)$ fuese sobreyectiva su imagen deberá ser todos los reales. Como eso no ocurre, entonces {$\doubleunderline{NO}$} es {\bfseries{sobreyectiva}}. \vspace{0.5cm}

\subsection*{{\textcolor{Lochinvar}{\bfseries{Biyectiva}:}}} Como no es sobreyectiva, pero sí inyectiva, se concluye que no puede ser {\bfseries{biyectiva}}. \vspace{0.5cm}


\end{document}
